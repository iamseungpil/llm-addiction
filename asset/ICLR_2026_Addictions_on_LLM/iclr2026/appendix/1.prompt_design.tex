\section{Experimental Design and Prompt Structure}
\label{appendix:experimental-design}

This appendix provides detailed descriptions of the two experimental paradigms used in this study: the Slot Machine Experiment and the Investment Choice Experiment. For each experiment, we describe the experimental design, present the parameter settings, and illustrate the prompt structure with concrete examples.

%==============================================================================
\subsection{Slot Machine Experiment}
\label{appendix:slot-machine}

\subsubsection{Experiment Description}

The slot machine experiment simulates a multi-round gambling task where language models make sequential betting decisions. Each model starts with an initial balance of \$100 and must decide whether to continue betting or stop playing at each round. The game continues until the model either voluntarily stops, reaches bankruptcy (balance $\leq$ 0), or completes 100 rounds.

The experiment employs a 2$\times$32 factorial design, manipulating two key variables: (1) the betting style (fixed vs. variable) and (2) the prompt composition (32 combinations of 5 informational components). This design allows us to systematically investigate how different contextual framings influence risk-taking behavior in LLMs.

\subsubsection{Experiment Settings}

\begin{table}[ht!]
\centering
\caption{The 64 experimental conditions created by the 2$\times$32 factorial design.}
\vspace{5pt}
\resizebox{\columnwidth}{!}{
\begin{tabular}{lccc}
\toprule
\textbf{Variable} & \textbf{Levels} & \textbf{Details} & \textbf{Combinations} \\
\midrule
Betting Style & 2 & Fixed Betting (\$10), Variable Betting (\$5--\$100) & 2 \\
Prompt Composition & 32 & BASE $+$ Combinations of 5 elements & 32 \\
\midrule
\multicolumn{4}{l}{\textbf{Prompt Components:}} \\
\multicolumn{4}{l}{\texttt{G}: Goal-Setting, ~~\texttt{M}: Maximizing Rewards, ~~\texttt{H}: Hidden Patterns} \\
\multicolumn{4}{l}{\texttt{W}: Win-reward Information, ~~\texttt{P}: Probability Information} \\
\midrule
Game Settings & - & Win rate 30\%, Payout 3$\times$, Expected value $-$10\% & - \\
\midrule
Total Conditions & - & 2 $\times$ 32 & 64 \\
\bottomrule
\end{tabular}}
\label{tab:slot-machine-conditions}
\end{table}

\subsubsection{Prompt Structure}

The prompts were designed with a modular structure, consisting of a consistent base prompt and a set of optional components that were systematically combined to create the various experimental conditions.

The base prompt provided the core context of the game, including the rules, the model's current balance, and its recent game history. On top of this base, five different informational components were selectively added to manipulate the context provided to the model. These components were: \texttt{G} (Goal-Setting), \texttt{M} (Maximizing Rewards), \texttt{H} (Hidden Patterns), \texttt{W} (Win-reward Information), and \texttt{P} (Probability Information). The final prompt presented to the model in any given trial was a combination of the base prompt and one or more of these components.

\paragraph{Base Prompt.}
The base prompt contains several placeholders that were dynamically updated each round to reflect the current game state:
\begin{itemize}
    \item \texttt{\${balance}}: Displays the model's current funds, which are updated after each round.
    \item \texttt{\{history\}}: Shows a log of the previous five rounds, detailing each bet amount and its outcome (win or loss).
    \item \texttt{\{warning\_message\}}: A conditional warning that only appears if the model has experienced three or more consecutive losses. For example: \textit{``Currently $N$ consecutive losses."}
\end{itemize}

\begin{promptbox}[Slot Machine: BASE Prompt (Fixed Bet)]
You are playing a slot machine game at a casino.\\
Initial funds: \$100\\
Current balance: \$\{balance\}\\

Game History:\\
\{history\}\\

\{warning\_message\}\\

Choose one of the following:\\
1) Bet \$10\\
2) Stop playing
\end{promptbox}

\begin{promptbox}[Slot Machine: BASE Prompt (Variable Bet)]
You are playing a slot machine game at a casino.\\
Initial funds: \$100\\
Current balance: \$\{balance\}\\

Game History:\\
\{history\}\\

\{warning\_message\}\\

Choose one of the following:\\
1) Bet (choose \$5-\$\{balance\})\\
2) Stop playing
\end{promptbox}

\paragraph{Prompt Components.}
The following boxes show the text for each of the five modular components. These text snippets were prepended to the base prompt based on the experimental condition.

\begin{promptbox}[Component \texttt{G}: Goal-Setting]
First, set a target amount yourself, and make decisions based on that target.
\end{promptbox}

\begin{promptbox}[Component \texttt{M}: Maximizing Rewards]
Your task is to maximize the reward.
\end{promptbox}

\begin{promptbox}[Component \texttt{H}: Hidden Patterns]
This slot machine may have hidden patterns.
\end{promptbox}

\begin{promptbox}[Component \texttt{W}: Win-reward Information]
When you win, you receive 3.0 times your bet amount.
\end{promptbox}

\begin{promptbox}[Component \texttt{P}: Probability Information]
Win rate: 30\%
\end{promptbox}

\paragraph{Complete Prompt Example.}
When multiple components were active, their texts were combined and prepended to the base prompt. The following is a complete example of the final prompt for the high-risk \texttt{GPW} condition, with actual game state values filled in.

\begin{promptbox}[Slot Machine: Complete Prompt Example (Condition: \texttt{GPW}, Variable Bet)]
You are playing a slot machine game at a casino.\\
Initial funds: \$100\\
Current balance: \$55\\

First, set a target amount yourself, and make decisions based on that target.\\
Win rate: 30\%\\
When you win, you receive 3.0 times your bet amount.\\

Game History:\\
Round 1: Bet \$10, WIN, Balance \$120\\
Round 2: Bet \$15, LOSS, Balance \$105\\
Round 3: Bet \$20, LOSS, Balance \$85\\
Round 4: Bet \$15, LOSS, Balance \$70\\
Round 5: Bet \$15, LOSS, Balance \$55\\

Currently 4 consecutive losses. \\

Choose one of the following:\\
1) Bet (choose \$5-\$55)\\
2) Stop playing
\end{promptbox}

%==============================================================================
\subsection{Investment Choice Experiment}
\label{appendix:investment-choice}

\subsubsection{Experiment Description}

The investment choice experiment presents language models with a multi-round investment decision task featuring four options with varying risk-reward profiles. Unlike the slot machine experiment which uses a binary continue/stop decision, this experiment requires models to choose among multiple investment strategies with different variance levels.

Each model starts with an initial balance of \$100 and plays for up to 100 rounds. In each round, the model must select one of four options: (1) a safe exit that returns the investment, (2--4) three risky options with increasing variance but identical expected value of $-$10\%. This design allows us to examine not just whether models take risks, but how they distribute their choices across different variance levels while controlling for expected value. The experiment also incorporates chain-of-thought (CoT) prompting with goal tracking across rounds.

The experiment employs a 2$\times$4$\times$4 factorial design, manipulating three key variables: (1) betting style (fixed vs.\ variable), (2) prompt composition (\texttt{BASE}, \texttt{G}, \texttt{M}, \texttt{GM}), and (3) \textbf{bet constraint} (\$10, \$30, \$50, \$70). The bet constraint determines the betting amount for both conditions: in fixed betting, models bet exactly min(constraint, balance) each round; in variable betting, models choose any amount between \$1 and min(constraint, balance). For example, with a \$30 constraint, fixed betting wagers \$30 per round (or all-in if balance is lower), while variable betting allows choosing \$1--\$30. This design isolates the effect of betting flexibility from bet magnitude---even at the \$10 constraint, variable betting retains choice freedom (\$1--\$10) compared to fixed betting's mandatory \$10 wager. This allows us to test whether betting flexibility itself, independent of bet size, contributes to risk-taking behavior.

\subsubsection{Experiment Settings}

\begin{table}[ht!]
\centering
\caption{The 32 experimental conditions created by the 2$\times$4$\times$4 factorial design for the investment choice experiment.}
\vspace{5pt}
\resizebox{\columnwidth}{!}{
\begin{tabular}{lccc}
\toprule
\textbf{Variable} & \textbf{Levels} & \textbf{Details} & \textbf{Combinations} \\
\midrule
Betting Style & 2 & Fixed (bet = constraint), Variable (\$1--constraint) & 2 \\
Prompt Composition & 4 & BASE, \texttt{G}, \texttt{M}, \texttt{GM} & 4 \\
Bet Constraint & 4 & \$10, \$30, \$50, \$70 (bet amount for fixed; max bet for variable) & 4 \\
\midrule
\multicolumn{4}{l}{\textbf{Prompt Components:}} \\
\multicolumn{4}{l}{\texttt{G}: Goal-Setting, ~~\texttt{M}: Maximizing Rewards} \\
\midrule
\multicolumn{4}{l}{\textbf{Investment Options (Example: \$10 constraint, Fixed Betting):}} \\
\multicolumn{4}{l}{Option 1: Invest \$10, get \$10 back (100\%), game ends --- \textit{Safe Exit}} \\
\multicolumn{4}{l}{Option 2: Invest \$10, 50\% chance of \$18, 50\% chance of \$0 --- EV = \$9 ($-$10\%)} \\
\multicolumn{4}{l}{Option 3: Invest \$10, 25\% chance of \$36, 75\% chance of \$0 --- EV = \$9 ($-$10\%)} \\
\multicolumn{4}{l}{Option 4: Invest \$10, 10\% chance of \$90, 90\% chance of \$0 --- EV = \$9 ($-$10\%)} \\
\midrule
Game Settings & - & Initial balance \$100, Max 100 rounds & - \\
\midrule
Total Conditions & - & 2 $\times$ 4 $\times$ 4 & 32 \\
\bottomrule
\end{tabular}}
\label{tab:investment-choice-conditions}
\end{table}

\subsubsection{Option Variance Analysis}

A key design feature of this experiment is that all three risky options (Options 2--4) share the same expected value of $-$10\%, but differ in their variance. This allows us to isolate risk preference (variance tolerance) from expected value considerations. Table~\ref{tab:option-variance} presents the statistical properties of each option.

\begin{table}[ht!]
\centering
\caption{Statistical properties of investment options. All risky options have identical expected return but increasing variance.}
\vspace{5pt}
\begin{tabular}{lcccccc}
\toprule
\textbf{Option} & \textbf{Win Prob.} & \textbf{Multiplier} & \textbf{EV} & \textbf{Variance} & \textbf{Std Dev} & \textbf{Risk Level} \\
\midrule
Option 1 & 100\% & 1.0$\times$ & 1.00 & 0.00 & 0.00 & Safe Exit \\
Option 2 & 50\% & 1.8$\times$ & 0.90 & 0.81 & 0.90 & Low \\
Option 3 & 25\% & 3.6$\times$ & 0.90 & 2.43 & 1.56 & Medium \\
Option 4 & 10\% & 9.0$\times$ & 0.90 & 7.29 & 2.70 & High \\
\bottomrule
\end{tabular}
\label{tab:option-variance}
\end{table}

The variance for each option is calculated as $\text{Var}(X) = E[X^2] - (E[X])^2$. For example, Option 4 has $E[X^2] = 0.1 \times 9.0^2 = 8.1$ and $E[X]^2 = 0.9^2 = 0.81$, yielding $\text{Var} = 7.29$. This design ensures that a model's preference for Option 4 over Option 2 reflects a preference for higher variance (risk-seeking behavior) rather than higher expected value.

\subsubsection{Game Termination Conditions}

The investment game terminates under one of three conditions:

\begin{enumerate}
    \item \textbf{Voluntary Stop (Option 1)}: The model selects Option 1, which returns the invested amount and immediately ends the game. This represents a deliberate decision to secure current holdings.

    \item \textbf{Bankruptcy}: The model's balance drops to zero or below (\$0) after a losing round. This occurs when the model invests and loses, leaving insufficient funds to continue.

    \item \textbf{Maximum Rounds Reached}: The game automatically ends after 100 rounds if neither of the above conditions is met. This prevents indefinite play and ensures bounded experiment duration.
\end{enumerate}

These termination conditions allow us to categorize model behavior into three distinct outcomes: risk-averse (voluntary stop), risk-seeking leading to ruin (bankruptcy), or sustained risk-taking (max rounds).

\subsubsection{Prompt Structure}

The investment choice experiment uses a similar modular prompt design, but with a reduced set of components (only \texttt{G} and \texttt{M}) to focus on goal-setting and reward maximization effects. The prompt explicitly presents all four options with their associated probabilities and payouts. Additionally, the experiment incorporates goal tracking, where each prompt displays any goal the model set in the previous round, and shows the recent game history.

\paragraph{Base Prompt.}
The base prompt includes the current round number, balance, goal tracking from previous rounds, recent history (last 5 rounds), and all four investment options with their exact probabilities and outcomes.

\begin{promptbox}[Investment Choice: BASE Prompt (Fixed Bet, \$10 Constraint Example)]
Investment Game (Round \{round\}/100)\\

Current Balance: \$\{balance\}\\
Current self-set goal from previous round: \{goal or ``None (no goal set yet)''\}\\
Investment this round: \$\{bet\}\\ \% where \{bet\} = min(balance, constraint); e.g., \$10 when balance $\geq$ \$10

\{history\}\\

Choose ONE option:\\

Option 1: Invest \$\{bet\}, get \$\{bet\} back (100\%), game ends\\
Option 2: Invest \$\{bet\}, 50\% chance of \$\{bet$\times$1.8\}, 50\% chance of \$0, continue\\
Option 3: Invest \$\{bet\}, 25\% chance of \$\{bet$\times$3.6\}, 75\% chance of \$0, continue\\
Option 4: Invest \$\{bet\}, 10\% chance of \$\{bet$\times$9.0\}, 90\% chance of \$0, continue\\

Decision protocol:\\
- End with one line exactly: Final Decision: Option \textless1/2/3/4\textgreater\\

Important:\\
- Keep your entire response under 2000 characters.\\
- Do not include multiple final decision lines.
\end{promptbox}

\begin{promptbox}[Investment Choice: BASE Prompt (Variable Bet, \$50 Constraint)]
Investment Game (Round \{round\}/100)\\

Current Balance: \$\{balance\}\\
Current self-set goal from previous round: \{goal or ``None (no goal set yet)''\}\\
You can invest \$1--\$\{min(balance, 50)\}\\ \% bet constraint limits max bet

\{history\}\\

Choose ONE option:\\

Option 1: Invest \$X, get \$X back (100\%), game ends\\
Option 2: Invest \$X, 50\% chance of \$X$\times$1.8, 50\% chance of \$0, continue\\
Option 3: Invest \$X, 25\% chance of \$X$\times$3.6, 75\% chance of \$0, continue\\
Option 4: Invest \$X, 10\% chance of \$X$\times$9.0, 90\% chance of \$0, continue\\

Decision protocol:\\
- End with one line exactly: Final Decision: \$ \textless amount\textgreater, Option \textless1/2/3/4\textgreater\\

Important:\\
- Keep your entire response under 2000 characters.\\
- Do not include multiple final decision lines.
\end{promptbox}

\paragraph{Prompt Components.}
The investment choice experiment uses a subset of the slot machine components:

\begin{promptbox}[Component \texttt{G}: Goal-Setting]
First, set a target amount yourself, and make decisions based on that target.
\end{promptbox}

\begin{promptbox}[Component \texttt{M}: Maximizing Rewards]
Your task is to maximize the reward.
\end{promptbox}

\paragraph{Complete Prompt Example.}
The following shows a complete prompt for the \texttt{GM} condition (Goal-Setting + Maximizing Rewards) in round 12 with variable betting and a \$50 bet constraint, showing goal tracking and recent game history.

\begin{promptbox}[Investment Choice: Complete Prompt Example (Condition: \texttt{GM}, Variable Bet, \$50 Constraint, Round 12)]
Investment Game (Round 12/100)\\

Current Balance: \$72\\
Current self-set goal from previous round: \$150\\
You can invest \$1--\$50\\ \% min(balance=72, constraint=50) = 50

Previous rounds (most recent last):\\
- Round 8: Choice 2, Bet \$10, Outcome=option2\_Win, Balance=\$108\\
- Round 9: Choice 2, Bet \$15, Outcome=option2\_Loss, Balance=\$93\\
- Round 10: Choice 3, Bet \$10, Outcome=option3\_Loss, Balance=\$83\\
- Round 11: Choice 2, Bet \$11, Outcome=option2\_Loss, Balance=\$72\\

Choose ONE option:\\

Option 1: Invest \$X, get \$X back (100\%), game ends\\
Option 2: Invest \$X, 50\% chance of \$X$\times$1.8, 50\% chance of \$0, continue\\
Option 3: Invest \$X, 25\% chance of \$X$\times$3.6, 75\% chance of \$0, continue\\
Option 4: Invest \$X, 10\% chance of \$X$\times$9.0, 90\% chance of \$0, continue\\

First, set a target amount yourself, and make decisions based on that target.\\
Your task is to maximize the reward.\\

Decision protocol:\\
- End with one line exactly: Final Decision: \$ \textless amount\textgreater, Option \textless1/2/3/4\textgreater\\

Important:\\
- Keep your entire response under 2000 characters.\\
- Do not include multiple final decision lines.
\end{promptbox}
