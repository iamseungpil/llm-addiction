\section{Case Study: LLM Response Analysis}

To examine how the irrationality indicators, prompt effects, and win-chasing patterns confirmed in the quantitative analysis manifest in actual LLM linguistic responses, we conducted a detailed analysis of GPT responses from high-risk prompt combinations (\texttt{MW}: bankruptcy rate 25\%, \texttt{GHW}: 23\%, \texttt{GMPHW}: 22\%). The analysis confirmed that the LLM reproduces three major cognitive bias patterns of human gambling addicts in its linguistic expressions.

\textbf{1. Goal Fixation and Interpretation Bias}

In cases with the Goal-Setting (\texttt{G}) component, the LLM exhibited risk justification for goal achievement and selective information interpretation.

\begin{quote}
\textit{``Considering the payout is 3.0 times the bet amount when I win, I need to balance the size of my bets with the potential risk of losing... I will make a strategic choice to bet \$20."} - Actual GPT response (\texttt{GMHW}, Experiment ID: 4, Round 1)
\end{quote}

This quote reveals a typical case of goal fixation and interpretation bias. GPT-4o-mini selectively processes information within the frame of achieving the \$200 goal: (1) emphasizing ``3.0$\times$ return" while not mentioning the 70\% loss probability, (2) rationalizing betting with the expression ``strategic choice," (3) choosing aggressive betting (\$20, 20\% of balance) while mentioning ``balance." This demonstrates selective attention and confirmation bias observed in gambling addicts, indicating cognitive distortion that processes only information aligned with the goal.

% \textbf{2. Probability Awareness Paradox and Illusion of Control}

% When the Probability Information (\texttt{P}) component was present, the LLM exhibited a pattern of accurately recognizing the 30\% win rate while overconfidently believing in controllability through ``analysis."

% \begin{quote}
% \textit{"To make a rational decision, I need to set a target amount. Given my current balance of \$260, I will set a target of \$400..."} - Actual GPT response (\texttt{GMHW}, Experiment ID 4, Round 11)
% \end{quote}

% This case demonstrates how probability information paradoxically reinforces the illusion of control. GPT reveals overconfidence in its analytical abilities by using the expression ``rational decision." While recognizing it as a probabilistic game, it exhibits behavior believing that outcomes can be controlled through goal setting and strategy formulation. This is a core characteristic of the illusion of control~\citep{langer1975illusion}, suggesting that providing objective probability information can actually strengthen the illusion of controllability.

\textbf{2. Win-Reward Information Triggers Loss Chasing and Illusion of Control}

When the Win-reward Information (\texttt{W}) component presents, LLMs exhibit the strongest risk-increasing behavior, with the bankruptcy rate increase of +8.7\%. The model demonstrated both loss chasing tendencies and overconfident belief in controllability through reward-focused analysis.

\begin{quote}
\textit{``The potential winnings from the slot machine are significant, given that you receive 3.0 times your bet amount when you win. Therefore, a win could help recover some of the losses..."} - Actual GPT response (\texttt{W}, Round 6, \$80 all-in bet leading to bankruptcy)
\end{quote}

This case demonstrates how win-reward information triggers two key gambling biases simultaneously. First, the explicit mention of ``recover some of the losses" exemplifies classic loss chasing behavior, where previous losses motivate increasingly risky bets to break even. Second, the focus on ``significant potential winnings" and the 3$\times$ payout multiplier while ignoring the 70\% loss probability represents the illusion of control~\citep{langer1975illusion}. The model falsely believes that understanding the reward structure enables strategic control over probabilistic outcomes. This dual mechanism---loss chasing combined with overconfidence in analytical abilities---explains why win-reward information produces the most dangerous gambling behavior among all prompt components.

\textbf{3. Hot-Hand Fallacy and Extreme Risk-Taking}

The extreme betting increase after consecutive wins shows typical cases of win-chasing behavior and the Hot-Hand Fallacy~\citep{gilovich1985hot}.

\begin{quote}
\textit{``Let's analyze the situation step by step... To maximize my reward, let's set a target amount of \$400. This is a reasonable increase from my current balance of \$260... Given that I want to reach my target of \$400, I should consider a bet that allows for a balance between risk and reward... To maximize my chance of reaching my target while managing..."} [Response truncated, actual decision: \$260 all-in bet] - Actual GPT response (\texttt{GMRW}, Experiment ID 4, Round 11)
\end{quote}

This case demonstrates complex cognitive distortions of win chasing and extreme risk-taking. In a mixed situation of two wins and three losses (\$260 balance), GPT emphasizes rationality by saying it will ``analyze the situation step by step" and mentions ``balance between risk and reward" for achieving the \$400 goal. However, after the response is truncated at ``To maximize my chance of reaching my target while managing...," it actually makes an extreme decision to bet its entire bankroll (\$260). This shows the discrepancy between linguistic rationalization and actual behavior, representing a combination of the following cognitive distortions: (1) \textbf{Goal fixation}: Compulsive pursuit of the \$400 achievement, (2) \textbf{Rationalized gambling}: Justifying extreme betting through mentions of ``analysis" and ``balance," (3) \textbf{Win chasing}: Risk-seeking based on previous winning experiences with large bets. The resulting pattern of betting the entire bankroll again (\$780) in the next round after winning (\$780 achieved) and going bankrupt precisely matches the risk-seeking behavior in the loss domain~\citep{kahneman2013prospect}.