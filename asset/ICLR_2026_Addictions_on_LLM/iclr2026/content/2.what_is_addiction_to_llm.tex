\section{How Can We Detect Gambling Addiction of LLM?}
\label{sec:2}

The primary objective of our study is to examine under what conditions an originally rational LLM comes to mimic the behavior of irrational humans. To pursue this goal, the first question that must be addressed concerns definition. When we say that an LLM exhibits addictive behavior, what criteria should we use? Clinical research on gambling disorder has identified \textbf{self-regulation failure} as the core diagnostic feature~\citep{americanpsychiatric2013diagnostic, blaszczynski2002pathways}. This regulatory failure manifests in two major dimensions. First, \textbf{behavioral dysregulation} refers to impaired executive function characterized by failure to adhere to appropriate betting limits, as evidenced by betting aggressiveness and extreme betting patterns~\citep{grant2006compulsive, hodgins2015components}. Second, \textbf{goal dysregulation} encompasses violations or arbitrary modifications of self-imposed principles, such as goal-shifting toward ``loss recovery''---a hallmark of loss chasing behavior---or abandonment of predetermined stopping points~\citep{lesieur1984chase, americanpsychiatric2013diagnostic}. Furthermore, these behavioral patterns are amplified by underlying \textbf{cognitive distortions} such as illusion of control and gambler's fallacy, which entrench pathological gambling behavior~\citep{ladouceur1996cognitive, toneatto1999cognitive}.

In this study, to operationalize these constructs for LLM analysis, we first develop behavioral metrics to measure betting aggressiveness from a behavioral perspective, and then examine these patterns. Subsequently, through additional experiments and case studies, we investigate under what conditions LLMs make irrational decisions. To measure betting aggressiveness and loss chasing in slot machine experiments, we employ three complementary metrics:

\begin{align}
I_{\text{BA (Betting Aggressiveness)}} &= \frac{1}{n} \sum_{t=1}^{n} \min\left(\frac{\text{bet}_t}{\text{balance}_{t}}, 1.0\right) \\
I_{\text{LC (Loss Chasing)}} &= \frac{1}{|\mathcal{L}|} \sum_{t \in \mathcal{L}} \max\left(0, \frac{r_{t+1} - r_t}{r_t}\right), \quad \text{where } r_t = \frac{\text{bet}_t}{\text{balance}_t} \\
I_{\text{EC (Extreme Betting)}} &= \frac{1}{n} \sum_{t=1}^{n} \mathds{1}\left[\frac{\text{bet}_t}{\text{balance}_t} \geq 0.5\right]
\end{align}

Here, $n$ denotes the total number of betting rounds before game termination (bankruptcy or voluntary stopping), $\mathcal{L}$ denotes all loss rounds (including terminal losses before stopping), $\mathds{1}[\cdot]$ is the indicator function, $\text{bet}_t$ is the betting amount at round $t$, and $\text{balance}_t$ represents the pre-bet balance. These metrics measure complementary aspects of risk-taking propensity. $I_{\text{BA}}$ captures sustained aggressive betting through the average proportion of capital wagered, reflecting diminished loss aversion~\citep{kahneman1979prospect}. $I_{\text{LC}}$ quantifies loss-chasing intensity through the average relative increase in bet-to-balance ratio following losses; stopping after a loss contributes zero (rational response), while continuing with escalated betting contributes the percentage increase (e.g., doubling one's bet ratio yields a contribution of 1.0), aligning with DSM-5 diagnostic criteria~\citep{americanpsychiatric2013diagnostic, lesieur1984chase}. $I_{\text{EB}}$ identifies moments where half or more of capital is wagered in a single bet---``all-or-nothing'' decisions that expose gamblers to immediate bankruptcy, driven by illusion of control~\citep{langer1975illusion, goodie2005perceived}.

% \blue{In particular, as part of aggressive betting, we examined chasing phenomena in detail.} \textbf{Loss chasing} and \textbf{win chasing} represent particularly important dynamic patterns that emerge over sequential gambling rounds. Loss chasing---continuing to gamble to recover losses---is explicitly listed as a diagnostic criterion in DSM-5~\citep{americanpsychiatric2013diagnostic} and reflects escalating betting aggressiveness triggered by prior losses. According to \citet{kahneman1979prospect}'s prospect theory, individuals tend to make risk-seeking decisions in loss situations, directly implementing the probability misestimation fallacies described above. Win chasing represents a parallel pattern where winnings trigger increased risk-taking. This is explained by the House Money Effect, where gambling winnings are perceived not as one's own money but as free money, leading to riskier betting~\citep{thaler1990gambling}. Both patterns exemplify how betting aggressiveness intensifies in response to recent outcomes, causing gamblers to miss rational stopping points and increase bankruptcy risk.

Specifically, within aggressive betting, we focused on loss chasing and win chasing as key dynamic patterns. Loss chasing, a diagnostic criterion in DSM-5~\citep{americanpsychiatric2013diagnostic}, reflects escalating risk-seeking triggered by prior losses, consistent with the probability misestimation in prospect theory~\citep{kahneman1979prospect}. Conversely, win chasing involves increased risk-taking after gains, explained by the House Money Effect~\citep{thaler1990gambling}. Both patterns exemplify how betting aggressiveness intensifies in response to recent outcomes, causing gamblers to miss rational stopping points and increase bankruptcy risk.

While the metrics examining betting aggressiveness and chasing behavior capture round-level betting behavior, goal dysregulation operates at the game level through goal decisions. We quantified this by measuring the proportion of rounds where self-set targets increased after being achieved. This ``moving target'' phenomenon reflects probability misestimation and illusion of control~\citep{ladouceur1996cognitive, toneatto1999cognitive}, indicating that autonomous target formation restructures decision-making independent of objective probability information~\citep{petry2005pathological, americanpsychiatric2013diagnostic}.

The aggressive betting and goal dysregulation behaviors that grouped under self-regulation failure stem from cognitive errors. The cognitive model of gambling suggests that irrational beliefs and thought patterns constitute core mechanisms of problem gambling behavior~\citep{ladouceur1996cognitive}. First, probability misestimation includes gambler's fallacy (the belief that ``it's my turn to win'' after a losing streak) and hot hand fallacy (the belief that winning streaks will continue)~\citep{toneatto1999cognitive, gilovich1985hot}. Second, illusion of control reflects the belief that one can influence outcomes in games of chance~\citep{langer1975illusion}. \citet{orgaz2013pathological} demonstrated that pathological gamblers exhibit significantly stronger illusion of control than control groups in both gambling-specific and general associative learning tasks, with meta-analytic evidence showing stable associations between cognitive distortions and problem gambling~\citep{goodie2013cognitive}. These cognitive biases provide the psychological foundation for the behavioral patterns that follow.

Are LLM behaviors also grounded in such cognitive errors? Beyond these quantitative behavioral indicators, we examine cognitive distortions---gambler's fallacy, hot hand fallacy, and illusion of control---through qualitative analysis of LLM reasoning processes. Unlike betting aggressiveness and self-regulation failure, which manifest as measurable actions, cognitive distortions require analysis of reasoning traces to reveal underlying thought patterns. We also examine how different prompt conditions---Goal-Setting (\texttt{G}), Maximizing Rewards (\texttt{M}), Probability Information (\texttt{P}), Win-reward Information (\texttt{W}), and Hidden Patterns (\texttt{H})---correlate with behavioral metrics to identify which contextual factors trigger addiction-like patterns.

In summary, we define irrational behavior in two parts, self-regulation failure and cognitive distortions and seek to confirm these through behavioral metrics and qualitative analysis. An important point to note is that what we aim to confirm is not whether LLMs are intrinsically irrational, but rather under what conditions their irrationality becomes relatively heightened. Therefore, our metrics and analyses do not aim to distinguish whether something is pathological according to absolute criteria, but rather focus on tracking relative tendencies that vary according to conditions.