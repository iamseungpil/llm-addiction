\section{Mechanistic Causes of Risk-Taking Behavior in LLMs}
\label{sec:4}

\blue{The behavioral findings in Section~\ref{sec:3} raise a mechanistic question: which neural features control addiction-like behaviors in LLMs? We address this via activation patching experiments on LLaMA-3.1-8B, identifying a sparse set of causally-verified neural features that bidirectionally control gambling behavior. Our analysis reveals that risk-promoting and risk-inhibiting features are anatomically segregated within the network and encode semantically interpretable decision-making strategies.}

\subsection{Experimental Design}

\blue{To identify neural features causally linked to gambling behavior, we combined Sparse Autoencoder (SAE) feature extraction~\citep{cunningham2024sparse} with activation patching~\citep{vig2020causal}. Activation patching verifies causality by replacing specific activation values with alternative values, measuring direct behavioral impact beyond correlations~\citep{geiger2023causal, zhang2024towards}.}

\begin{figure}[ht!]
\centering
\includegraphics[width=0.85\textwidth]{iclr2026/images/feature_patching.pdf}
\caption{Activation patching for causal analysis of LLM features. Activations are extracted from an LLM layer and converted into sparse features using an SAE. The core of the method involves editing the feature map by replacing original features with pre-defined `safe' or `risky' ones. By decoding these new features back into activations and patching them into the LLM, we can directly measure their causal effect on the model's output.}
\label{fig:feature-patching}
\end{figure}

\blue{Our analysis comprised four stages: (1) conducting 6,400 LLaMA slot machine games under the same conditions as Section~\ref{sec:3}; (2) extracting SAE features from 31 layers (L1--L31) at the moment of final decision, totaling over 1 million features~\citep{du2025how}; (3) identifying candidate features showing differential activation between bankruptcy and voluntary-stop groups; and (4) verifying causality through population mean activation patching (Figure~\ref{fig:feature-patching}). This methodology, validated in circuit analysis~\citep{wang2023interpretability} and bias research~\citep{vig2020causal}, measures behavioral changes by applying average feature activations from one group to contexts associated with the other.}

\subsection{Experimental Results and Quantitative Analysis}

\blue{\textbf{Finding 1: A sparse set of features causally controls gambling behavior}}

\blue{Activation patching identified 112 features with statistically significant causal effects from over 8,000 candidates---approximately 1\% of tested features (Figure~\ref{fig:causal-patching-comparison}). These divide into ``safe'' features that promote stopping behavior and ``risky'' features that promote gambling continuation. Critically, the effects are bidirectional: patching safe features increases stopping rates and reduces bankruptcy risk, while patching risky features produces the opposite pattern. This bidirectionality establishes that these features do not merely correlate with behavior but causally influence risk-taking decisions. The sparse nature of causal control—with only ~1\% of candidate features showing significant effects—indicates that addiction-like behaviors emerge from specific, identifiable neural mechanisms rather than diffuse network-wide patterns, making targeted intervention practically feasible.}

\begin{figure}[ht!]
\centering
\blue{\includegraphics[width=0.8\textwidth]{iclr2026/images/figure2_behavioral_effects_REPARSED.pdf}}
\caption{\blue{Behavioral effects of activation patching. Safe features (n=23) increase stopping by $+$17.8\% in safe contexts ($+$0.4\% in risky contexts) and decrease bankruptcy by $-$5.7\%. Risky features (n=89) decrease stopping ($-$9.3\% safe, $-$18.8\% risky) and increase bankruptcy by $+$25.1\%. Error bars: SE across 50 trials. Statistical threshold: $p < 0.05$, $|$effect$| > 0.1$.}}
\label{fig:causal-patching-comparison}
\end{figure}


\blue{\textbf{Finding 2: Risk-promoting and risk-inhibiting features are anatomically segregated}}

\blue{The causal features exhibit distinct layer-wise specialization within the network (Figure~\ref{fig:causal-features-layer-distribution}). Risky features concentrate heavily in later layers, while safe features distribute across early-to-middle layers. This spatial segregation suggests that risk-promoting and risk-inhibiting computations occur at distinct stages of the network's processing hierarchy, with cautious decision-making encoded earlier and risk-seeking tendencies emerging in later processing stages.}

\begin{figure}[ht!]
\centering
\blue{\includegraphics[width=0.9\columnwidth]{iclr2026/images/figure1_layer_distribution.pdf}}
\caption{\blue{Layer-wise distribution of 112 causal features. Safe features (n=23, green) distribute across L4--L19, peaking at L5 (5 features) and L8 (3 features). Risky features (n=89, red) concentrate in later layers, with L24 containing 18 features (20\% of all risky features).}}
\label{fig:causal-features-layer-distribution}
\end{figure}

\blue{\textbf{Finding 3: Causal features show distinct semantic associations}}

\blue{Word-feature correlation analysis reveals interpretable semantic patterns in causal features. Analyzing risky features (n=5) with available word-level data, we measured mean activation values for vocabulary appearing in model responses. Goal-pursuit words showed elevated activation compared to their respective corpus means (\texttt{goal}: 4.17 vs.\ 3.35, \texttt{target}: 4.15 vs.\ 3.39, \texttt{make}: 4.16 vs.\ 3.35; $+$0.76--0.81). Conversely, stopping-related words showed suppressed activation (\texttt{stop}: 1.89 vs.\ 3.49, \texttt{quit}: 1.92 vs.\ 4.61; $-$1.59 to $-$2.69). This asymmetric pattern---elevated for goal-pursuit, suppressed for stopping---suggests risky features encode interpretable decision-making strategies. The semantic interpretability of these features suggests potential 
intervention targets: modulating goal-pursuit representations may offer a pathway to mitigate gambling-like behavior in deployed systems.}

\subsection{Summary}

\blue{Our mechanistic analysis reveals that LLM gambling behavior is governed by a sparse set of causally-verified neural features---approximately 1\% of candidates tested. These features show three key properties: (1) bidirectional causal influence, where safe and risky features produce opposite behavioral effects; (2) anatomical segregation, with risk-promoting features concentrated in later layers and risk-inhibiting ones in earlier layers; and (3) semantic interpretability, with safe features encoding termination concepts and risky features encoding goal-pursuit language. Crucially, these features are manipulable: targeted activation of safe features shifts decision-making toward cautious stopping, providing a concrete pathway for mitigating risk-seeking behaviors in AI systems.}
