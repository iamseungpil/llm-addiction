\section{Conclusion}

This study empirically demonstrates that LLMs exhibit behavioral patterns and neural mechanisms resembling human gambling addiction. Through systematic experiments, we confirmed that models consistently reproduce cognitive distortions—such as illusion of control and asymmetric chasing—and that these patterns are driven by causally identifiable neural features.

Our research makes three key contributions: (1) a behavioral framework grounded in clinical psychology for evaluating addiction-like behaviors via betting metrics; (2) the identification of triggering conditions, particularly variable betting and goal-setting, where greater autonomy amplifies irrationality; and (3) the discovery of causal neural features controllable via activation patching.

However, limitations remain regarding our reliance on a single gambling paradigm, the discrepancy between models used for behavioral versus neural analyses, and the open question of normative rationality standards. Generalization to other risk-related tasks and cross-model neural comparisons require further validation.

These findings suggest that AI systems have internalized human-like risk-seeking mechanisms, making the understanding and control of these patterns critical as LLMs enter high-stakes domains. We emphasize the necessity of continuous monitoring, particularly during reward optimization processes where such behaviors may emerge unexpectedly.