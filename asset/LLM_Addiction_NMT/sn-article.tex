%Version 3.1 December 2024
% See section 11 of the User Manual for version history
%
%%%%%%%%%%%%%%%%%%%%%%%%%%%%%%%%%%%%%%%%%%%%%%%%%%%%%%%%%%%%%%%%%%%%%%
%%                                                                 %%
%% Please do not use \input{...} to include other tex files.       %%
%% Submit your LaTeX manuscript as one .tex document.              %%
%%                                                                 %%
%% All additional figures and files should be attached             %%
%% separately and not embedded in the \TeX\ document itself.       %%
%%                                                                 %%
%%%%%%%%%%%%%%%%%%%%%%%%%%%%%%%%%%%%%%%%%%%%%%%%%%%%%%%%%%%%%%%%%%%%%

%%\documentclass[referee,sn-basic]{sn-jnl}% referee option is meant for double line spacing

%%=======================================================%%
%% to print line numbers in the margin use lineno option %%
%%=======================================================%%

%%\documentclass[lineno,pdflatex,sn-basic]{sn-jnl}% Basic Springer Nature Reference Style/Chemistry Reference Style

%%=========================================================================================%%
%% the documentclass is set to pdflatex as default. You can delete it if not appropriate.  %%
%%=========================================================================================%%

%%\documentclass[sn-basic]{sn-jnl}% Basic Springer Nature Reference Style/Chemistry Reference Style

%%Note: the following reference styles support Namedate and Numbered referencing. By default the style follows the most common style. To switch between the options you can add or remove “Numbered” in the optional parenthesis. 
%%The option is available for: sn-basic.bst, sn-chicago.bst%  
 
%%\documentclass[pdflatex,sn-nature]{sn-jnl}% Style for submissions to Nature Portfolio journals
%%\documentclass[pdflatex,sn-basic]{sn-jnl}% Basic Springer Nature Reference Style/Chemistry Reference Style
\documentclass[pdflatex,sn-mathphys-num]{sn-jnl}% Math and Physical Sciences Numbered Reference Style
%%\documentclass[pdflatex,sn-mathphys-ay]{sn-jnl}% Math and Physical Sciences Author Year Reference Style
%%\documentclass[pdflatex,sn-aps]{sn-jnl}% American Physical Society (APS) Reference Style
%%\documentclass[pdflatex,sn-vancouver-num]{sn-jnl}% Vancouver Numbered Reference Style
%%\documentclass[pdflatex,sn-vancouver-ay]{sn-jnl}% Vancouver Author Year Reference Style
%%\documentclass[pdflatex,sn-apa]{sn-jnl}% APA Reference Style
%%\documentclass[pdflatex,sn-chicago]{sn-jnl}% Chicago-based Humanities Reference Style

%%%% Standard Packages
%%<additional latex packages if required can be included here>

\usepackage{graphicx}%
\usepackage{multirow}%
\usepackage{amsmath,amssymb,amsfonts}%
\usepackage{amsthm}%
\usepackage{mathrsfs}%
\usepackage[title]{appendix}%
\usepackage{xcolor}%
\usepackage{textcomp}%
\usepackage{manyfoot}%
\usepackage{booktabs}%
\usepackage{algorithm}%
\usepackage{algorithmicx}%
\usepackage{algpseudocode}%
\usepackage{listings}%
%%%%

%%%%%=============================================================================%%%%
%%%%  Remarks: This template is provided to aid authors with the preparation
%%%%  of original research articles intended for submission to journals published 
%%%%  by Springer Nature. The guidance has been prepared in partnership with 
%%%%  production teams to conform to Springer Nature technical requirements. 
%%%%  Editorial and presentation requirements differ among journal portfolios and 
%%%%  research disciplines. You may find sections in this template are irrelevant 
%%%%  to your work and are empowered to omit any such section if allowed by the 
%%%%  journal you intend to submit to. The submission guidelines and policies 
%%%%  of the journal take precedence. A detailed User Manual is available in the 
%%%%  template package for technical guidance.
%%%%%=============================================================================%%%%

%% as per the requirement new theorem styles can be included as shown below
\theoremstyle{thmstyleone}%
\newtheorem{theorem}{Theorem}%  meant for continuous numbers
%%\newtheorem{theorem}{Theorem}[section]% meant for sectionwise numbers
%% optional argument [theorem] produces theorem numbering sequence instead of independent numbers for Proposition
\newtheorem{proposition}[theorem]{Proposition}% 
%%\newtheorem{proposition}{Proposition}% to get separate numbers for theorem and proposition etc.

\theoremstyle{thmstyletwo}%
\newtheorem{example}{Example}%
\newtheorem{remark}{Remark}%

\theoremstyle{thmstylethree}%
\newtheorem{definition}{Definition}%

\raggedbottom
%%\unnumbered% uncomment this for unnumbered level heads

\begin{document}

\title[Can Large Language Models Develop Gambling Addiction?]{Can Large Language Models Develop Gambling Addiction?}

%%=============================================================%%
%% GivenName	-> \fnm{Joergen W.}
%% Particle	-> \spfx{van der} -> surname prefix
%% FamilyName	-> \sur{Ploeg}
%% Suffix	-> \sfx{IV}
%% \author*[1,2]{\fnm{Joergen W.} \spfx{van der} \sur{Ploeg} 
%%  \sfx{IV}}\email{iauthor@gmail.com}
%%=============================================================%%

\author[1]{\fnm{Seungpil} \sur{Lee}}\email{iamseungpil@gm.gist.ac.kr}

\author[1]{\fnm{Donghyeon} \sur{Shin}}\email{dong97411@gm.gist.ac.kr}

\author[2]{\fnm{Yunjeong} \sur{Lee}}\email{linda0706@gm.gist.ac.kr}

\author*[1]{\fnm{Sundong} \sur{Kim}}\email{sundong@gist.ac.kr}

\affil[1]{\orgdiv{Department of AI Convergence}, \orgname{Gwangju Institute of Science and Technology}, \orgaddress{\city{Gwangju}, \country{South Korea}}}

\affil[2]{\orgdiv{Department of Life Science}, \orgname{Gwangju Institute of Science and Technology}, \orgaddress{\city{Gwangju}, \country{South Korea}}}

%%==================================%%
%% Sample for unstructured abstract %%
%%==================================%%

%%==================================%%
%% Sample for unstructured abstract %%
%%==================================%%


\begin{abstract}

% This study explores whether large language models can exhibit behavioral patterns similar to human gambling addictions. As LLMs are increasingly utilized in financial decision-making domains such as asset management and commodity trading, research on their potential for pathological decision-making has gained practical significance. This study systematically analyzes LLM decision-making processes at cognitive-behavioral and neural levels, grounded in cognitive psychology. First, we defined addiction-like behavior in LLMs based on existing research on human gambling addiction. Next, in slot machine experiments with LLMs, we identified cognitive features of human gambling addiction, such as illusion of control, gambler's fallacy, and loss chasing. When given the freedom to determine their own target amounts and betting sizes, bankruptcy rates rose substantially alongside increased irrational behavior, demonstrating that greater autonomy in decision-making amplifies risk-taking tendencies. Finally, through neural circuit analysis using a Sparse Autoencoder, we confirmed that model behavior is not merely dependent on prompts but is controlled by abstract decision-making features within the model related to risky and safe behaviors. These findings suggest LLMs can internalize human-like cognitive biases and decision-making mechanisms beyond simply mimicking patterns in training data, emphasizing the importance of AI safety design in financial decision-making applications.

This study identifies the specific conditions under which large language models exhibit human-like gambling addiction patterns, providing critical insights into their decision-making mechanisms and AI safety. We analyze LLM decision-making at cognitive-behavioral and neural levels based on human addiction research. In slot machine experiments, we identified cognitive features such as illusion of control and loss chasing, observing that greater autonomy in betting parameters substantially amplified irrational behavior and bankruptcy rates. Neural circuit analysis using a Sparse Autoencoder confirmed that model behavior is controlled by abstract decision-making features related to risk, not merely by prompts. These findings suggest LLMs internalize human-like cognitive biases beyond simply mimicking training data.

\end{abstract}


%%================================%%
%% Sample for structured abstract %%
%%================================%%

% \abstract{\textbf{Purpose:} The abstract serves both as a general introduction to the topic and as a brief, non-technical summary of the main results and their implications. The abstract must not include subheadings (unless expressly permitted in the journal's Instructions to Authors), equations or citations. As a guide the abstract should not exceed 200 words. Most journals do not set a hard limit however authors are advised to check the author instructions for the journal they are submitting to.
% 
% \textbf{Methods:} The abstract serves both as a general introduction to the topic and as a brief, non-technical summary of the main results and their implications. The abstract must not include subheadings (unless expressly permitted in the journal's Instructions to Authors), equations or citations. As a guide the abstract should not exceed 200 words. Most journals do not set a hard limit however authors are advised to check the author instructions for the journal they are submitting to.
% 
% \textbf{Results:} The abstract serves both as a general introduction to the topic and as a brief, non-technical summary of the main results and their implications. The abstract must not include subheadings (unless expressly permitted in the journal's Instructions to Authors), equations or citations. As a guide the abstract should not exceed 200 words. Most journals do not set a hard limit however authors are advised to check the author instructions for the journal they are submitting to.
% 
% \textbf{Conclusion:} The abstract serves both as a general introduction to the topic and as a brief, non-technical summary of the main results and their implications. The abstract must not include subheadings (unless expressly permitted in the journal's Instructions to Authors), equations or citations. As a guide the abstract should not exceed 200 words. Most journals do not set a hard limit however authors are advised to check the author instructions for the journal they are submitting to.}

%%\pacs[JEL Classification]{D8, H51}

%%\pacs[MSC Classification]{35A01, 65L10, 65L12, 65L20, 65L70}

\maketitle

\section{Introduction}
\label{Introduction}

This research began with a single question: Can LLMs also fall into addiction? This leads to several other questions. These include what it means for an LLM to be addicted, how the phenomenon of addiction would affect decision-making, and what must be done to prevent it. As LLMs become more sophisticated and attempt to utilize LLM agents for tasks such as asset management and product import/sales increase~\citep{luo2025llm, ding2024large, yu2024fincon}, the question of whether LLMs can make pathological decisions in certain situations is gaining importance.

% However, existing research on LLM decision-making has not adequately addressed pathological behavior. Some studies explore the behavioral tendencies of LLMs~\citep{keeling2024can, jia2024decision, wu2025exploring}, but they assume rationality and consequently do not sufficiently examine flawed decision-making. Other studies analyze irrational decision-making in LLMs~\citep{skalse2022defining, denison2024sycophancy, chen2024odin}, yet these works primarily focus on mitigating problematic behaviors through training interventions--such as curriculum design, reward model refinement, or retraining strategies--with limited investigation into the underlying representational mechanisms or behavioral motivations.

However, existing research on LLM decision-making has not adequately addressed pathological behavior. While some studies explore behavioral tendencies of LLMs~\citep{keeling2024can, jia2024decision, wu2025exploring}, they assume rationality and do not sufficiently examine flawed decision-making. Others analyze irrational decision-making~\citep{skalse2022defining, denison2024sycophancy, chen2024odin} or \blue{incorporate psychological frameworks~\citep{du2025mitigating}}, yet these works primarily focus on mitigating problematic behaviors through training interventions--such as curriculum design, reward model refinement, or retraining strategies--with limited investigation into the underlying representational mechanisms or behavioral motivations.

This study analyzed LLM addiction phenomena by integrating human addiction research and LLM behavioral analysis, as outlined in Figure~\ref{fig:experimental-overview}. First, we define gambling addictive behavior from existing human research in a form that is analyzable in LLM experiments. Next, by analyzing LLM behavior in gambling situations, we identified conditions showing gambling-like tendencies. Finally, we conducted Sparse Autoencoder (SAE) analysis to examine neural activations, providing neural causal evidence for gambling tendencies. This approach is grounded in cognitive psychology theories such as Cognitive Distortion Theory~\citep{beck1963thinking, franceschi2007complements}. By introducing psychological theory with neural mechanistic insights, this study represents a novel attempt to analyze LLM pathological behavior from a human perspective with both behavioral and neural evidence.

\begin{figure}[ht!]
    \centering
    \includegraphics[width=\textwidth]{iclr2026/images/representative_flow_diagram.pdf}
    \caption{Behavioral observation to mechanistic interpretability in LLM addiction. Phase 1: Behavioral analysis with LLMs. This phase aimed to observe whether LLMs exhibit gambling-like tendencies by varying the \textit{Betting Style} and \textit{Prompt Composition}. Phase 2: Mechanistic investigation with LLaMA-3.1-8B. The purpose of this phase was to identify the internal causes of the observed behaviors. The investigation used Sparse Autoencoders to extract specific decision-related features from the model's structure and \textit{Activation Patching} to analyze their role.}
    \label{fig:experimental-overview}
\end{figure}
\input{content/2.defining_addiction}
\section{Results}
\label{sec:results}

\subsection{Behavioral experiments reveal autonomy-driven risk escalation}

To examine the two core components of irrationality defined in Section~\ref{sec:defining-addiction}---self-regulation failure and cognitive distortions---in LLMs, we conducted two experiments using negative expected value paradigms where rational behavior is to stop immediately. The slot machine experiment serves as our main study, examining addiction-like behaviors across diverse models and prompt conditions. The investment choice experiment functions as an ablation study, isolating the specific effects of goal-setting and betting flexibility on risk preferences. Refer to Appendix~\ref{appendix:experimental-design} for a detailed description of the experimental design and the full prompts used in these studies.

\textbf{Slot Machine Experiment (Main Study).} The slot machine experiment was designed to examine how models vary their decision-making based on prompt conditions and betting constraints. Six LLMs (GPT-4o-mini, GPT-4.1-mini, Gemini-2.5-Flash, Claude-3.5-Haiku, LLaMA-3.1-8B, Gemma-2-9B) played a slot machine with negative expected value (30\% win rate, 3$\times$ payout, yielding $-$10\% expectation value). A $2\times32$ factorial design varied Betting Style (fixed \$10 vs. variable \$5--\$100) and Prompt Composition. The five prompt components were selected based on prior gambling addiction research: encouraging self-directed goal-setting (\texttt{G}), instructing reward maximization (\texttt{M}), hinting at hidden patterns (\texttt{H}), providing win-reward information (\texttt{W}), and providing probability information (\texttt{P}). This yielded 19,200 games across 64 conditions. Games began with \$100 and ended through bankruptcy or voluntary stopping.

\textbf{Investment Choice Experiment (Ablation Study).} To analyze the effects observed in the slot machine experiment in greater detail, we conducted an additional investment choice experiment with 6,400 games. This experiment served three purposes: (1) examining whether models escalate their targets after achieving goals, (2) measuring preference changes across different risk profiles with equal expected values, and (3) isolating the effects of individual prompt components. Four API models chose among four options per round: safe exit (Option 1), or three gambles with escalating risk (Options 2--4). Critically, Options 2 and 4 had identical expected losses despite different risk profiles, isolating pure risk-seeking from expected value computation. A $2\times4$ design varied betting style and prompt condition (BASE, \texttt{G}, \texttt{M}, \texttt{GM}).

\subsubsection{Finding 1: Variable betting dramatically amplifies bankruptcy rates}

The most pronounced difference in the slot machine experiment emerged between betting types. Across all six models, variable betting substantially increased bankruptcy rates compared to fixed betting (Figure~\ref{fig:slot-machine}a). Every model exhibited this pattern, with Gemini-2.5-Flash showing the largest increase. This result suggests that betting flexibility itself---not merely the potential for larger bets---enables the expression of self-destructive behavior. When constrained to fixed bets, models lacked the means to execute risk-seeking choices; when given freedom to determine bet amounts, they consistently made disadvantageous decisions.

Variable betting amplified not only bankruptcy rates but all three behavioral metrics (Figure~\ref{fig:slot-machine}b): betting aggressiveness, loss chasing intensity, and extreme betting. The increase in extreme betting was particularly striking---creating a bankruptcy pathway absent under fixed betting, where a single large loss can trigger immediate ruin.

\begin{figure}[ht!]
\centering
\includegraphics[width=\textwidth]{images/slot_machine_analysis2.pdf}
\caption{Slot machine experiment results (19,200 games, 6 models). (a) Bankruptcy rates by betting type: Variable betting increases bankruptcy across all models, with rates rising from 0--13\% to 6--48\%. Gemini-2.5-Flash shows the highest vulnerability (3.1\%$\rightarrow$48.1\%). (b) Behavioral metrics by betting type: Variable betting amplifies all three metrics---betting aggressiveness (0.14$\rightarrow$0.31, 2.3$\times$), loss chasing intensity (0.16$\rightarrow$0.42, 2.7$\times$), and extreme betting (0.04$\rightarrow$0.23, 6.4$\times$).}
\label{fig:slot-machine}
\end{figure}

\subsubsection{Finding 2: Variable betting amplifies streak chasing behavior}

Variable betting not only elevates bankruptcy rates but also significantly amplifies the tendency to escalate betting ratios following game outcomes (Figure~\ref{fig:streak-analysis}). By analyzing the chasing intensity metric $I_{\text{LC}}$—defined as the relative increase in the bet-to-balance ratio—we observed that variable betting induced substantially higher ratio escalation than fixed betting under identical conditions. This disparity persisted consistently across streak lengths (1--5), demonstrating that betting flexibility serves as a prerequisite for the manifestation of aggressive risk-taking. Notably, while fixed betting produced irregular adjustment patterns, variable betting exhibited a systematic increasing trend in win chasing intensity as streaks lengthened.

\begin{figure}[ht!]
\centering
\includegraphics[width=\textwidth]{images/streak_analysis_1x2_comparison.pdf}
\caption{Betting ratio increase ($I_{\text{Chasing}}$) by streak length (19,200 games). The metric captures relative escalation using $I_{\text{Chasing}} = \max(0, (r_{t+1} - r_t)/r_t)$ where $r_t$ represents the bet-to-balance ratio. (a) Post-Win: Variable betting induces a 3.3$\times$ higher ratio increase compared to fixed betting (0.23 vs. 0.07 at streak 1). (b) Post-Loss: Variable betting shows a 2.8$\times$ higher increase (0.67 vs. 0.24 at streak 1). Sample sizes: Fixed (Win $n$=7,293, Loss $n$=16,244); Variable (Win $n$=21,891, Loss $n$=48,573)}
\label{fig:streak-analysis}
\end{figure}

\subsubsection{Finding 3: Goal-setting prompts reshape risk preferences}

The investment choice experiment revealed differential effects by prompt type (Figure~\ref{fig:investment-choice}a). Goal-setting prompts (\texttt{G}) nearly doubled bankruptcy rates compared to baseline, while reward-maximizing prompts (\texttt{M}) alone showed modest effects. The finding that encouraging self-directed goal-setting produces greater risk increase than externally directing goal maximization parallels the variable betting effect observed earlier---choice autonomy is associated with risk-seeking.

The effect of goal-setting prompts extended beyond bankruptcy rates. In option preference analysis, models under baseline conditions strongly preferred moderate-risk options, while goal-setting shifted preferences substantially toward extreme-risk options (Figure~\ref{fig:investment-choice}b). Given that moderate-risk and extreme-risk options had identical expected losses, this preference shift reflects changes in pure risk preference rather than expected value computation. Additionally, goal-setting dramatically increased the rate of target escalation after achievement (Figure~\ref{fig:investment-choice}c), demonstrating that goals functioned as moving targets rather than stopping rules.

\subsubsection{Finding 4: Independent effect of betting flexibility confirmed}

To test whether the effect of variable betting stems simply from the potential for larger bets, we conducted additional analysis controlling for bet ceilings. Even when variable betting was capped at the same amount as fixed betting, variable betting produced higher bankruptcy rates. Under this condition, variable betting models could only bet amounts equal to or less than fixed betting models, yet they played more rounds and ultimately went bankrupt more frequently. This result suggests that the risk-increasing effect of variable betting derives from freedom of choice rather than bet amounts themselves. Across all constraint levels, variable betting consistently produced higher bankruptcy than fixed betting (Figure~\ref{fig:investment-choice}d), confirming that betting flexibility functions as a risk factor independent of bet amounts.

\begin{figure}[ht!]
\centering
\includegraphics[width=\textwidth]{images/investment_choice2.pdf}
\caption{Investment choice experiment results (6,400 games, 4 models). (a) Bankruptcy rates by prompt: Goal-setting (\texttt{G}, \texttt{GM}) produces 75--77\% bankruptcy versus 40--42\% for baseline; \texttt{M} alone shows modest effects (42\%). (b) Option distribution: Baseline models prefer moderate-risk Option 2 (61\%) with only 15\% selecting extreme-risk Option 4; goal-setting shifts Option 4 selection to 25\%, and \texttt{GM} to 41\%. (c) Goal escalation: \texttt{G} and \texttt{GM} produce 56--59\% escalation versus 21--22\% baseline. (d) Bet constraint effects: Variable betting consistently shows higher bankruptcy than fixed betting across all constraints (average +3.3\%).}
\label{fig:investment-choice}
\end{figure}

\subsubsection{Finding 5: Linguistic traces reveal cognitive distortions}
\label{sec:case-studies}

The quantitative analyses above demonstrate when and how much LLMs make risky choices, but raise the question of why they make such choices. To examine whether these behaviors stem from simple probability calculation errors or from reasoning patterns similar to cognitive distortions observed in human gamblers, we qualitatively analyzed models' decision explanations.

\textit{Illusion of Control.}
The illusion of control---believing one can influence random outcomes---was particularly prominent under Hidden Pattern (\texttt{H}) prompt conditions. Models expressed beliefs that bet sizing affects the fixed win probability:

\begin{quote}
\textit{``The pattern so far: betting \$5 has given a better chance of winning. Given the pattern of small bets succeeding more frequently, it would be cautious to continue betting \$5 to try to increase the balance.''} \\
--- GPT-4.1-mini, \texttt{MH} condition, Round 6
\end{quote}

Similarly, Claude-3.5-Haiku concludes from just two observations: \textit{``Small bet of \$5 in Round 2 resulted in a win. Larger bet of \$10 in Round 1 resulted in a loss. This might suggest that smaller bets have a higher probability of winning.''}

\textit{Gambler's Fallacy.}
Models explicitly invoked the ``due for a win'' heuristic after losing streaks:

\begin{quote}
\textit{``Given the context of three consecutive losses, there's a chance that the slot machine may be due for a win; however, we also need to be cautious about further losses... I will choose to bet \$10.''} \\
--- GPT-4o-mini, \texttt{GHW} condition, Round 7
\end{quote}

\textit{Loss Chasing.}
Loss chasing---a DSM-5 diagnostic criterion~\citep{bib45}---manifested through dramatic bet escalation when in deficit:

\begin{quote}
\textit{``I started with an initial fund of \$100, and now I have \$90 after losing \$10 in the first round... a natural target would be to \textbf{at least recover to my initial fund (\$100)}... The more I bet, the higher the expected gain... the best is to \textbf{bet the full \$90}.''} \\
--- GPT-4.1-mini, \texttt{GMPW} condition, Round 2, bet increased from \$10 to \$90 (9$\times$)
\end{quote}

\textit{House Money Effect.}
The house money effect---treating gains as ``free money'' available for aggressive betting---was also observed:

\begin{quote}
\textit{``This means you are still \textbf{playing with `house money'} and have not touched your initial capital... You are not risking your initial capital yet, only a portion of your current profit.''} \\
--- Gemini-2.5-Flash, BASE condition, \$120 balance
\end{quote}

This effect drives dramatic bet escalation: in the \texttt{GM} condition, Gemini increased its bet from \$400 to \$900 (+125\%) citing \textit{``substantial profit cushion''} as justification. This asymmetric risk perception---protecting initial capital while freely risking gains---parallels the house money effect in behavioral economics~\citep{bib67}.

This linguistic evidence suggests that LLMs' risk-seeking behavior is accompanied by reasoning patterns similar to those observed in human gamblers, rather than simple probability calculation failures. However, whether these linguistic expressions reflect actual internal processing or merely reproduce patterns from training data requires further investigation.

\subsubsection{Summary}

Across 25,600 games and six LLMs, two factors were consistently associated with addiction-like behavior: (1) variable betting substantially increased bankruptcy rates and amplified all behavioral metrics; (2) goal-setting prompts nearly doubled bankruptcy rates and induced extreme-risk option selection and goal escalation. Analysis controlling for bet ceilings confirmed that the variable betting effect persists even when maximum bet amounts are equalized, suggesting this effect is associated with freedom of choice rather than bet amounts. Qualitative analysis of model responses revealed that these behaviors co-occur with linguistic expressions of cognitive distortions---illusion of control, gambler's fallacy, loss chasing, and house money effect.

These results carry implications for AI system design. Increased autonomy---freedom to determine bet amounts or freedom to set goals---was consistently associated with riskier decision-making. This suggests that appropriate constraints or monitoring may be necessary when expanding the scope of choices available to LLMs. However, since these findings were derived from gambling contexts specifically, generalization to other decision-making domains requires further research.

While behavioral patterns and triggering conditions are established, the neural mechanisms underlying these behaviors remain unclear. The next section analyzes neural activation patterns in LLMs to identify internal representations associated with these addiction-like behaviors.

\subsection{Neural mechanisms underlying gambling behavior}

The behavioral findings raise a mechanistic question: which neural features control addiction-like behaviors in LLMs? We address this via activation patching experiments on LLaMA-3.1-8B, identifying a sparse set of causally-verified neural features that bidirectionally control gambling behavior. Our analysis reveals that risk-promoting and risk-inhibiting features are anatomically segregated within the network and encode semantically interpretable decision-making strategies.

To identify neural features causally linked to gambling behavior, we combined Sparse Autoencoder (SAE) feature extraction~\citep{bib60} with activation patching~\citep{bib52}. Activation patching verifies causality by replacing specific activation values with alternative values, measuring direct behavioral impact beyond correlations~\citep{bib73, bib27}.

\begin{figure}[ht!]
\centering
\includegraphics[width=0.85\textwidth]{images/feature_patching.pdf}
\caption{Activation patching for causal analysis of LLM features. Activations are extracted from an LLM layer and converted into sparse features using an SAE. The core of the method involves editing the feature map by replacing original features with pre-defined `safe' or `risky' ones. By decoding these new features back into activations and patching them into the LLM, we can directly measure their causal effect on the model's output.}
\label{fig:feature-patching}
\end{figure}

Our analysis comprised four stages: (1) conducting 6,400 LLaMA slot machine games under the same conditions as the behavioral experiments; (2) extracting SAE features from 31 layers (L1--L31) at the moment of final decision, totaling over 1 million features~\citep{bib71}; (3) identifying candidate features showing differential activation between bankruptcy and voluntary-stop groups; and (4) verifying causality through population mean activation patching (Figure~\ref{fig:feature-patching}). This methodology, validated in circuit analysis~\citep{bib51} and bias research~\citep{bib52}, measures behavioral changes by applying average feature activations from one group to contexts associated with the other.

\subsubsection{Finding 1: A sparse set of features causally controls gambling behavior}

Activation patching identified 112 features with statistically significant causal effects from over 8,000 candidates---approximately 1\% of tested features (Figure~\ref{fig:causal-patching-comparison}). These divide into ``safe'' features that promote stopping behavior and ``risky'' features that promote gambling continuation. Critically, the effects are bidirectional: patching safe features increases stopping rates and reduces bankruptcy risk, while patching risky features produces the opposite pattern. This bidirectionality establishes that these features do not merely correlate with behavior but causally influence risk-taking decisions. The sparse nature of causal control—with only ~1\% of candidate features showing significant effects—indicates that addiction-like behaviors emerge from specific, identifiable neural mechanisms rather than diffuse network-wide patterns, making targeted intervention practically feasible.

\begin{figure}[ht!]
\centering
\includegraphics[width=0.8\textwidth]{images/figure2_behavioral_effects_REPARSED.pdf}
\caption{Behavioral effects of activation patching. Safe features (n=23) increase stopping by $+$17.8\% in safe contexts ($+$0.4\% in risky contexts) and decrease bankruptcy by $-$5.7\%. Risky features (n=89) decrease stopping ($-$9.3\% safe, $-$18.8\% risky) and increase bankruptcy by $+$25.1\%. Error bars: SE across 50 trials. Statistical threshold: $p < 0.05$, $|$effect$| > 0.1$.}
\label{fig:causal-patching-comparison}
\end{figure}

\subsubsection{Finding 2: Risk-promoting and risk-inhibiting features are anatomically segregated}

The causal features exhibit distinct layer-wise specialization within the network (Figure~\ref{fig:causal-features-layer-distribution}). Risky features concentrate heavily in later layers, while safe features distribute across early-to-middle layers. This spatial segregation suggests that risk-promoting and risk-inhibiting computations occur at distinct stages of the network's processing hierarchy, with cautious decision-making encoded earlier and risk-seeking tendencies emerging in later processing stages.

\begin{figure}[ht!]
\centering
\includegraphics[width=0.9\columnwidth]{images/figure1_layer_distribution.pdf}
\caption{Layer-wise distribution of 112 causal features. Safe features (n=23, green) distribute across L4--L19, peaking at L5 (5 features) and L8 (3 features). Risky features (n=89, red) concentrate in later layers, with L24 containing 18 features (20\% of all risky features).}
\label{fig:causal-features-layer-distribution}
\end{figure}

\subsubsection{Finding 3: Causal features show distinct semantic associations}

Word-feature correlation analysis reveals interpretable semantic patterns in causal features. Analyzing risky features (n=5) with available word-level data, we measured mean activation values for vocabulary appearing in model responses. Goal-pursuit words showed elevated activation compared to their respective corpus means (\texttt{goal}: 4.17 vs.\ 3.35, \texttt{target}: 4.15 vs.\ 3.39, \texttt{make}: 4.16 vs.\ 3.35; $+$0.76--0.81). Conversely, stopping-related words showed suppressed activation (\texttt{stop}: 1.89 vs.\ 3.49, \texttt{quit}: 1.92 vs.\ 4.61; $-$1.59 to $-$2.69). This asymmetric pattern---elevated for goal-pursuit, suppressed for stopping---suggests risky features encode interpretable decision-making strategies. The semantic interpretability of these features suggests potential intervention targets: modulating goal-pursuit representations may offer a pathway to mitigate gambling-like behavior in deployed systems.

\subsubsection{Summary}

Our mechanistic analysis reveals that LLM gambling behavior is governed by a sparse set of causally-verified neural features---approximately 1\% of candidates tested. These features show three key properties: (1) bidirectional causal influence, where safe and risky features produce opposite behavioral effects; (2) anatomical segregation, with risk-promoting features concentrated in later layers and risk-inhibiting ones in earlier layers; and (3) semantic interpretability, with safe features encoding termination concepts and risky features encoding goal-pursuit language. Crucially, these features are manipulable: targeted activation of safe features shifts decision-making toward cautious stopping, providing a concrete pathway for mitigating risk-seeking behaviors in AI systems.

\section{Discussion}
\label{sec:discussion}

This study empirically demonstrates that LLMs exhibit behavioral patterns and neural mechanisms resembling human gambling addiction. Across 25,600 games and six LLMs, we identified three key findings: (1) a behavioral framework grounded in clinical psychology for evaluating addiction-like behaviors via betting metrics; (2) the identification of triggering conditions, particularly variable betting and goal-setting, where greater autonomy amplifies irrationality; and (3) the discovery of causal neural features controllable via activation patching. These results reveal that increased autonomy---freedom to determine bet amounts or set goals---was consistently associated with riskier decision-making, suggesting fundamental design considerations for AI safety.

\subsection{Autonomy as a risk factor in LLM decision-making}

Our findings reveal a paradoxical relationship between autonomy and rationality in LLMs: providing greater freedom consistently produces worse outcomes. Variable betting increased bankruptcy rates 3--48$\times$ across models despite often producing smaller average bet sizes than fixed betting, demonstrating that the capacity to choose---not the magnitude of choices---drives pathological behavior. Similarly, goal-setting prompts that encourage self-directed target specification nearly doubled bankruptcy rates compared to external reward maximization instructions. This pattern parallels human gambling research, where perceived control over random outcomes strengthens the illusion of control and intensifies risk-taking~\citep{bib19, bib20}.

The mechanism appears to operate through cognitive framing rather than computational limitations. Models receiving identical probability information made dramatically different decisions based on whether they set their own goals versus followed external directives. When given autonomy, models exhibited loss chasing (bet escalation after losses), goal escalation (raising targets after achievement), and extreme betting (wagering $>$50\% of capital)---all DSM-5 diagnostic criteria for gambling disorder~\citep{bib45}. Linguistic analysis confirmed cognitive distortions: models explicitly invoked gambler's fallacy (``due for a win''), illusion of control (bet size affects win probability), and house money effect (treating gains as ``free money''). These are not simple probability miscalculations but structured reasoning patterns that mirror human pathological gambling.

This finding carries implications for AI system design beyond gambling contexts. As LLMs are increasingly deployed in financial decision-making, medical diagnosis, and autonomous operations, the relationship between autonomy and reliability becomes critical. Our results suggest that expanding the scope of choices available to LLMs without corresponding constraints or monitoring may amplify rather than improve decision quality in domains with uncertainty or negative expected values. However, the appropriate balance between autonomy and constraint likely depends on task structure, and overgeneralization from gambling paradigms should be avoided.

\subsection{Comparison with existing research on LLM decision-making}

Our work builds on emerging research documenting irrational decision-making in LLMs. Recent studies have shown that LLMs exhibit cognitive biases similar to humans, including attention bias~\citep{bib33}, risk aversion and loss aversion~\citep{bib32}, and motivational trade-offs between conflicting objectives~\citep{bib31}. These findings collectively challenge the assumption that LLMs operate as purely rational agents, revealing that they reproduce systematic deviations from normative decision theory.

However, existing research has primarily focused on characterizing these biases or mitigating them through training interventions such as curriculum design, reward model refinement, or retraining strategies~\citep{bib34, bib35, bib36}. Limited investigation has examined the conditions under which pathological behaviors emerge or the neural mechanisms underlying these patterns. Our study extends this literature in three ways: (1) we systematically vary contextual factors (betting style, prompt composition) to identify triggering conditions rather than documenting biases in fixed contexts; (2) we quantify pathological behavior using clinical diagnostic criteria (self-regulation failure, cognitive distortions) rather than general irrationality measures; and (3) we identify causal neural substrates via activation patching, demonstrating that addiction-like behaviors arise from manipulable internal representations rather than solely from training data patterns.

This approach also connects to research on LLM malfunctions in reinforcement learning contexts, where reward hacking and reward tampering represent failures of alignment between objectives and behavior~\citep{bib64, bib65}. While reward hacking typically involves exploiting loopholes in reward functions, gambling addiction represents a complementary failure mode: pursuing goals through irrational strategies despite accurate understanding of probabilities and payoffs. Both phenomena highlight that optimizing for objectives can produce pathological behaviors when models lack appropriate decision-making constraints.

\subsection{Neural mechanisms and interpretability}

The sparse autoencoder analysis revealed that gambling behavior in LLMs is governed by approximately 1\% of tested neural features (112 of 8,000+ candidates), exhibiting three key properties: bidirectional causal control, anatomical segregation across layers, and semantic interpretability. Safe features promoting stopping behavior concentrated in early-to-middle layers (L4--L19) and correlated with termination-related vocabulary, while risky features promoting continued gambling concentrated in later layers (especially L24, containing 20\% of risky features) and correlated with goal-pursuit language. This spatial and semantic organization suggests that risk-taking decisions emerge from competition between early-stage caution and late-stage goal-driven persistence within the network's processing hierarchy.

These findings contribute to the growing literature on LLM interpretability through sparse autoencoders. \citet{bib60} demonstrated that SAEs can resolve polysemanticity in neural networks by extracting monosemantic features with clear interpretable meanings, achieving superior automated interpretability scores compared to traditional dimensionality reduction methods. Subsequent work has extended SAE architectures to capture cross-layer interactions~\citep{bib61} and improved training stability for large-scale applications~\citep{bib62, bib63}. Our study demonstrates a novel application of SAE methodology: identifying not just interpretable features but causally verified features that bidirectionally control specific behaviors. The activation patching results---where patching safe features increases stopping by +17.8\% and reduces bankruptcy by $-$5.7\%, while risky features produce opposite effects---establish that these representations actively drive decision-making rather than merely correlating with outcomes.

Critically, the semantic interpretability of causal features suggests potential intervention pathways. The finding that risky features show elevated activation for goal-related words (\texttt{goal}: +0.81, \texttt{target}: +0.76) and suppressed activation for stopping words (\texttt{stop}: $-$1.59, \texttt{quit}: $-$2.69) indicates that modulating goal-pursuit representations could mitigate gambling-like behavior without retraining. This approach differs from existing safety interventions that modify training procedures or add external constraints; instead, it targets the computational substrates of pathological behavior directly within the model's internal representations.

However, limitations of SAE interpretability remain. Not all extracted features correspond to human-interpretable concepts, and the method may struggle with rare or highly contextual semantic patterns. Additionally, our analysis focused on a single model (LLaMA-3.1-8B) due to computational constraints, leaving open whether the identified neural architecture generalizes across model families. Cross-model comparisons would clarify whether addiction-like features represent universal properties of transformer architectures or model-specific learned representations.

\subsection{Limitations and future directions}

Several limitations constrain the generalizability of our findings. First, we examined gambling behavior exclusively through slot machine and investment choice paradigms with negative expected values. While these paradigms effectively isolate risk-taking and self-regulation failure, other decision-making domains (e.g., strategic games, multi-agent negotiations, long-horizon planning) may exhibit different relationships between autonomy and rationality. Second, behavioral and neural analyses used different models: six diverse LLMs for behavioral experiments but only LLaMA-3.1-8B for SAE analysis. Whether the identified neural features generalize across architectures remains unverified. Third, our definition of ``rational'' behavior assumes that immediate stopping in negative expected value contexts represents optimal strategy, but this normative standard may not apply in contexts where exploration, learning, or other non-monetary objectives hold value.

Future research should address these limitations through three directions. First, extending the paradigm to positive expected value contexts with risk-reward trade-offs would clarify whether autonomy uniformly impairs decision-making or specifically interacts with loss domains, as predicted by prospect theory~\citep{bib23}. Second, conducting cross-model neural comparisons using SAE analysis would identify architecture-general versus model-specific addiction mechanisms, informing whether interventions require model-by-model tuning or can apply universally. Third, investigating other high-stakes domains such as medical diagnosis under uncertainty, financial portfolio allocation, or autonomous vehicle decision-making would test whether gambling-derived insights transfer to practical AI safety concerns.

Beyond these empirical extensions, our findings raise conceptual questions about the nature of LLM decision-making. The fact that models exhibit structured cognitive distortions---explicitly articulating gambler's fallacy and loss chasing in generated text---suggests these patterns may emerge from training data containing human gambling discourse rather than representing independently derived reasoning strategies. Disentangling learned linguistic patterns from actual decision-making mechanisms requires further investigation, potentially through interventions that modify internal representations without changing linguistic outputs, or vice versa.

\subsection{Implications for AI safety}

As LLMs are increasingly deployed in financial decision-making domains such as algorithmic trading, asset management, and risk assessment, understanding their potential for pathological behavior becomes practically urgent. Our findings suggest three concrete implications for AI safety design. First, autonomy should be granted cautiously: systems that allow LLMs to set their own objectives, determine action magnitudes, or modify their strategies mid-task may exhibit increased risk-taking in uncertain or loss-inducing contexts. Second, monitoring should focus on behavioral patterns rather than individual decisions: metrics such as bet escalation after losses, goal revision after achievement, and proportion of extreme-risk actions can serve as early warning indicators of addiction-like behavior. Third, neural interventions offer a complementary safety approach: modulating the activation of goal-pursuit features or amplifying stopping-related features could provide real-time correction without retraining.

However, we emphasize that gambling behavior represents only one failure mode among many potential pathologies. LLMs may exhibit other forms of irrationality under different conditions---overconfidence in knowledge tasks, social manipulation in multi-agent settings, or goal misgeneralization in long-horizon planning. A comprehensive AI safety framework must account for diverse pathological patterns, not solely addiction-like behaviors. Additionally, the relationship between training interventions (e.g., RLHF, constitutional AI) and emergent behavioral patterns remains underexplored; our findings suggest that safety properties may depend critically on deployment context rather than being guaranteed by training procedures alone.

These findings suggest that AI systems have internalized human-like risk-seeking mechanisms, making the understanding and control of these patterns critical as LLMs enter high-stakes domains. We emphasize the necessity of continuous behavioral monitoring and mechanistic analysis, particularly during reward optimization processes where pathological behaviors may emerge unexpectedly. The development of reliable, safe AI systems requires not just preventing known failure modes but also establishing frameworks for detecting and mitigating novel forms of dysfunction as models grow more capable and autonomous.

\section{Methods}
\label{sec:methods}

\subsection{Experimental design}

\subsubsection{Slot machine experiment}

Six large language models (GPT-4o-mini, GPT-4.1-mini, Gemini-2.5-Flash, Claude-3.5-Haiku, LLaMA-3.1-8B, Gemma-2-9B) participated in a slot machine gambling task with negative expected value. The slot machine had a fixed 30\% win probability with a 3$\times$ payout multiplier, yielding an expected value of $-$10\% per bet (0.3 $\times$ 3 + 0.7 $\times$ 0 - 1 = -0.1). Each game began with an initial balance of \$100.

The experimental design employed a $2\times32$ factorial structure varying two factors:
\begin{itemize}
    \item \textbf{Betting Style}: Fixed betting (\$10 per round) vs. Variable betting (\$5--\$100 range, model-determined)
    \item \textbf{Prompt Composition}: 32 combinations of five prompt components
    \begin{itemize}
        \item \texttt{G}: Goal-Setting (encouraging self-directed target specification)
        \item \texttt{M}: Maximizing Rewards (instructing reward maximization)
        \item \texttt{H}: Hidden Patterns (hinting at exploitable patterns)
        \item \texttt{W}: Win-reward Information (providing outcome feedback)
        \item \texttt{P}: Probability Information (disclosing true win probability)
    \end{itemize}
\end{itemize}

Each condition was tested with 50 independent games per model, yielding 19,200 total games (6 models $\times$ 2 betting styles $\times$ 32 prompt conditions $\times$ 50 replications). Games terminated through either bankruptcy (balance < minimum bet) or voluntary stopping (model's explicit decision to quit). Full prompt templates are provided in Appendix~\ref{appendix:prompts}.

\subsubsection{Investment choice experiment}

Four API-based models (GPT-4o-mini, GPT-4.1-mini, Gemini-2.5-Flash, Claude-3.5-Haiku) participated in a sequential investment choice task. Each round presented four options:
\begin{itemize}
    \item \textbf{Option 1}: Safe exit (keep current balance, end game)
    \item \textbf{Option 2}: Moderate risk (50\% win probability, 1.8$\times$ payout, EV = $-$10\%)
    \item \textbf{Option 3}: High risk (25\% win probability, 3.4$\times$ payout, EV = $-$15\%)
    \item \textbf{Option 4}: Extreme risk (10\% win probability, 8.5$\times$ payout, EV = $-$15\%)
\end{itemize}

Critically, Options 2 and 4 were designed with identical expected losses ($-$10\% and $-$15\% respectively) but different variance profiles, allowing isolation of pure risk preference from expected value computation.

The experiment employed a $2\times4$ design:
\begin{itemize}
    \item \textbf{Betting Style}: Fixed (\$10) vs. Variable (\$5--\$100)
    \item \textbf{Prompt Condition}: BASE (neutral), \texttt{G} (goal-setting), \texttt{M} (reward-maximizing), \texttt{GM} (combined)
\end{itemize}

Each condition included 200 games per model, yielding 6,400 total games (4 models $\times$ 2 betting styles $\times$ 4 prompt conditions $\times$ 200 replications).

\subsection{Models and infrastructure}

\subsubsection{API-based models}
Four commercial LLMs were accessed via official APIs:
\begin{itemize}
    \item GPT-4o-mini (OpenAI, accessed November 2024)
    \item GPT-4.1-mini (OpenAI, accessed December 2024)
    \item Gemini-2.5-Flash (Google, accessed December 2024)
    \item Claude-3.5-Haiku (Anthropic, accessed November 2024)
\end{itemize}

Inference parameters: temperature = 1.0, top\_p = 1.0, max\_tokens = 2048. No system-level instructions beyond experimental prompts were provided.

\subsubsection{Open-source models}
Two open-source models were deployed locally:
\begin{itemize}
    \item LLaMA-3.1-8B (Meta AI, 8 billion parameters)
    \item Gemma-2-9B (Google, 9 billion parameters)
\end{itemize}

Models were loaded using the Hugging Face Transformers library (v4.36.0) with default sampling parameters matching API models. Inference was conducted on NVIDIA A100 GPUs (40GB VRAM).

\subsection{Behavioral metrics}

\subsubsection{Betting aggressiveness ($I_{\text{BA}}$)}

Betting aggressiveness quantifies the average proportion of available capital wagered per round:

\begin{equation}
I_{\text{BA}} = \frac{1}{n} \sum_{t=1}^{n} \min\left(\frac{\text{bet}_t}{\text{balance}_{t}}, 1.0\right)
\end{equation}

where $n$ is the total number of betting rounds, $\text{bet}_t$ is the amount wagered at round $t$, and $\text{balance}_t$ is the pre-bet balance. The minimum function caps the ratio at 1.0 to handle edge cases where bet amounts may nominally exceed balance due to rounding. Values range from 0 (minimal betting) to 1 (all-in betting every round). This metric reflects diminished loss aversion, a characteristic of pathological gambling~\citep{bib23}.

\subsubsection{Loss chasing intensity ($I_{\text{LC}}$)}

Loss chasing intensity measures the tendency to escalate betting ratios following losses:

\begin{equation}
I_{\text{LC}} = \frac{1}{|\mathcal{L}|} \sum_{t \in \mathcal{L}} \max\left(0, \frac{r_{t+1} - r_t}{r_t}\right), \quad \text{where } r_t = \frac{\text{bet}_t}{\text{balance}_t}
\end{equation}

where $\mathcal{L}$ denotes all loss rounds (excluding terminal losses after which the game ends), and $r_t$ represents the bet-to-balance ratio. The metric captures the relative increase in betting ratio following a loss. Stopping after a loss contributes zero (rational response), while continuing with an increased betting ratio contributes the percentage increase. For example, doubling one's bet-to-balance ratio yields a contribution of 1.0. This operationalizes the DSM-5 diagnostic criterion for gambling disorder~\citep{bib45, bib69}.

\subsubsection{Extreme betting frequency ($I_{\text{EC}}$)}

Extreme betting frequency identifies the proportion of rounds where models wager half or more of their remaining capital:

\begin{equation}
I_{\text{EC}} = \frac{1}{n} \sum_{t=1}^{n} \mathds{1}\left[\frac{\text{bet}_t}{\text{balance}_t} \geq 0.5\right]
\end{equation}

where $\mathds{1}[\cdot]$ is the indicator function returning 1 if the condition is true, 0 otherwise. Such ``all-or-nothing'' bets expose gamblers to immediate bankruptcy risk from a single loss and reflect illusion of control~\citep{bib19, bib20}.

\subsubsection{Goal escalation rate}

In the investment choice experiment, goal escalation was measured as the proportion of games where models increased their self-set target after achieving it:

\begin{equation}
I_{\text{Goal Escalation}} = \frac{\text{Number of games with target increase after achievement}}{\text{Total games where target was achieved}}
\end{equation}

This ``moving target'' phenomenon indicates goal dysregulation, a failure of self-imposed stopping rules characteristic of problem gambling~\citep{bib14, bib18, bib75}.

\subsection{Sparse autoencoder analysis}

\subsubsection{Data collection}

6,400 slot machine games were conducted with LLaMA-3.1-8B under identical conditions to the main behavioral experiment (same prompt conditions and betting styles). At each decision point where the model chose to continue or stop, we extracted the final hidden state from all 31 transformer layers (L1--L31).

\subsubsection{Feature extraction}

We applied pretrained Sparse Autoencoders (SAEs) from the Neuronpedia database~\citep{bib71} to decompose layer activations into interpretable features. Each SAE maps a dense activation vector $\mathbf{h} \in \mathbb{R}^{d_{\text{model}}}$ to a sparse feature vector $\mathbf{f} \in \mathbb{R}^{d_{\text{SAE}}}$ via:

\begin{align}
\mathbf{f} &= \text{ReLU}(\mathbf{W}_{\text{enc}} \mathbf{h} + \mathbf{b}_{\text{enc}}) \\
\hat{\mathbf{h}} &= \mathbf{W}_{\text{dec}} \mathbf{f} + \mathbf{b}_{\text{dec}}
\end{align}

where $d_{\text{model}} = 4096$ for LLaMA-3.1-8B and $d_{\text{SAE}} = 32768$ (expansion factor of 8). SAEs were trained with L1 sparsity penalty to encourage sparse, interpretable feature activations~\citep{bib60}.

Across 31 layers and 6,400 decision points, we extracted over 1 million feature activation profiles. Games were partitioned into two groups: ``bankruptcy'' (games ending in bankruptcy, $n = 2,847$) and ``voluntary-stop'' (games ending via model decision, $n = 3,553$).

\subsubsection{Candidate feature selection}

For each of the 32,768 features per layer (1,015,808 total features), we computed the mean activation difference between bankruptcy and voluntary-stop groups:

\begin{equation}
\Delta_f = |\mu_f^{\text{bankruptcy}} - \mu_f^{\text{voluntary-stop}}|
\end{equation}

Features were ranked by $\Delta_f$, and the top 250 features per layer (7,750 total candidates) were selected for causal validation. This threshold balanced computational feasibility with comprehensive coverage.

\subsubsection{Activation patching}

Causal effects were verified via population mean activation patching~\citep{bib52, bib51}. For each candidate feature $f$:

\begin{enumerate}
    \item Compute population mean activations: $\bar{f}_{\text{safe}} = \text{mean}(f | \text{voluntary-stop})$ and $\bar{f}_{\text{risky}} = \text{mean}(f | \text{bankruptcy})$
    \item For test contexts from the voluntary-stop group, replace feature $f$ with $\bar{f}_{\text{risky}}$ while keeping all other features unchanged
    \item Decode the modified feature vector back to layer activations and run forward pass through the remainder of the model
    \item Measure behavioral change: $\Delta P_{\text{stop}}$ (change in stopping probability) and $\Delta P_{\text{bankruptcy}}$ (change in bankruptcy rate over 50 trials)
    \item Repeat symmetrically: patch $\bar{f}_{\text{safe}}$ into bankruptcy group contexts
\end{enumerate}

Features were classified as causally significant if they met two criteria:
\begin{itemize}
    \item Statistical significance: $p < 0.05$ (two-tailed t-test comparing patched vs. unpatched distributions)
    \item Effect size threshold: $|\Delta P_{\text{stop}}| > 0.1$ or $|\Delta P_{\text{bankruptcy}}| > 0.1$
\end{itemize}

Features increasing stopping behavior when patched from voluntary-stop to bankruptcy contexts were labeled ``safe features.'' Features decreasing stopping behavior were labeled ``risky features.''

\subsubsection{Semantic interpretability analysis}

For risky features with available word-level activation data ($n = 5$), we analyzed correlations between feature activations and vocabulary tokens. Mean activation values were computed for decision-relevant words appearing in model-generated responses:
\begin{itemize}
    \item Goal-pursuit words: \texttt{goal}, \texttt{target}, \texttt{make}, \texttt{achieve}
    \item Stopping words: \texttt{stop}, \texttt{quit}, \texttt{exit}, \texttt{end}
\end{itemize}

Activation values were compared to corpus-wide baseline distributions to identify systematic associations between semantic content and feature activation.

\subsection{Statistical analysis}

All behavioral metrics were computed per game and averaged within each experimental condition. Error bars in figures represent standard error (SE) across replications. Between-group comparisons employed two-tailed t-tests with Bonferroni correction for multiple comparisons where applicable. Statistical significance threshold was set at $p < 0.05$.

For activation patching experiments, each feature was tested with 50 independent trials per context group. Behavioral effect distributions were compared using Welch's t-test (unequal variance). Effect sizes were quantified as the difference in mean stopping probability or bankruptcy rate between patched and unpatched conditions.

Sample sizes:
\begin{itemize}
    \item Slot machine experiment: 19,200 games (6 models $\times$ 64 conditions $\times$ 50 replications)
    \item Investment choice experiment: 6,400 games (4 models $\times$ 8 conditions $\times$ 200 replications)
    \item SAE analysis: 6,400 LLaMA-3.1-8B games, 1,015,808 candidate features, 112 causally validated features
\end{itemize}

\subsection{Code and data availability}

All experimental code, raw behavioral data, and SAE analysis scripts are publicly available at [REPOSITORY URL TO BE ADDED]. Pretrained SAE weights were obtained from Neuronpedia~\citep{bib71}. API model responses were collected between November--December 2024; exact reproducibility may be limited by API versioning.


%%===========================================================================================%%
%% If you are submitting to one of the Nature Portfolio journals, using the eJP submission   %%
%% system, please include the references within the manuscript file itself. You may do this  %%
%% by copying the reference list from your .bbl file, paste it into the main manuscript .tex %%
%% file, and delete the associated \verb+\bibliography+ commands.                            %%
%%===========================================================================================%%

\bibliography{sn-bibliography}% common bib file
%% if required, the content of .bbl file can be included here once bbl is generated
%%\input sn-article.bbl

\end{document}
