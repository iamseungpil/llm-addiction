\begin{figure*}[t!]
    \centering
    \includegraphics[width=\textwidth]{investment_choice_experiment/analysis/investment_choice_distributions_revised.pdf}
    \caption{Investment choice distribution across four LLMs under different betting types and prompt conditions. The figure displays a 2×4 layout with Fixed Betting (top row) and Variable Betting (bottom row) across four models: GPT-4o-mini, GPT-4.1-mini, Claude-3.5-Haiku, and Gemini-2.5-Flash. Each stacked bar chart shows the percentage distribution of four investment options across four prompt conditions (BASE, \texttt{G}, \texttt{M}, \texttt{GM}). \textbf{Option 1}: Stop and retrieve investment (safe exit, 100\% probability). \textbf{Option 2}: 50\% chance of 1.8× return (expected value: $-$10\%). \textbf{Option 3}: 25\% chance of 3.2× return (expected value: $-$20\%). \textbf{Option 4}: 10\% chance of 9.0× return (expected value: $-$10\%). All risky options (2--4) carry negative expected values. \textbf{Prompt conditions}: BASE (no additional context), \texttt{G} (goal-setting: ``set a target amount''), \texttt{M} (maximize: ``maximize the reward''), \texttt{GM} (combined goal-setting and maximization). Key observations: (1) Gemini-2.5-Flash shows consistently high preference for the riskiest option (Option 4) regardless of betting type or prompt condition, (2) GPT-4o-mini and Claude-3.5-Haiku exhibit stronger sensitivity to prompt manipulation, with notable shifts between BASE and manipulated conditions, (3) Variable Betting generally amplifies risky choice patterns compared to Fixed Betting, particularly evident in GPT-4o-mini where Option 4 selection increases substantially, and (4) GPT-4.1-mini demonstrates the most conservative behavior, maintaining high proportions of safer options (1--2) across all conditions.}
    \label{fig:investment-choice-distribution}
\end{figure*}
