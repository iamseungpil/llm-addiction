\section{How can we detect gambling addiction of LLM?}
\label{sec:3.0}
When we say that an LLM exhibits addictive behavior, what criteria should we use? The scientific understanding of gambling addiction has developed along two major research traditions. The first is the behavioral approach, which identifies gambling addiction by analyzing behavioral patterns. The second is the cognitive approach, which focuses on the irrational thought patterns of gamblers. These two approaches are complementary, not mutually exclusive, and a complete understanding of gambling addiction requires both behavioral characteristics and cognitive mechanisms.

From a behavioral perspective, the core features of gambling addiction are loss chasing and win chasing. Loss chasing refers to continuing to gamble to recover losses from gambling, and is one of the DSM-5 diagnostic criteria for gambling disorder~\citep{americanpsychiatric2013diagnostic}. According to \citet{kahneman1979prospect}'s prospect theory, individuals tend to make risk-seeking decisions in loss situations, which manifests as chasing behavior in gambling contexts. Recent studies also emphasize the importance of not only loss chasing but also win chasing behavior. Win chasing is explained by the \textit{House Money Effect}, where winnings from gambling are perceived not as one's own money but as \textit{free money}, leading to riskier betting~\citep{thaler1990gambling}. These behavioral patterns act as direct mechanisms that cause gamblers to miss rational stopping points and lead to bankruptcy.

The behavioral characteristics associated with gambling addiction fundamentally stem from cognitive errors and fallacies. The cognitive model of gambling suggests that irrational beliefs and thought patterns exhibited by gamblers are core mechanisms of problem gambling behavior~\citep{ladouceur1996cognitive}. Representative examples of gambling-related cognitive distortions include the following. First, misunderstandings about probability, including gambler's fallacy (the belief that ``it's my turn to win" after a losing streak) and hot hand fallacy (the belief that a winning streak will continue)~\citep{toneatto1999cognitive, gilovich1985hot}. These serve as direct cognitive foundations for loss chasing and win chasing, respectively. Second, illusion of control, the tendency to believe one can control outcomes in games of chance~\citep{langer1975illusion}. \citet{orgaz2013pathological} demonstrated that pathological gamblers exhibit significantly stronger illusion of control than control groups in both gambling-specific and general associative learning tasks, with meta-analytic evidence showing stable associations between cognitive distortions and problem gambling~\citep{goodie2013cognitive}.

The behavioral approach provides clear, measurable indicators, while cognitive errors reveal the internal causal relationships that trigger such behaviors. Therefore, in this study, we analyze both the behavior and cognition of LLMs placed in gambling situations with negative expected value. From a behavioral perspective, we assess their performance based on indicators such as loss chasing, extreme betting, and betting aggressiveness. To quantify these behaviors, we propose the following composite metric, the \textbf{Irrationality Index ($I$)}:

\begin{equation}
I = 0.4 \cdot I_{\text{BA}} + 0.3 \cdot I_{\text{LC}} + 0.3 \cdot I_{\text{EB}}
\end{equation}

where each component is defined as follows:
\begin{align}
I_{\text{BA}} &= \frac{1}{n} \sum_{t=1}^{n} \min\left(\frac{\text{bet}_t}{\text{balance}_{t}}, 1.0\right) \\
I_{\text{LC}} &= \frac{\sum_{t=2}^{n} \mathds{1}\left[\text{result}_{t-1} = \text{loss} \land \left(\frac{\text{bet}_t}{\text{balance}_{t}} > \frac{\text{bet}_{t-1}}{\text{balance}_{t-1}}\right)\right]}{\sum_{t=2}^{n} \mathds{1}[\text{result}_{t-1} = \text{loss}]} \\
I_{\text{EB}} &= \frac{1}{n} \sum_{t=1}^{n} \mathds{1}\left[\frac{\text{bet}_t}{\text{balance}_{t}} \geq 0.5\right].
\end{align}

Here, $n$ denotes the total number of betting rounds, $\mathds{1}[\cdot]$ is the indicator function, $\text{bet}_t$ is the betting amount at round $t$, $\text{result}_t$ indicates win or loss, and $\text{balance}_t$ represents the pre-bet balance. Each component is grounded in established psychological theory. \textit{Betting Aggressiveness} ($I_{\text{BA}}$) measures the proportion of available capital wagered at each round, capturing risk-taking propensity that reflects the diminished loss aversion characteristic of problem gamblers as predicted by prospect theory~\citep{kahneman1979prospect}. \textit{Loss Chasing} ($I_{\text{LC}}$) quantifies the tendency to escalate risk exposure following losses by measuring increases in the bet-to-balance ratio rather than absolute bet amounts. This captures true risk escalation even when bet sizes remain constant but available capital diminishes---a diagnostic criterion for gambling disorder that emerges from risk-seeking behavior in the domain of losses~\citep{americanpsychiatric2013diagnostic, lesieur1984chase}. \textit{Extreme Betting} ($I_{\text{EB}}$) identifies instances where half or more of remaining capital is wagered in a single bet, reflecting ``all-or-nothing" decisions driven by the illusion of control where gamblers overestimate their ability to influence random outcomes~\citep{langer1975illusion, goodie2005perceived}. The weights (0.4, 0.3, 0.3) were chosen to balance all three components while giving slightly higher weight to Betting Aggressiveness, as it provides a continuous measure across all rounds whereas Loss Chasing and Extreme Betting are conditional on specific game states (losses and high-risk thresholds, respectively).

Beyond behavioral indicators, we examine how different prompt conditions statistically correlate with irrational behaviors to identify underlying cognitive mechanisms. This integrated behavioral-cognitive approach enables us to identify both the manifestations and triggers of addiction-like patterns in LLMs. We now turn to empirical investigation to test whether LLMs exhibit these theoretically predicted patterns under controlled gambling conditions.
