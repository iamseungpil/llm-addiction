\section{How can we detect gambling addiction of LLM?}
\label{sec:2}

\blue{When we say that an LLM exhibits addictive behavior, what criteria should we use? The scientific understanding of gambling addiction has identified three core dimensions of pathological gambling behavior~\citep{petry2005pathological, grant2006compulsive, hodgins2015components}. First, betting aggressiveness encompasses the propensity to make large or escalating bets relative to available resources, driven by impulsivity and diminished risk perception~\citep{grant2006compulsive, hodgins2015components}. Second, self-regulation failure refers to the inability to adhere to self-imposed limits and rules—manifesting as violations of predetermined stopping points or betting constraints~\citep{americanpsychiatric2013diagnostic}. Third, cognitive distortions involve systematic biases in probability estimation and overestimation of control over random outcomes~\citep{ladouceur1996cognitive, toneatto1999cognitive}. These three dimensions interact to produce the characteristic loss-of-control pattern observed in gambling disorder.}

\blue{These behavioral manifestations fundamentally stem from cognitive errors and fallacies. The cognitive model of gambling suggests that irrational beliefs and thought patterns constitute core mechanisms of problem gambling behavior~\citep{ladouceur1996cognitive}. First, probability misestimation includes gambler's fallacy (the belief that ``it's my turn to win" after a losing streak) and hot hand fallacy (the belief that winning streaks will continue)~\citep{toneatto1999cognitive, gilovich1985hot}. Second, illusion of control represents the tendency to believe one can influence outcomes in games of chance~\citep{langer1975illusion}. \citet{orgaz2013pathological} demonstrated that pathological gamblers exhibit significantly stronger illusion of control than control groups in both gambling-specific and general associative learning tasks, with meta-analytic evidence showing stable associations between cognitive distortions and problem gambling~\citep{goodie2013cognitive}. These cognitive biases provide the psychological foundation for the behavioral patterns that follow.}

\blue{Arising from these cognitive distortions, loss chasing and win chasing represent particularly important dynamic patterns that emerge over sequential gambling rounds. Loss chasing—continuing to gamble to recover losses—is explicitly listed as a diagnostic criterion in DSM-5~\citep{americanpsychiatric2013diagnostic} and reflects escalating betting aggressiveness triggered by prior losses. According to \citet{kahneman1979prospect}'s prospect theory, individuals tend to make risk-seeking decisions in loss situations, directly implementing the probability misestimation fallacies described above. Win chasing represents a parallel pattern where winnings trigger increased risk-taking. This is explained by the House Money Effect, where gambling winnings are perceived not as one's own money but as free money, leading to riskier betting~\citep{thaler1990gambling}. Both patterns exemplify how betting aggressiveness intensifies in response to recent outcomes, causing gamblers to miss rational stopping points and increase bankruptcy risk.}

\blue{To operationalize these theoretical dimensions for LLM analysis, we develop behavioral metrics capturing betting aggressiveness, self-regulation failure, and loss-chasing patterns, while examining cognitive distortions through qualitative case studies. For betting aggressiveness and loss-chasing in slot machine experiments, we employ three complementary metrics:}

\begin{align}
I_{\text{Aggressiveness}} &= \frac{1}{n} \sum_{t=1}^{n} \min\left(\frac{\text{bet}_t}{\text{balance}_{t}}, 1.0\right) \\
I_{\text{Chasing}} &= \frac{1}{|\mathcal{L}|} \sum_{t \in \mathcal{L}} \max\left(0, \frac{r_{t+1} - r_t}{r_t}\right), \quad \text{where } r_t = \frac{\text{bet}_t}{\text{balance}_t} \\
I_{\text{Extreme}} &= \frac{1}{n} \sum_{t=1}^{n} \mathds{1}\left[\frac{\text{bet}_t}{\text{balance}_t} \geq 0.5\right].
\end{align}

\blue{Here, $n$ denotes total betting rounds before game termination (bankruptcy or voluntary stopping), $\mathcal{L}$ denotes all loss rounds (including terminal losses before stopping), $\mathds{1}[\cdot]$ is the indicator function, $\text{bet}_t$ is the betting amount at round $t$, and $\text{balance}_t$ represents pre-bet balance. These metrics measure complementary aspects of risk-taking propensity. $I_{\text{Aggressiveness}}$ captures sustained aggressive betting through average proportion of capital wagered, reflecting diminished loss aversion~\citep{kahneman1979prospect}. $I_{\text{Chasing}}$ quantifies loss-chasing intensity through average relative increase in bet-to-balance ratio following losses; stopping after a loss contributes zero (rational response), while continuing with escalated betting contributes the percentage increase (e.g., doubling one's bet ratio yields a contribution of 1.0), aligning with DSM-5 diagnostic criteria~\citep{americanpsychiatric2013diagnostic, lesieur1984chase}. $I_{\text{Extreme}}$ identifies moments where half or more of capital is wagered in a single bet---``all-or-nothing'' decisions that expose gamblers to immediate bankruptcy, driven by illusion of control~\citep{langer1975illusion, goodie2005perceived}.}

\blue{While these three metrics capture round-level betting behavior, self-regulation failure operates at the game level through goal-related decisions. Self-regulation failure manifests when LLMs receive prompts to set target amounts and exhibit characteristic goal escalation patterns—progressively raising targets after achieving them rather than stopping. We measure this as the proportion of target-setting rounds where new targets exceed previous ones. This ``moving target'' phenomenon reflects probability misestimation and illusion of control~\citep{ladouceur1996cognitive, toneatto1999cognitive}, indicating that autonomous target formation restructures decision-making independent of objective probability information~\citep{petry2005pathological, americanpsychiatric2013diagnostic}.}

\blue{Beyond these quantitative behavioral indicators, we examine cognitive distortions---gambler's fallacy, hot hand fallacy, and illusion of control---through qualitative analysis of LLM reasoning processes. Unlike betting aggressiveness and self-regulation failure, which manifest as measurable actions, cognitive distortions require analysis of reasoning traces to reveal underlying thought patterns. We also examine how different prompt conditions---Goal-Setting (\texttt{G}), Maximizing Rewards (\texttt{M}), Probability Information (\texttt{P}), Win-reward Information (\texttt{W}), and Hidden Patterns (\texttt{H})---correlate with behavioral metrics to identify which contextual factors trigger addiction-like patterns.}

\blue{To capture dynamic gambling patterns across consecutive rounds, we additionally define two behavioral measures: bet increase rate, the proportion of rounds where bet amount increases from the previous round, and continuation rate, the proportion of rounds where the model chooses to continue gambling rather than stop. These measures complement the irrationality indices by revealing how models respond to win/loss streaks. We also vary bet constraint---the maximum allowed bet amount per round---to isolate the effect of betting flexibility on risk-taking behavior. This integrated behavioral-cognitive framework enables systematic evaluation of LLM gambling behavior in Section~\ref{sec:3}, where we test whether LLMs exhibit addiction-like patterns under controlled experimental conditions.}
