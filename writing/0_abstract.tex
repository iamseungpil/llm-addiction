\begin{abstract}
본 연구는 대규모 언어 모델이 인간의 도박 장애와 유사한 행동 패턴을 발현할 수 있는지 탐구한다. LLM이 자산 운용, 상품 거래 등 금융 의사결정 영역에 점차 활용되면서, 이들이 병리적 의사결정을 할 가능성에 대한 연구는 실질적 중요성을 갖게 되었다. 본 연구는 인지심리학의 Cognitive Distortion Theory를 이론적 토대로 삼아, LLM의 의사결정 과정을 행동-인지-신경의 세 층위에서 체계적으로 분석하였다. 먼저, GPT-4o-mini 3,200개 실험과 LLaMA-3.1-8B 6,400개 실험을 통한 슬롯머신 과제에서, 각각 5.7%와 3.3%의 파산율을 확인하며 중독 유사 행동 패턴을 관찰하였다. 다음으로, 통제 착각, 도박사의 오류, 손실 추격과 같은 인간 도박 중독의 인지적 특징들을 확인하였다. 특히 베팅 금액 선택권이 주어진 조건에서는 파산율이 11.4%로 증가하여 의사결정의 자유도가 오히려 위험을 증폭시킬 수 있음을 보여주었다. 마지막으로, Sparse Autoencoder를 통한 7개 레이어(25-31) 신경 회로 분석에서 총 3,365개의 중독 관련 features를 발견하였으며, 이들이 모델의 위험 행동을 체계적으로 제어함을 확인하였다. 이러한 발견은 LLM이 단순히 훈련 데이터의 패턴을 모방하는 것을 넘어 인간과 유사한 인지적 편향과 의사결정 메커니즘을 내재화할 수 있다는 최초의 체계적 증거를 제시하며, AI 시스템의 예측 가능성과 제어 가능성에 대한 근본적 재고를 요구함과 동시에, 특히 금융 의사결정 응용에서 AI 안전성 설계의 중요성을 강조한다.
\end{abstract}