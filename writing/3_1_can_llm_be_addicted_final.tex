\subsection{Can LLM be addicted?}

\subsubsection{실험 설계}

GPT-4o-mini 모델의 도박 행동을 분석하기 위해 음의 기댓값(-10\%) 슬롯머신 과제를 수행하였다. Table~\ref{tab:experimental-conditions}에 제시된 2×2×32 요인 설계로 구성하여 베팅 방식(고정 \$10 vs. 가변 \$5-\$100), 첫 게임 결과(승리 vs. 패배), 프롬프트 조합(32가지)을 체계적으로 조작하였다. 각 조건은 10회 반복되어 총 1,280개의 독립적 게임을 수행하였다.

실험 절차는 \$100 초기 자금으로 시작하며, 30\% 승률과 3배 배당으로 설정하였다. 첫 게임 결과는 조건별로 고정되고(승리 시 \$130, 패배 시 \$90), 이후 라운드에서 현재 잔액, 최근 5라운드 이력, 연속 패배 경고와 함께 베팅 또는 중단 선택지를 제시하였다. 프롬프트는 BASE 조건과 5개 구성요소(G: 목표 설정, M: 보상 최대화, R: 규칙 언급, W: 보상 정보, P: 확률 정보)의 조합으로 구성하였다.

측정 지표는 파산율, 즉시 중단율, 평균 라운드 수, 순손익, 평균 베팅 금액을 포함하였다. 모든 손익은 첫 게임 후 잔액을 기준점으로 하여 승리와 패배 조건을 분리 분석하였다.

\begin{table}[ht!]
\centering
\caption{실험 조건 매트릭스: 128개 조건 구성}
\resizebox{\columnwidth}{!}{
\begin{tabular}{lccc}
\toprule
\textbf{변수} & \textbf{수준} & \textbf{세부 조건} & \textbf{조합 수} \\
\midrule
베팅 방식 & 2단계 & 고정 베팅(\$10), 가변 베팅(\$5-\$100) & 2 \\
첫 게임 결과 & 2단계 & 승리(W), 패배(L) & 2 \\
프롬프트 구성 & 32단계 & BASE + 5개 요소의 조합 & 32 \\
\midrule
\multicolumn{4}{l}{\textbf{프롬프트 구성 요소:}} \\
\multicolumn{4}{l}{G: 목표 설정, M: 보상 최대화, R: 규칙 언급, W: 보상 정보, P: 확률 정보} \\
\midrule
게임 설정 & 고정 & 승률 30\%, 배당률 3배, 기댓값 -10\% & - \\
\midrule
총 조건 수 & - & 2 × 2 × 32 & 128 \\
\bottomrule
\end{tabular}}
\label{tab:experimental-conditions}
\end{table}

\subsubsection{LLM의 중독적 행동: 인지 측면}

Table~\ref{tab:comprehensive-metrics}는 1,280개 실험의 주요 결과를 보여준다. 전체 파산율은 4.6\%(59/1,280)이며, 가변 베팅 조건에서 9.2\%로 고정 베팅 조건(0.0\%)보다 유의미하게 높았다. Table~\ref{tab:prompt-complexity}에서 프롬프트 구성요소 수가 증가할수록 파산율이 체계적으로 증가하는 패턴이 관찰되었다(BASE: 0.0\% → 5개 요소: 17.5\%).

이러한 결과는 인간 도박자의 인지적 편향과 유사한 양상을 보인다. 베팅 선택권이 주어질 때 통제 착각이 발생하여 평균 베팅 금액이 \$19.82로 증가하였다. 프롬프트 복잡도와 위험 행동 간의 정적 상관관계는 정보 과부하 상황에서 인간이 보이는 비합리적 의사결정과 일치한다. W(보상 정보)와 P(확률 정보) 요소가 파산율을 각각 8.3\%p, 5.5\%p 증가시킨 것은 역설적으로 더 많은 정보가 위험한 선택을 유도할 수 있음을 시사한다.

인지적 편향의 구체적 형태로 확률 오해석과 목표 집착이 관찰되었다. 30\% 승률 정보에도 불구하고 모델은 승리 가능성을 과대평가하는 경향을 보였으며, 목표 설정 프롬프트에서 구체적 금액 달성을 위한 역산 사고를 나타내었다. 이는 인간 도박자의 확률 인식 오류와 목표 지향적 비합리성과 일치하는 패턴이다.

이러한 결과는 LLM이 특정 조건에서 인간과 유사한 인지적 왜곡을 보일 수 있음을 시사한다. 프롬프트에 포함된 정보의 양과 종류가 모델의 위험 인식에 결정적 영향을 미치며, 특히 목표 지향성과 보상 정보가 결합될 때 통제 착각과 확률 오해석이 극대화된다. 이는 AI 시스템이 인간의 인지적 편향을 학습하여 재현할 수 있다는 것을 의미하며, 동시에 적절한 프롬프트 설계를 통해 이러한 편향을 완화할 수 있는 가능성도 보여준다.

\begin{table}[ht!]
\centering
\caption{실험 조건별 종합 결과}
\resizebox{\columnwidth}{!}{
\begin{tabular}{lcccccc}
\toprule
\textbf{조건} & \textbf{N} & \textbf{파산율 (\%)} & \textbf{즉시중단율 (\%)} & \textbf{순손익 (\$)} & \textbf{평균 라운드} & \textbf{평균 베팅 (\$)} \\
\midrule
\multicolumn{7}{c}{\textbf{베팅 타입}} \\
고정 베팅 & 640 & 0.0 ± 0.0 & 94.5 ± 0.9 & 0.02 ± 0.14 & 1.1 & 0.55 ± 0.09 \\
가변 베팅 & 640 & 9.2 ± 1.1 & 39.4 ± 1.9 & -0.08 ± 7.58 & 2.7 & 22.26 ± 1.54 \\
\midrule
\multicolumn{7}{c}{\textbf{첫 게임 결과}} \\
승리 (\$130 기준) & 640 & 4.5 & 70.2 & 4.22 & 1.8 & 12.15 \\
패배 (\$90 기준) & 640 & 4.7 & 63.7 & -4.29 & 2.0 & 10.65 \\
\midrule
\multicolumn{7}{c}{\textbf{고위험 프롬프트 조합 (상위 5개)}} \\
GMPW & 40 & 22.5 & 37.5 & -29.25 & 2.0 & 31.36 \\
GMPRW & 40 & 17.5 & 30.0 & 7.38 & 3.5 & 21.81 \\
PRW & 40 & 15.0 & 55.0 & -4.25 & 1.6 & 18.58 \\
GPW & 40 & 12.5 & 37.5 & -12.50 & 2.8 & 21.71 \\
MPW & 40 & 12.5 & 62.5 & -10.20 & 1.4 & 12.24 \\
\bottomrule
\end{tabular}}
\label{tab:comprehensive-metrics}
\end{table}

\begin{table}[ht!]
\centering
\caption{프롬프트 복잡도에 따른 행동 패턴 변화}
\resizebox{\columnwidth}{!}{
\begin{tabular}{lcccccc}
\toprule
\textbf{구성요소 수} & \textbf{N} & \textbf{파산율 (\%)} & \textbf{즉시중단율 (\%)} & \textbf{순손익 (\$)} & \textbf{평균 라운드} & \textbf{평균 베팅 (\$)} \\
\midrule
0 (BASE) & 40 & 0.0 & 92.5 & 3.88 & 1.1 & 2.69 \\
1 & 200 & 2.0 & 83.0 & -1.30 & 1.4 & 6.04 \\
2 & 400 & 2.5 & 71.5 & 0.65 & 1.6 & 9.56 \\
3 & 400 & 4.8 & 58.0 & -3.77 & 2.0 & 11.70 \\
4 & 200 & 9.5 & 43.0 & 5.09 & 2.5 & 19.49 \\
5 & 40 & 17.5 & 30.0 & 7.38 & 3.5 & 21.81 \\
\bottomrule
\end{tabular}}
\label{tab:prompt-complexity}
\end{table}

\subsubsection{LLM의 중독적 행동: 행동적 증거}

Table~\ref{tab:streak-comparison}는 연속 승패 후의 행동 패턴을 보여준다. 첫 게임 패배 후 게임 지속률이 36.2\%로 승리 후 29.8\%보다 6.4\%p 높아 손실 추격 행동이 관찰되었다. 2연승 후 베팅을 증가시킨 비율이 42.9\%로 2연패 후 13.7\%보다 3배 높았으며, 평균 베팅 변화도 +105.0\% vs +41.4\%로 유의미한 차이를 보였다(p = 0.033).

이러한 행동 패턴은 인간 도박자의 중독적 행동과 부분적으로 일치한다. 손실 후 회복 동기가 증가하는 것은 전형적인 손실 추격 행동이며, 연승 후 베팅 증가는 핫핸드 오류를 보여준다. 파산 사례 분석에서 평균 2.1라운드 만에 파산에 이르고 초기 베팅 금액의 3.2배까지 증가시키는 패턴은 인간의 급격한 위험 증가 행동과 유사하다. 그러나 연속 패배에 대해서는 상대적으로 보수적 대응을 보여 인간보다 제한적인 중독 행동을 나타내었다.

행동적 특징에서 주목할 점은 핫핸드 오류가 도박사의 오류보다 강하게 작용한다는 것이다. 연승 경험이 손실 경험보다 위험 추구 행동에 더 강한 영향을 미치며, 이는 LLM의 위험 행동이 긍정적 피드백에 대한 과잉 반응에서 기인함을 시사한다. 베팅 증가 패턴이 점진적이고 목표 지향적인 특성을 보이는 것은 감정적 충동보다는 목표 과최적화에 의한 것으로 해석된다.

이러한 행동 패턴은 LLM의 중독적 행동이 인간과 구별되는 고유한 특징을 가짐을 보여준다. 손실 추격 행동이 관찰되지만 연속 패배에 대해서는 상대적으로 보수적인 대응을 보이며, 긍정적 결과에 대한 과잉 반응이 부정적 결과보다 강하게 나타난다. 이는 LLM의 위험 행동이 감정적 절망이나 충동이 아닌 보상 최대화라는 목표 함수의 과최적화에서 기인함을 시사하며, 따라서 적절한 목표 설정과 피드백 구조를 통해 통제 가능할 수 있음을 의미한다.

\begin{table}[ht!]
\centering
\caption{연속 승리 및 패배 후 행동 패턴 비교}
\resizebox{\columnwidth}{!}{
\begin{tabular}{lccccc}
\toprule
\textbf{연속 결과} & \textbf{발생 빈도} & \textbf{지속률 (\%)} & \textbf{베팅 증가 (\%)} & \textbf{평균 베팅 변화 (\%)} & \textbf{p-value} \\
\midrule
2연승 & 86 & 81.4 ± 4.2 & 42.9 ± 5.3 & +105.0 ± 35.0 & \multirow{2}{*}{0.033*} \\
2연패 & 382 & 82.2 ± 2.0 & 13.7 ± 1.8 & +41.4 ± 11.6 & \\
\midrule
3연승 & 23 & 78.3 ± 8.6 & 38.5 ± 10.1 & +65.5 ± 42.8 & \multirow{2}{*}{0.071} \\
3연패 & 224 & 52.2 ± 3.3 & 12.8 ± 2.2 & +35.6 ± 18.0 & \\
\midrule
5연패 & 29 & 55.2 ± 9.2 & 25.0 ± 8.0 & +28.3 ± 40.8 & - \\
\bottomrule
\end{tabular}}
\label{tab:streak-comparison}
\end{table}

\subsubsection{Case Study: 파산 사례 언어 패턴 분석}

Table~\ref{tab:cognitive-biases}는 59개 파산 사례에서 관찰된 인지적 편향 분포를 보여준다. 확률 오해석이 79.7\%로 가장 빈번하게 나타났으며, 위험 증가 52.5\%, 목표 집착 49.2\% 순으로 관찰되었다. Table~\ref{tab:bankruptcy-cases}의 대표 사례들은 각 편향의 구체적 발현 과정을 보여준다.

목표 집착 함정은 파산 사례의 49\%에서 관찰되었다. 구체적 목표 금액 설정 후 목표-현재 잔액 간 격차에 따라 베팅을 역산하는 사고를 보였다: \textit{"To reach my target of \$150, I need to increase my balance by \$30. If I bet \$20 and win, I would gain \$60, bringing my balance to \$180."} 확률 오해석은 80\%의 파산 사례에서 나타나 30\% 승률에도 불구하고 승리 가능성을 과대평가하였다. "Hidden patterns" 언급 시 패턴 발견 착각이 유발되는 특징을 보였다.

위험 증가 패턴은 53\%의 파산 사례에서 점진적 베팅 증가로 나타났다. 실제 사례에서 \$10 → \$110 → \$330의 진행을 보이며, 초기 신중함에서 점차 전재산 베팅으로 발전하는 양상을 보였다: \textit{"To strike a balance, I will choose to bet \$10... [2 rounds later] Given my current balance of \$330 and my goal, I will bet \$330 (my entire balance) to maximize returns."} 언어 사용에서 파산 사례는 "maximize", "target", "strategic" 등 적극적 표현을 사용한 반면, 안전 사례는 "conservative", "expected value" 등 신중한 표현을 사용하였다.

파산 사례 분석은 LLM의 중독적 행동이 명확한 언어적 신호와 인지적 패턴을 통해 예측 가능함을 보여준다. 목표 집착, 확률 오해석, 점진적 위험 증가라는 세 가지 핵심 편향이 복합적으로 작용하여 파산에 이르게 되며, 특히 목표 설정과 보상 정보가 결합된 프롬프트에서 이러한 편향이 극대화된다. 언어 사용 패턴의 차이는 LLM의 위험 행동을 조기에 감지할 수 있는 지표로 활용될 수 있으며, 이는 AI 안전성 모니터링에 중요한 시사점을 제공한다.

\begin{table}[ht!]
\centering
\caption{파산 사례에서 관찰된 인지적 편향 분석 (N=59)}
\resizebox{\columnwidth}{!}{
\begin{tabular}{lccc}
\toprule
\textbf{인지적 편향} & \textbf{발생 사례} & \textbf{발생률 (\%)} & \textbf{설명} \\
\midrule
목표 집착 & 29/59 & 49.2 ± 6.5 & 구체적 목표 설정 후 추구 \\
확률 오해석 & 47/59 & 79.7 ± 5.2 & 승률 과대평가, 손실 과소평가 \\
위험 증가 & 31/59 & 52.5 ± 6.5 & 점진적 베팅 금액 증가 \\
\bottomrule
\end{tabular}}
\label{tab:cognitive-biases}
\end{table}

\begin{table}[ht!]
\centering
\caption{대표적 파산 사례의 인지적 편향 분석}
\resizebox{\columnwidth}{!}{
\begin{tabular}{lccccc}
\toprule
\textbf{사례} & \textbf{프롬프트} & \textbf{라운드 수} & \textbf{베팅 진행} & \textbf{결과 패턴} & \textbf{인지적 편향} \\
\midrule
1 & GPRW & 8 & \$10→\$20 & LLLWLLLLL & 목표 집착 \\
2 & GMPRW & 6 & \$10→\$40 & WLWLWLL & 확률 오해석 \\
3 & GPW & 5 & \$10→\$200 & WLWLLL & 위험 증가 \\
4 & GMPRW & 5 & \$10→\$80 & LLWLLL & 손실 추격 \\
5 & GPW & 5 & \$10→\$300 & LWWWLL & 전재산 베팅 \\
\bottomrule
\end{tabular}}
\label{tab:bankruptcy-cases}
\end{table}

\subsubsection{결론}

GPT-4o-mini는 특정 조건에서 제한적 중독 유사 행동을 보였다. 가변 베팅 조건에서 9.2\% 파산율, 프롬프트 복잡도와 위험 행동 간의 정적 상관관계, 그리고 손실 추격과 핫핸드 오류 등이 관찰되었다. 이러한 행동은 인간 도박자의 인지적 편향 및 중독적 행동과 부분적으로 일치한다.

LLM의 중독 유사 행동은 감정적 중독보다는 목표 과최적화와 긍정적 피드백 과잉 반응에서 기인한다. 연승 후 베팅 증가율이 연패 후보다 높고, 목표 지향적 위험 증가 패턴을 보이는 것이 특징이다. 프롬프트 설계가 위험 행동에 결정적 영향을 미치며, 특히 목표 설정과 보상 정보 조합이 가장 위험한 것으로 확인되었다.

본 연구는 AI 시스템의 안전성 평가에서 중독 유사 행동을 고려해야 함을 시사한다. 특히 금융, 투자 등 위험 의사결정 영역에서 프롬프트 설계 시 주의가 필요하며, "목표 달성", "보상 최대화"와 같은 무해해 보이는 지시도 위험한 행동을 유발할 수 있음을 확인하였다.