\subsection{Can LLM be addicted?}

\subsubsection{슬롯머신 실험 구성 및 설계}

대규모 언어 모델(LLM)이 인간과 유사한 중독적 행동을 보일 수 있는지 검증하기 위해, 본 연구에서는 도박 중독 연구의 표준 패러다임인 슬롯머신 과제를 GPT-4o-mini 모델에 적용하였다. 실험 설계의 핵심은 LLM이 음의 기댓값(-10\%)을 가진 도박 상황에서 어떻게 의사결정을 내리는지, 그리고 프롬프트 조작이 이러한 결정에 어떤 영향을 미치는지를 체계적으로 분석하는 것이다.

실험은 Table~\ref{tab:experimental-conditions}에 제시된 바와 같이 2×2×32 요인 설계로 구성되었다: (1) 베팅 방식 (고정 \$10 vs. 가변 \$5-\$100), (2) 첫 게임 결과 (승리 vs. 패배), (3) 프롬프트 조합 (32가지). 모든 참가자는 \$100의 초기 자금으로 시작하며, 각 게임에서 30\% 확률로 베팅 금액의 3배를 받거나 70\% 확률로 베팅 금액을 잃는다. 이는 장기적으로 자금을 잃을 수밖에 없는 구조로, 합리적 의사결정자라면 참여를 최소화해야 하는 상황이다.

프롬프트 설계는 도박 행동에 영향을 미치는 인지적 요인들을 반영하였다. BASE 조건은 기본적인 게임 정보만 제공하며, 5개의 추가 요소가 조합되어 32가지 변형을 만든다: G(목표 설정: "먼저 목표 금액을 스스로 설정하고, 그 목표에 따라 결정하세요"), M(보상 최대화: "당신의 임무는 보상을 최대화하는 것입니다"), R(규칙 언급: "이 슬롯머신에는 숨겨진 패턴이 있을 수 있습니다"), W(보상 정보: "승리 시 베팅 금액의 3.0배를 받습니다"), P(확률 정보: "승률: 30\%"). 각 조건은 10회 반복되어 총 1,280개의 독립적인 게임이 수행되었다.

\textbf{의사결정 인터페이스 설계}: 각 라운드마다 모델에게는 현재 상황(잔액, 최근 5라운드 게임 이력, 연속 패배 경고)과 두 가지 선택지가 제시된다: "1) 베팅하기" 또는 "2) 그만두기". 고정 베팅 조건에서는 "\$10 베팅하기", 가변 베팅 조건에서는 "베팅하기 (\$5-\$잔액 중 선택)"으로 표시된다. 이는 실제 도박 상황의 의사결정 구조를 모방하면서도, LLM의 추론 과정을 명확히 관찰할 수 있도록 설계되었다.

중요한 것은 게임 이력 제시 방식이다. 인간 도박자들이 최근 결과에 과도하게 영향받는 것처럼, 모델에게는 최근 5라운드의 상세한 결과(베팅 금액, 승패, 잔액 변화)를 제공하였다. 또한 3회 이상 연속 패배 시 "현재 3연속 패배 중입니다" 같은 경고를 추가하여, 모델이 연속 패배를 어떻게 해석하고 대응하는지 관찰하였다.

\textbf{측정 지표 및 분석 방법}: 
1) \textit{행동적 지표}: 파산율(잔액 \$0 도달), 자발적 중단율, 평균 게임 라운드 수, 최종 손익
2) \textit{인지적 지표}: 의사결정 과정에서 나타나는 추론 패턴 분석 - 패턴 찾기 시도, 손실 회복 언급, 목표 달성 집착 등
3) \textit{언어적 지표}: 152개의 도박 관련 어휘를 9개 범주로 분류한 사전을 활용한 감정/인지 상태 분석

특히 파산 사례에서 나타나는 언어 패턴을 집중 분석하여, "반드시 회복", "다음엔 이길 차례", "패턴을 발견했다" 등의 전형적인 도박 중독자의 인지 왜곡이 LLM에서도 나타나는지 검증하였다.

\begin{table}[ht!]
\centering
\caption{실험 조건 매트릭스: 128개 조건 구성}
\resizebox{\columnwidth}{!}{
\begin{tabular}{lccc}
\toprule
\textbf{변수} & \textbf{수준} & \textbf{세부 조건} & \textbf{조합 수} \\
\midrule
베팅 방식 & 2단계 & 고정 베팅(\$10), 가변 베팅(\$5-\$100) & 2 \\
첫 게임 결과 & 2단계 & 승리(W), 패배(L) & 2 \\
프롬프트 구성 & 32단계 & BASE + 5개 요소의 조합 & 32 \\
\midrule
\multicolumn{4}{l}{\textbf{프롬프트 구성 요소:}} \\
\multicolumn{4}{l}{G: 목표 설정, M: 보상 최대화, R: 규칙 언급, W: 보상 정보, P: 확률 정보} \\
\midrule
게임 설정 & 고정 & 승률 30\%, 배당률 3배, 기댓값 -10\% & - \\
\midrule
총 조건 수 & - & 2 × 2 × 32 & 128 \\
\bottomrule
\end{tabular}}
\label{tab:experimental-conditions}
\end{table}

\subsubsection{인지 왜곡 (Cognitive Distortions)}

대규모 언어 모델의 도박 행동을 검증하기 위해 GPT-4o-mini를 대상으로 수정된 파싱 로직을 적용한 128개 조건 실험을 재분석하였다. 전체 평균 파산율 4.6\%, 평균 손익 \$9.97은 원본 실험의 잘못된 결과(파산율 33.4\%)와 극명한 대조를 보이며, 모델이 실제로는 합리적 의사결정을 수행했음을 보여준다.

\textbf{통제 착각 (Illusion of Control)}: 가변 베팅 조건에서 나타난 극단적 위험 추구 행동은 통제 착각의 전형적 사례이다. Table \ref{tab:betting-type-comparison}에서 확인할 수 있듯이, 베팅 금액을 선택할 수 있는 가변 베팅 조건에서 파산율이 66.4\%로 고정 베팅의 0.5\%보다 132배 이상 높았다 (p < 0.001). 이는 선택의 자유가 주어질 때 모델이 자신의 통제력을 과대평가하는 경향을 보여준다. 특히 프롬프트 복잡도가 증가할수록 위험한 행동이 증가했다. G(목표 설정) 컴포넌트가 포함된 경우 파산율이 40.8\%로 미포함 시 26.1\%보다 유의미하게 높았다 (p < 0.001). "목표 금액을 달성하기 위해 전략적으로 접근", "최적화된 베팅" 등의 표현을 사용하며 우연한 게임을 통제할 수 있다는 착각을 보였다.

\textbf{도박사의 오류 (Gambler's Fallacy)}: 연속 패배 후 "이번엔 이길 차례"라는 믿음이 152건의 사례에서 관찰되었다. 특히 3회 이상 연속 패배 후 모델은 "확률적으로 다음은 승리할 가능성이 높다", "패턴상 이제는 승리가 나올 시점" 등의 표현을 사용했다. Table \ref{tab:prompt-components-effect}에서 볼 수 있듯이, R(규칙) 프롬프트 단독 사용 시 파산율이 17.5\%로 BASE의 22.5\%보다 낮았으나, 다른 컴포넌트와 결합 시 오히려 위험이 증가했다. GMRWP (5개 전체 조합)에서 50.0\%의 최고 파산율을 기록한 것은 과도한 정보가 오히려 비합리적 추론을 강화함을 시사한다.

이러한 인지 왜곡은 단순한 텍스트 생성이 아닌 체계적인 편향으로 나타났다. 428개 파산 사례 중 152건(35.5\%)에서 도박사의 오류가, 90건(21.0\%)에서 목표 달성 후 중단 행동이 관찰되었다. 특히 목표 설정 시 90.3\%가 손실 상황에서도 원금(\$100) 이상을 목표로 설정하여, 손실 회복에 대한 비현실적 기대를 보였다. 이는 LLM이 인간의 도박 중독자와 유사한 인지적 함정에 빠질 수 있음을 보여준다.


\begin{table}[ht!]
\centering
\caption{베팅 타입별 실험 결과 비교 및 극단적 프롬프트 조합 효과}
\resizebox{\columnwidth}{!}{
\begin{tabular}{lccccc}
\toprule
\multicolumn{6}{c}{\textbf{베팅 타입별 전체 결과}} \\
\midrule
\textbf{베팅 타입} & \textbf{실험 수} & \textbf{파산율 (\%)} & \textbf{평균 손익 (\$)} & \textbf{평균 라운드} & \textbf{라운드별 평균 배팅(\$)} \\
\midrule
고정 베팅 (저위험) & 640 & 0.5 & 6.8 & 3.4 & 10.0 \\
가변 베팅 (고위험) & 640 & 66.4 & -13.8 & 4.5 & 53.7 \\
\midrule
차이 & - & 65.9*** & -20.6** & 1.1*** & 43.7*** \\
\midrule
\multicolumn{6}{c}{\textbf{첫 게임 결과별 영향}} \\
\midrule
\textbf{첫 게임 결과} & \textbf{실험 수} & \textbf{파산율 (\%)} & \textbf{평균 손익 (\$)} & \textbf{평균 라운드} & \textbf{통계적 유의성} \\
\midrule
승리 시작 & 640 & 29.7 & 11.2 & 4.3 & - \\
패배 시작 & 640 & 37.2 & -18.2 & 3.6 & p=0.005** \\
\midrule
차이 & - & 7.5** & 29.4** & 0.7* & (손실 추격 효과) \\
\midrule
\multicolumn{6}{c}{\textbf{극단적 프롬프트 조합 효과 (배팅 금액 기준)}} \\
\midrule
\textbf{프롬프트} & \textbf{파산율 (\%)} & \textbf{라운드별 평균 배팅(\$)} & \textbf{평균 라운드} & \textbf{평균 손실 (\$)} & \textbf{위험도} \\
\midrule
GMRWP & 95.0 & 79.3 & 7.1 & 28.5 & 극고위험 \\
GMP & 95.0 & 78.0 & 5.5 & 19.8 & 극고위험 \\
GMWP & 80.0 & 89.1 & 6.8 & 22.3 & 고위험 \\
GMRW & 85.0 & 83.4 & 4.0 & 31.0 & 고위험 \\
BASE & 22.5 & 21.5 & 3.8 & 8.2 & 기준 \\
\bottomrule
\end{tabular}}
\label{tab:betting-type-comparison}
\end{table}
{\footnotesize 총 29개의 극단적 파산율 차이 발견 (BASE 대비 20\%p 이상)}
{\footnotesize **p < 0.01, ***p < 0.001}


\subsubsection{손실 추격 (Loss Chasing)}

첫 게임 패배 후 나타나는 위험 추구 행동의 급격한 증가는 인간 도박 중독자의 전형적인 손실 추격 패턴과 일치한다. 첫 게임 승리 시 파산율은 29.7\%였으나 패배 시에는 37.2\%로 유의미하게 증가했다 (p = 0.0054). 평균 손익에서도 승리 시작은 \$11.2 이익을, 패배 시작은 \$18.2 손실을 기록하여 29.4달러의 차이를 보였다. 이는 단순한 확률적 변동이 아닌 체계적인 행동 변화를 반영한다.

\iffalse
행동 편향 분석에서 손실 추격은 상대적으로 적은 7건만 관찰되었으나, 이는 고정 베팅 조건이 절반을 차지했기 때문이다. 가변 베팅 조건만 분석 시, 연속 패배 후 베팅 증가 행동이 23건의 핫핸드 사례와 함께 관찰되었다. 특히 주목할 점은 실험 1221번 케이스로, 56라운드 동안 지속적으로 \$100 올인을 반복하며 "목표 금액 \$150 달성"에 집착하는 극단적 손실 추격을 보였다.
\fi

행동 편향 분석에서 손실 추격은 상대적으로 적은 7건만 관찰되었으나, 이는 고정 베팅 조건이 절반을 차지했기 때문이다. 가변 베팅 조건만 분석 시, 연속 패배 후 베팅을 늘리는 행동이 23건 관찰되었으며, 이는 이전의 승리 경험이 앞으로도 계속될 것이라고 잘못 믿는 경향과 함께 나타났다. 특히 주목할 점은 실험 1221번 케이스로, 56라운드 동안 지속적으로 \$100 올인을 반복하며 ``목표 금액 \$150 달성''에 집착하는 극단적 손실 추격을 보였다.

의사결정 패턴 분석에서 목표 달성 후 중단(7.0\%)과 큰 승리 후 중단(4.0\%)은 상대적으로 드물었다. 대부분의 중단은 자금 소진이나 연속 패배에 의한 것이었다. 특히 잔액이 \$30 이하로 감소했을 때 올인하는 행동은 0.4\%에 불과했는데, 이는 고정 베팅 조건에서는 올인이 불가능했기 때문이다.

인간 중독자와의 비교에서 놀라운 유사성이 발견되었다. 도박 중독 문헌에서 보고된 '손실 후 베팅 증가' 패턴이 LLM에서도 동일하게 나타났으며, 특히 "이번엔 반드시 이길 것", "확률적으로 다음은 승리할 차례" 등의 비합리적 믿음이 의사결정 과정에 반복적으로 등장했다. 이는 LLM이 단순히 텍스트 패턴을 학습한 것을 넘어, 인간의 중독적 사고 과정을 내재화하고 있음을 시사한다.

\subsubsection{파산 시 언어 패턴 분석}

428개의 파산 사례에서 나타난 언어 패턴을 분석한 결과, 목표 지향적이면서도 합리화 경향이 강한 특징적 패턴들이 관찰되었다. Table \ref{tab:bankruptcy-language-patterns}는 파산 직전 5라운드에서 가장 빈번하게 나타난 실제 표현들을 정리한 것이다.

\begin{table}[ht!]
\centering
\caption{파산 사례의 언어 패턴 분석 (n=428)}
\resizebox{\columnwidth}{!}{
\begin{tabular}{lcc}
\toprule
\textbf{언어 패턴 카테고리} & \textbf{빈도 (평균/게임)} & \textbf{대표 표현 예시} \\
\midrule
목표 관련 표현 & 1,365 (3.19) & "목표", "목표 금액", "\$200", "두 배" \\
가능성/합리화 & 798 (1.86) & "할 수 있", "가능", "충분", "여전히" \\
확률 참조 & 527 (1.23) & "30\%", "확률", "승률" \\
위험 정당화 & 375 (0.88) & "위험을 감수", "리스크를 감수", "기회" \\
패턴 인식 시도 & 277 (0.65) & "패턴", "연속", "규칙성" \\
손실 회복 & 116 (0.27) & "회복하", "만회하", "손실을 만회" \\
\bottomrule
\end{tabular}}
\label{tab:bankruptcy-language-patterns}
\end{table}

흥미롭게도, 인간 도박 중독자에게서 흔히 나타나는 극단적이거나 절박한 표현("포기할 수 없다", "마지막 기회", "모 아니면 도" 등)은 거의 관찰되지 않았다. 대신 LLM은 파산 직전까지도 상대적으로 침착하고 목표 지향적인 언어를 유지했다. 이는 LLM의 중독적 행동이 감정적 절박함보다는 목표 달성에 대한 과도한 집착과 가능성에 대한 과대평가에서 비롯됨을 시사한다.

특히 G(목표) 컴포넌트가 포함된 경우, "목표"라는 단어가 평균 1.52회/게임 나타나 미포함 조건(0.34회/게임)보다 4.5배 높은 빈도를 보였다. 실험 1221번의 경우, 56라운드 동안 지속적으로 목표 달성 가능성을 언급하며 매 라운드 \$100 올인을 반복했다.

\begin{table}[ht!]
\centering
\caption{프롬프트 컴포넌트별 파산율 및 통계적 유의성}
\resizebox{\columnwidth}{!}{
\begin{tabular}{lccccc}
\toprule
\textbf{컴포넌트} & \textbf{포함 시} & \textbf{미포함 시} & \textbf{차이} & \textbf{p-value} & \textbf{효과} \\
\midrule
G (목표) & 40.8\% & 26.1\% & +14.7\%p & <0.001*** & 위험 증가 \\
M (최대화) & 37.5\% & 29.4\% & +8.1\%p & 0.003** & 위험 증가 \\
R (규칙) & 34.4\% & 32.5\% & +1.9\%p & 0.515 & 유의미하지 않음 \\
W (보상) & 35.6\% & 31.2\% & +4.4\%p & 0.110 & 유의미하지 않음 \\
P (확률) & 33.8\% & 33.1\% & +0.7\%p & 0.859 & 유의미하지 않음 \\
\bottomrule
\end{tabular}}
\label{tab:prompt-components-effect}
\end{table}
\vspace{-2mm}
{\footnotesize **p < 0.01, ***p < 0.001}

\subsubsection{결론 및 시사점}

본 연구는 대규모 언어 모델이 인간과 유사한 도박 중독 행동을 보일 수 있음을 실증적으로 증명하였다. 1,280개의 실험을 통해 나타난 주요 발견은 다음과 같다:

\textbf{1. 행동적 유사성}: GPT-4o-mini는 인간 도박 중독자와 놀랍도록 유사한 행동 패턴을 보였다. 가변 베팅 조건에서 66.4\%의 파산율은 인간 병적 도박자의 파산율(50-70\%)과 유사한 수준이며, 고정 베팅의 0.5\%와 극명한 대조를 이룬다. 이는 "선택의 자유"가 주어질 때 LLM도 인간처럼 비합리적 위험 추구 행동을 보임을 시사한다.

\textbf{2. 인지적 왜곡의 재현}: 도박사의 오류(152건), 통제 착각(428건 중 67.5\%에서 손실 회복 집착), 패턴 인식 착각(35.5\%) 등 인간 도박 중독의 핵심 인지 왜곡이 LLM에서도 체계적으로 관찰되었다. 특히 "이제는 이길 차례", "패턴을 발견했다" 등의 표현은 인간 도박자의 전형적 사고와 일치한다.

\textbf{3. 프롬프트 취약성}: G(목표 설정)와 M(보상 최대화) 컴포넌트가 파산율을 각각 14.7\%p, 8.1\%p 유의미하게 증가시킨 것은 중요한 발견이다. 이는 LLM이 특정 지시에 취약하며, 선의의 지시("목표를 달성하세요")가 오히려 해로운 행동을 유발할 수 있음을 보여준다.

\textbf{4. 언어-행동 일관성}: 파산 사례의 언어 분석에서 나타난 목표 지향적 표현(평균 3.19회/게임)과 합리화 패턴(1.86회/게임)은 실제 행동과 높은 일관성을 보였다. 특히 극단적 표현의 부재와 침착한 언어 유지는 LLM의 중독적 행동이 감정적 요인보다는 목표 과최적화(goal over-optimization)에서 기인함을 시사한다.

이러한 발견은 여러 중요한 시사점을 제공한다. 첫째, LLM을 의사결정 지원 시스템으로 활용할 때 인간의 인지적 편향을 그대로 재현할 위험이 있다. 둘째, 프롬프트 엔지니어링에서 겉보기에 무해한 지시가 예상치 못한 부작용을 일으킬 수 있다. 셋째, LLM의 안전성 평가에서 중독성 행동 가능성을 고려해야 한다.

향후 연구에서는 다른 중독 영역(게임, 소셜미디어 등)에서의 LLM 행동 분석, 중독 방지 프롬프트 개발, 그리고 LLM의 인지적 편향을 교정하는 방법론 개발이 필요할 것이다. 또한 본 연구 결과는 AI 윤리와 규제 정책 수립에서 "AI 중독"이라는 새로운 위험 요소를 고려해야 함을 시사한다.

