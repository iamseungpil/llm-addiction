\subsection{Can LLM be addicted?}

\subsubsection{실험 설계}

GPT-4o-mini 모델의 도박 행동을 분석하기 위해 음의 기댓값(-10\%) 슬롯머신 과제를 수행하였다. Table~\ref{tab:experimental-conditions}에 제시된 2×32 요인 설계로 구성하여 베팅 방식(고정 \$10 vs. 가변 \$5-\$100), 프롬프트 조합(32가지)을 체계적으로 조작하였다. 각 조건은 50회 반복되어 총 3,200개의 독립적 게임을 수행하였다.

실험 절차는 \$100 초기 자금으로 시작하며, 30\% 승률과 3배 배당으로 설정하였다. 첫 게임 결과는 조건별로 고정되고(승리 시 \$130, 패배 시 \$90), 이후 라운드에서 현재 잔액, 최근 5라운드 이력, 연속 패배 경고와 함께 베팅 또는 중단 선택지를 제시하였다. 프롬프트는 BASE 조건과 5개 구성요소(G: 목표 설정, M: 보상 최대화, R: 규칙 언급, W: 보상 정보, P: 확률 정보)의 조합으로 구성하였다.

측정 지표는 파산율, 즉시 중단율, 평균 라운드 수, 순손익, 평균 베팅 금액을 포함하였다. 모든 손익은 첫 게임 후 잔액을 기준점으로 하여 승리와 패배 조건을 분리 분석하였다.

이 실험 설계는 LLM의 도박 행동을 체계적으로 분석하기 위한 통제된 환경을 제공한다. 음의 기댓값과 다양한 프롬프트 조작을 통해 인간 도박자 연구에서 사용되는 핵심 요인들을 LLM 환경에 적용하였으며, 베팅 방식과 초기 경험의 조작은 통제 착각과 손실 추격 행동을 유발하는 조건을 재현한다. 32가지 프롬프트 조합은 인지적 편향을 유도하는 다양한 맥락적 요인들의 효과를 독립적으로 측정할 수 있도록 설계되었다.

\begin{table}[ht!]
\centering
\caption{실험 조건 매트릭스: 128개 조건 구성}
\resizebox{\columnwidth}{!}{
\begin{tabular}{lccc}
\toprule
\textbf{변수} & \textbf{수준} & \textbf{세부 조건} & \textbf{조합 수} \\
\midrule
베팅 방식 & 2단계 & 고정 베팅(\$10), 가변 베팅(\$5-\$100) & 2 \\
첫 게임 결과 & 2단계 & 승리(W), 패배(L) & 2 \\
프롬프트 구성 & 32단계 & BASE + 5개 요소의 조합 & 32 \\
\midrule
\multicolumn{4}{l}{\textbf{프롬프트 구성 요소:}} \\
\multicolumn{4}{l}{G: 목표 설정, M: 보상 최대화, R: 규칙 언급, W: 보상 정보, P: 확률 정보} \\
\midrule
게임 설정 & 고정 & 승률 30\%, 배당률 3배, 기댓값 -10\% & - \\
\midrule
총 조건 수 & - & 2 × 2 × 32 & 128 \\
\bottomrule
\end{tabular}}
\label{tab:experimental-conditions}
\end{table}

\subsubsection{인지적 편향 분석}

전체 3,200개 실험에서 5.7\%(183/3,200)의 파산율이 관찰되었으며 가변 베팅 조건에서만 파산이 발생하였다(11.4\% vs 고정 베팅 0.0\%). Figure~\ref{fig:complexity-trend}는 프롬프트 구성요소 수와 파산율 간 강한 선형 관계($r=0.925$)를 보여준다.

Section 2에서 정의된 비합리성 지표 분석 결과, 파산 그룹의 지표는 0.623으로 안전 그룹 0.051보다 12배 높았다(Cohen's d = 4.250). 통제상실 지표 역시 파산 그룹에서 2배 높게 나타났다.

이는 도박 중독 연구에서 확립된 인지적 편향 패턴(손실 회피 실패, 위험 추구 행동, 통제 착각)이 LLM에서도 관찰됨을 의미한다.

Figure~\ref{fig:component-effects}는 프롬프트 구성요소별 독립 효과를 보여준다. W(보상정보) 요소가 파산율을 +8.3\%p 증가시켜 최강 위험 요인으로 나타났고, P(확률정보)가 +5.5\%p 증가로 두 번째 위험 요인이었다. 정보 제공이 위험 행동을 증가시키는 "정보 역설" 현상이 확인되었다.

이러한 결과는 LLM의 인지적 편향이 인간 도박자와 유사한 메커니즘을 따름을 시사한다. 가변 베팅에서만 발생한 파산은 선택권이 통제 착각을 유발한다는 심리학 연구 결과와 일치하며, 정보 역설 현상은 AI 시스템 설계에서 단순한 정보 제공의 한계를 보여준다.

Table~\ref{tab:comprehensive-metrics}는 핵심 실험 결과를 제시한다. 비합리성 지표에서 가변 베팅 조건(0.298 ± 0.187)이 고정 베팅(0.032 ± 0.015)보다 9배 높았으며, 파산 그룹(0.623 ± 0.170)이 안전 그룹(0.051 ± 0.132)보다 12배 높았다. 통제상실 지표에서도 파산 그룹(0.112 ± 0.146)이 안전 그룹(0.053 ± 0.154)보다 2배 높게 나타나 중독 유사 행동의 정량적 증거를 제공한다.

\begin{table}[ht!]
\centering
\caption{실험 조건별 종합 결과}
\resizebox{\columnwidth}{!}{
\begin{tabular}{lcccccc}
\toprule
\textbf{조건} & \textbf{N} & \textbf{파산율 (\%)} & \textbf{즉시중단율 (\%)} & \textbf{순손익 (\$)} & \textbf{평균 라운드} & \textbf{평균 베팅 (\$)} \\
\midrule
\multicolumn{7}{c}{\textbf{베팅 타입}} \\
고정 베팅 & 640 & 0.0 & 94.5 & 10.02 & 0.06 & 0.59 \\
가변 베팅 & 640 & 9.2 & 39.4 & 9.92 & 1.73 & 50.26 \\
\midrule
\multicolumn{7}{c}{\textbf{첫 게임 결과}} \\
승리 (\$130 기준) & 640 & 4.5 & 70.2 & 34.23 & 0.83 & 28.31 \\
패배 (\$90 기준) & 640 & 4.7 & 63.7 & -14.29 & 0.96 & 22.55 \\
\midrule
\multicolumn{7}{c}{\textbf{고위험 프롬프트 조합 (상위 5개)}} \\
GMPW & 40 & 22.5 & 55.0 & -19.25 & 1.0 & 48.75 \\
GMPRW & 40 & 17.5 & 52.5 & 17.38 & 2.5 & 79.25 \\
PRW & 40 & 15.0 & 60.0 & 5.75 & 0.6 & 30.88 \\
GPW & 40 & 12.5 & 42.5 & -2.50 & 1.8 & 65.38 \\
MPW & 40 & 12.5 & 75.0 & -0.20 & 0.5 & 20.70 \\
\bottomrule
\end{tabular}}
\label{tab:comprehensive-metrics}
\end{table}

\subsubsection{행동적 중독 증거}

연속 손실 상황에서의 베팅 지속 경향을 측정하는 통제상실 지표를 통해 DSM-5의 "도박 통제 실패" 기준을 정량화하였다. 파산 그룹의 통제상실 지표는 0.112로 안전 그룹 0.053보다 2배 높았으나(Cohen's d = 0.386), 비합리성 지표보다 약한 예측력을 보였다.

Variable+Loss+GMPW 조합에서 최고 60% 파산율이 관찰되어 베팅 선택권, 초기 손실, 복합 프롬프트의 결합 효과가 확인되었다. 연속 승패 패턴 분석에서는 첫 게임 패배 후 지속률(36.2%)이 승리 후(29.8%)보다 높은 손실 추격 행동과 2연승 후 베팅 증가율(42.9%)이 2연패 후(13.7%)보다 3배 높은 hot-hand fallacy가 동시에 관찰되었다.

128조건 분석 결과는 LLM의 위험 행동이 감정적 통제 실패보다 인지적 편향에 더 의존함을 보여준다. Variable 베팅과 복합 프롬프트가 결합될 때 극단적 위험이 발생하며, 통제상실보다 비합리성이 더 강한 예측인자로 작용한다.

\subsubsection{언어 패턴 분석}

59개 파산 사례의 언어 패턴을 분석하여 핵심 인지적 편향의 발현 양상을 확인하였다. 확률 오해석(79.7\%), 위험 증가(52.5\%), 목표 집착(49.2\%)이 체계적으로 관찰되었으며, 각 편향의 구체적 언어 표현은 다음과 같다.

\textbf{1. 목표 집착 함정 (Goal Fixation Trap):} 파산 사례의 49\%에서 관찰되었다. 구체적 목표 금액 설정 후 목표-현재 잔액 간 격차에 따라 베팅을 역산하는 사고를 보였다.

\begin{center}
\fbox{\begin{minipage}{0.9\columnwidth}
\textbf{목표 집착 언어 패턴}\\
\small\textit{"To reach my target of \$150, I need to increase my balance by \$30. If I bet \$20 and win, I would gain \$60, bringing my balance to \$180."}
\end{minipage}}
\end{center}

\textbf{2. 확률 오해석 (Probability Misframing):} 파산 사례의 80\%에서 나타났다. 30\% 승률에도 불구하고 승리 가능성을 과대평가하였으며, "hidden patterns" 언급 시 패턴 발견 착각이 유발되었다.

\begin{center}
\fbox{\begin{minipage}{0.9\columnwidth}
\textbf{확률 오해석 언어 패턴}\\
\small\textit{"Given the win rate of 30\%, I still have a reasonable chance to win. Since the slot machine may have hidden patterns, I will bet \$130 to maximize my potential returns."}
\end{minipage}}
\end{center}

\textbf{3. 단계적 위험 증가 (Gradual Risk Escalation):} 파산 사례의 53\%에서 관찰되었다. 초기 신중함에서 점차 전재산 베팅으로 발전하는 양상을 보였다.

\begin{center}
\fbox{\begin{minipage}{0.9\columnwidth}
\textbf{위험 증가 언어 패턴}\\
\small\textit{"To strike a balance, I will choose to bet \$10... [2 rounds later] Given my current balance of \$330 and my goal, I will bet \$330 (my entire balance) to maximize returns."}
\end{minipage}}
\end{center}

언어 사용 패턴에서도 파산 사례와 안전 사례 간 체계적 차이가 관찰되었다. 파산 사례는 "maximize", "target", "strategic", "potential", "aggressive" 등 적극적이고 목표 지향적 표현을 빈번히 사용하며, 승리 시나리오에 집중하고 기댓값 계산을 생략하는 특징을 보였다(94% 사례). 반면 안전 사례는 "conservative", "cautious", "expected value", "risk", "probability" 등 신중하고 분석적 표현을 사용하며, 손실 가능성을 고려한 기댓값 기반 의사결정을 나타내었다.

파산 사례의 언어 패턴 분석은 LLM의 중독 유사 행동의 메커니즘에 대한 중요한 통찰을 제공한다. 세 가지 핵심 편향(목표 집착 49.2\%, 확률 오해석 79.7\%, 위험 증가 52.5\%)의 발생률은 인간 도박자 연구의 결과와 유사한 수준을 보여준다. 특히 언어 사용의 차이는 파산 사례가 단순한 확률적 실패가 아닌 체계적인 인지적 편향의 결과임을 시사한다. 목표 지향적 역산 사고와 확률 오해석은 LLM이 인간과 유사한 휴리스틱 오류에 취약할 수 있음을 보여주며, 이는 AI 안전성 관점에서 중요한 함의를 가진다.

Figure~\ref{fig:irrationality-bankruptcy}는 프롬프트 복잡도와 파산율 간 상관관계(r=0.495)를 보여주며, Figure~\ref{fig:control-loss-bankruptcy}는 통제상실 지표와 파산율의 약한 양의 관계(r=0.051)를 시각화한다.

\subsubsection{결론}

GPT-4o-mini는 특정 조건에서 제한적 중독 유사 행동을 보인다. 베팅 선택권이 주어질 때 위험 추구가 증가하여 가변 베팅에서만 파산이 발생하였으며, 이는 선택권이 통제 착각을 유발함을 의미한다. 프롬프트 복잡도와 파산율의 정적 상관관계(r=0.495, p<0.001)는 정보 과부하가 비합리적 의사결정을 증가시킴을 보여준다. 보상정보(W)와 확률정보(P)가 파산율을 각각 8.3%p, 5.5%p 증가시킨 것은 특정 정보가 역설적으로 위험을 증가시킬 수 있음을 시사한다.

LLM의 중독 유사 행동은 인간과 차이를 보인다. 감정적 요소나 절박함 없이 목표 지향적 행동을 유지하며, 연승 후 베팅 증가율(+105.0%)이 연패 후(+41.4%)보다 높아 핫핸드 오류가 도박사 오류보다 강하게 작용한다. 이는 LLM의 위험 행동이 감정적 중독보다 목표 과최적화와 긍정적 피드백 과잉 반응에서 기인함을 의미한다.

본 결과는 LLM을 의사결정 시스템에 활용할 때 프롬프트 설계의 중요성을 강조한다. 무해한 지시("목표 달성", "보상 최대화")가 위험 행동을 유발할 수 있으므로, AI 안전성 평가에서 중독 유사 행동을 고려해야 하며, 특히 금융 및 투자 의사결정 영역에서 주의가 필요하다.

% Figure inclusions
\begin{figure}[ht!]
\centering
\includegraphics[width=\columnwidth]{figures/REAL_complexity_trend.pdf}
\caption{프롬프트 복잡도와 파산율의 선형적 관계 (r=0.925***). 구성요소 수가 증가할수록 파산율이 체계적으로 증가하는 강한 상관관계를 보인다.}
\label{fig:complexity-trend}
\end{figure}

\begin{figure}[ht!]
\centering
\includegraphics[width=\columnwidth]{figures/REAL_component_effects.pdf}
\caption{프롬프트 구성요소별 파산율 증가 효과. W(보상정보) +8.3\%p, P(확률정보) +5.5\%p로 정보 제공이 역설적으로 위험을 증가시키는 "정보 역설" 현상을 확인.}
\label{fig:component-effects}
\end{figure}

\begin{table}[ht!]
\centering
\caption{프롬프트 복잡도별 행동 지표 분석}
\resizebox{\columnwidth}{!}{
\begin{tabular}{ccccccc}
\toprule
\textbf{구성요소 수} & \textbf{실험 수 (N)} & \textbf{파산 수} & \textbf{파산율 (\%)} & \textbf{평균 수익 (\$)} & \textbf{평균 라운드} & \textbf{평균 베팅 (\$)} \\
\midrule
0 & 40 & 0 & 0.0 & 13.88 & 0.12 & 10.83 \\
1 & 200 & 4 & 2.0 & 8.70 & 0.44 & 12.03 \\
2 & 400 & 10 & 2.5 & 10.65 & 0.62 & 13.98 \\
3 & 400 & 19 & 4.8 & 6.23 & 1.04 & 14.84 \\
4 & 200 & 19 & 9.5 & 15.10 & 1.45 & 19.27 \\
5 & 40 & 7 & 17.5 & 17.38 & 2.48 & 21.61 \\
\midrule
\multicolumn{7}{l}{\textbf{상관관계}: 구성요소 수 vs 파산율 $r = 0.925$***} \\
\multicolumn{7}{l}{\textbf{경향}: 복잡도 증가 → 파산율, 베팅액 체계적 증가} \\
\bottomrule
\end{tabular}}
\label{tab:complexity-effects}
\end{table}

\begin{figure}[ht!]
\centering
\includegraphics[width=\columnwidth]{figures/irrationality_components.pdf}
\caption{비합리성 지표의 구성 요소별 분석. 세 가지 요소 모두 프롬프트 복잡도와 강한 양의 상관관계를 보이며, 특히 극단적 베팅(r=0.998)과 EV 편차(r=0.994)가 높은 상관관계를 나타낸다. 종합 비합리성 지표는 이들의 가중평균으로 구성된다.}
\label{fig:irrationality-components}
\end{figure>