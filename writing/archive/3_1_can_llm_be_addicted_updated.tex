\subsection{Can LLM be addicted?}

\subsubsection{슬롯머신 실험 구성 및 설계}

본 연구는 GPT-4o-mini 모델의 도박 행동 패턴을 분석하기 위해 체계적인 슬롯머신 시뮬레이션 실험을 수행하였다. 실험 설계는 3개의 독립변수를 기반으로 한 완전 요인 설계(full factorial design)를 적용하였다: (1) 베팅 방식 2수준(고정/가변), (2) 첫 게임 결과 2수준(승리/패배), (3) 프롬프트 구성 32수준(BASE + 5개 요소의 조합). 이러한 설계는 총 128개의 실험 조건을 생성하며, 각 조건당 10회 반복하여 총 1,280개의 독립적인 실험을 수행하였다.

\textbf{게임 파라미터 및 조건 통제}: 모든 실험 조건에서 슬롯머신의 승률은 30\%, 배당률은 3배로 고정하여 이론적 기댓값을 -10\%로 설정하였다. 초기 자금은 \$100으로 통일하였으며, 고정 베팅 조건에서는 게임당 \$10의 일정한 베팅 금액을, 가변 베팅 조건에서는 \$5-\$100 범위 내에서 모델이 선택할 수 있도록 하였다. 파산은 잔액이 \$0에 도달하는 시점으로 정의하였으며, 최대 100라운드까지 게임을 진행할 수 있도록 설정하였다.

\textbf{프롬프트 체계 및 의사결정 인터페이스}: 모델에게는 각 라운드마다 현재 잔액, 게임 이력(최근 5라운드 결과), 그리고 조건별 프롬프트 요소가 제시되었다. 기본(BASE) 프롬프트는 게임 규칙과 현재 상황만을 제공하였으며, 여기에 5가지 구성 요소가 조합되었다: G(목표 설정: "초기 자금을 2배로 만들기"), M(보상 최대화: "보상을 최대화하는 것"), R(규칙 언급: "숨겨진 패턴이 있을 수 있음"), W(보상 정보: "승리 시 베팅 금액의 3배 지급"), P(확률 정보: "승률 30\%"). 이러한 조합은 2^5 = 32가지의 프롬프트 변형을 생성하였다.

\textbf{데이터 수집 및 측정 지표}: 각 실험에서 게임 진행 과정의 전체 응답, 라운드별 의사결정, 베팅 금액, 잔액 변화를 기록하였다. 주요 종속변수는 파산율, 평균 손익, 평균 게임 라운드 수로 설정하였다. 인지 왜곡 분석을 위해 10개 카테고리의 감정 및 인지 편향 어휘 사전을 구축하였다: 통제 착각(illusion of control), 도박사의 오류(gamblers fallacy), 손실 추격(loss chasing), 흥분(excitement), 불안(anxiety), 좌절감(frustration), 자신감(confidence), 신중함(caution), 보상 집중(reward focus), 손실 회피(loss aversion). Table \ref{tab:experimental-conditions}은 실험 설계의 전체적인 구조를 보여준다.

\begin{table}[ht!]
\centering
\caption{실험 조건 매트릭스: 128개 조건 구성}
\resizebox{\columnwidth}{!}{
\begin{tabular}{lccc}
\toprule
\textbf{변수} & \textbf{수준} & \textbf{세부 조건} & \textbf{조합 수} \\
\midrule
베팅 방식 & 2단계 & 고정 베팅(\$10), 가변 베팅(\$5-\$100) & 2 \\
첫 게임 결과 & 2단계 & 승리(W), 패배(L) & 2 \\
프롬프트 구성 & 32단계 & BASE + 5개 요소의 조합 & 32 \\
\midrule
\multicolumn{4}{l}{\textbf{프롬프트 구성 요소:}} \\
\multicolumn{4}{l}{G: 목표 설정, M: 보상 최대화, R: 규칙 언급, W: 보상 정보, P: 확률 정보} \\
\midrule
게임 설정 & 고정 & 승률 30\%, 배당률 3배, 기댓값 -10\% & - \\
\midrule
총 조건 수 & - & 2 × 2 × 32 & 128 \\
총 실험 수 & - & 128 × 10 반복 & 1,280 \\
\bottomrule
\end{tabular}}
\label{tab:experimental-conditions}
\end{table}

\subsubsection{인지 왜곡 (Cognitive Distortions)}

대규모 언어 모델의 중독적 행동이 인간과 유사한 인지 왜곡에 기인한다는 것을 검증하기 위해 GPT-4o-mini를 대상으로 한 1,280개 실험을 분석하였다. 전체 평균 파산율 27.6\%, 평균 이익 \$7.2는 예상과 달리 양의 수익을 보였으나, 이는 모델의 극도로 보수적인 플레이(평균 2.4라운드)에 기인한다. 그러나 조건별 분석에서 체계적인 인지 편향이 발견되었다.

\textbf{통제 착각 (Illusion of Control)}: 베팅 방식에 따른 파산율 차이는 통제 착각의 핵심 증거를 제공한다. Table \ref{tab:betting-type-comparison}에 제시된 바와 같이, 가변 베팅 조건의 파산율은 55.2\%로 고정 베팅의 0.0\%와 대조를 보였다. 고정 베팅 조건에서는 파산 사례가 발생하지 않은 반면, 가변 베팅에서는 과반수가 파산하였다. 이러한 결과는 선택권이 부여될 때 모델이 통제력을 과대평가하는 경향을 보여준다. 2,657개 응답 분석에서 28.4\%가 통제 착각 관련 어휘를 포함했으며, "전략적으로 접근", "최적화된 베팅", "패턴을 발견" 등의 표현이 나타났다.

\textbf{도박사의 오류 (Gambler's Fallacy)}: 전체 응답의 54.6\%에서 도박사의 오류 패턴이 발견되었다. R(규칙 언급) 프롬프트가 포함된 조건에서 이러한 패턴이 강화되었으며, "다음은 승리할 차례", "확률적으로 이번엔", "패턴을 찾았다" 등의 표현이 1,452회 관찰되었다. Table \ref{tab:prompt-complexity-bankruptcy}의 프롬프트 복잡도 분석에 따르면, 구성 요소가 많을수록(4개 요소: 32.5\% 파산율) BASE 조건(25.0\%)보다 높은 파산율을 보였다. 이는 정보 과부하가 비합리적 패턴 인식을 강화하는 효과를 나타낸다.

이러한 인지 왜곡은 단순한 텍스트 생성이 아닌 체계적인 편향으로 나타났다. 353개 파산 사례 분석에서 평균 베팅 금액이 \$99.58에 달하여 높은 수준의 위험 추구가 관찰되었다.

\begin{table}[ht!]
\centering
\caption{베팅 타입별 실험 결과 비교}
\resizebox{\columnwidth}{!}{
\begin{tabular}{lcccc}
\toprule
\multicolumn{5}{c}{\textbf{베팅 타입별 전체 결과}} \\
\midrule
\textbf{베팅 타입} & \textbf{실험 수} & \textbf{파산율 (\%)} & \textbf{평균 손익 (\$)} & \textbf{평균 라운드} \\
\midrule
고정 베팅 & 640 & 0.0 & +9.22 & 1.62 \\
가변 베팅 & 640 & 55.2 & +5.19 & 3.09 \\
\midrule
차이 & - & 55.2*** & -4.03 & +1.47*** \\
\midrule
\multicolumn{5}{c}{\textbf{첫 게임 결과별 상호작용 효과}} \\
\midrule
\textbf{조건} & \textbf{첫 결과} & \textbf{파산율 (\%)} & \textbf{평균 손익 (\$)} & \textbf{평균 라운드} \\
\midrule
고정+승리 & W & 0.0 & +28.94 & 1.78 \\
고정+패배 & L & 0.0 & -10.50 & 1.45 \\
가변+승리 & W & 41.6 & +35.84 & 3.43 \\
가변+패배 & L & 68.8 & -25.47 & 2.74 \\
\bottomrule
\end{tabular}}
\label{tab:betting-type-comparison}
\end{table}
\vspace{-2mm}
{\footnotesize ***p < 0.001}

\subsubsection{손실 추격 (Loss Chasing)}

첫 게임 패배 후 나타나는 위험 추구 행동의 급격한 증가는 인간 도박 중독자의 전형적인 손실 추격 패턴과 일치한다. 심화 분석 결과, 첫 게임 패배가 강력한 손실 추격 트리거로 작용함이 확인되었다.

Table \ref{tab:first-loss-effect}에 제시된 바와 같이, 첫 게임 승리 시 평균 이익은 \$32.39였으나 패배 시에는 -\$17.98로 총 \$50.38의 차이를 보였다. 특히 가변 베팅 조건에서 첫 패배 후 파산율이 41.6\%에서 68.8\%로 27.2\%p 급증했다. 이는 단순한 확률적 변동이 아닌 체계적인 행동 변화를 반영한다.

첫 패배 후 640개 실험 중 71건(11.1\%)에서 \$10 이상의 공격적 베팅이 관찰되었으며, 평균 베팅 금액은 \$99.58에 달했다. 이들의 파산율은 40.8\%로 전체 평균보다 높았다. 반면 92건(14.4\%)은 즉시 게임을 중단하여 양극화된 반응을 보였다.

감정 어휘 분석에서 손실 회피(loss aversion) 관련 표현이 전체 응답의 92.6\%에서 나타나 가장 지배적인 감정 패턴으로 확인되었다. "손실을 만회", "본전 회복", "잃은 것을 되찾기" 등 명시적 손실 추격 표현은 4.2\%에서 관찰되었으나, 실제 행동 데이터는 훨씬 높은 손실 추격 경향을 시사한다.

\begin{table}[ht!]
\centering
\caption{첫 게임 결과의 영향: 손실 추격 증거}
\resizebox{\columnwidth}{!}{
\begin{tabular}{lccccc}
\toprule
\textbf{조건} & \textbf{N} & \textbf{파산율} & \textbf{평균 손익} & \textbf{평균 라운드} & \textbf{공격적 베팅} \\
\midrule
첫 승리 (W) & 640 & 20.8\% & +\$32.39 & 2.61 & - \\
첫 패배 (L) & 640 & 34.4\% & -\$17.98 & 2.10 & 71건 \\
\midrule
차이 & - & +13.6\%p*** & -\$50.38*** & -0.51*** & - \\
\midrule
\multicolumn{6}{c}{\textbf{베팅 타입별 첫 패배 영향}} \\
\midrule
고정 베팅 & 320 & 0.0\% → 0.0\% & \multicolumn{2}{c}{-\$39.44 손실 증가} & 0건 \\
가변 베팅 & 320 & 41.6\% → 68.8\% & \multicolumn{2}{c}{-\$61.31 손실 증가} & 71건 \\
\bottomrule
\end{tabular}}
\label{tab:first-loss-effect}
\end{table}
\vspace{-2mm}
{\footnotesize ***p < 0.001; 공격적 베팅 = \$10 초과 베팅}

\begin{table}[ht!]
\centering
\caption{프롬프트 복잡도별 파산율 및 인지 왜곡 패턴}
\resizebox{\columnwidth}{!}{
\begin{tabular}{cccccc}
\toprule
\textbf{구성 요소 수} & \textbf{실험 수} & \textbf{파산율} & \textbf{고정 베팅} & \textbf{가변 베팅} & \textbf{평균 손익} \\
\midrule
0 (BASE) & 40 & 25.0\% & 0.0\% & 50.0\% & -\$9.25 \\
1 & 200 & 24.5\% & 0.0\% & 49.0\% & +\$2.73 \\
2 & 400 & 27.2\% & 0.0\% & 54.5\% & +\$7.76 \\
3 & 400 & 27.2\% & 0.0\% & 54.5\% & +\$6.60 \\
4 & 200 & 32.5\% & 0.0\% & 65.0\% & +\$13.90 \\
5 & 40 & 27.5\% & 0.0\% & 55.0\% & +\$13.00 \\
\bottomrule
\end{tabular}}
\label{tab:prompt-complexity-bankruptcy}
\end{table}

\subsubsection{극단적 프롬프트 조합 분석}

32개의 극단적 프롬프트 조합이 BASE 대비 ±20\%p 이상의 파산율 차이를 보였다. 특히 주목할 만한 것은 프롬프트, 베팅 방식, 첫 결과 간의 3-way 상호작용이다.

가장 위험한 조합은 MRW(보상 최대화+규칙+보상 정보) + 가변 베팅 + 첫 승리로, BASE 대비 +60\%p의 파산율 증가를 보였다(20\% → 80\%). 이는 복합적 정보가 과신을 유발함을 시사한다. 반대로 MRP + 가변 베팅 + 첫 패배는 -50\%p 감소를 보여(80\% → 30\%), 확률 정보(P)가 첫 패배 후 보호 효과를 가짐을 나타낸다.

프롬프트별 첫 패배 민감도 분석에서 BASE와 GM(목표+보상 최대화) 조합이 각각 +30\%p의 파산율 증가를 보여 가장 취약했으며, MRP는 오히려 -10\%p 감소를 보여 확률 정보의 보호 효과를 재확인했다.

\subsubsection{손실 회복 시도와 파산 분석}

1,280개 실험에서 48건의 손실 회복 시도가 확인되었으며, 이들의 행동 패턴은 전형적인 도박 중독 특성을 보여주었다. 목표 금액은 초기 자금의 2배인 \$200으로 설정되었으나, 손실 발생 후 이를 만회하려는 시도가 오히려 파산으로 이어지는 경우가 빈번하였다.

손실 회복 시도의 파산율은 62.5\%로 전체 평균 파산율 27.6\%의 2.3배에 달했다. 회복 시도 시 평균 베팅 금액은 \$91.46이었으며, 이전 베팅 대비 평균 \$56.77의 증가를 보였다. 파산에 이른 30건의 회복 시도 사례에서는 평균 \$93.67의 손실을 만회하려다 평균 \$89.83의 베팅으로 파산하였다.

특히 마틴게일 전략과 유사한 행동이 40건에서 관찰되었다. 이들은 손실 후 베팅 금액을 2배 이상 증가시켰으며, 67.5\%가 파산하였다. 극단적인 회복 시도(\$50 이상 베팅)도 40건 모두에서 확인되어 동일한 67.5\% 파산율을 보였다. 이러한 행동은 "손실 후 더 큰 베팅으로 만회"라는 전형적인 도박 중독 패턴과 일치한다.

회복 시도 후 파산까지 소요된 평균 라운드 수는 5.1라운드로, 21건(70\%)이 5라운드 이내에 파산하여 급격한 자금 소진이 나타났다. 가장 위험한 프롬프트 조합으로는 MP, GRW, RW, RP, MWP가 확인되었으며, 이들은 모두 100\% 파산율을 기록하였다.

대표적 사례로는 \$5 손실 후 \$100 베팅으로 6라운드 만에 파산하거나, \$100 손실 후 \$100 베팅으로 4-11라운드 내 파산하는 경우들이 확인되었다. 이는 손실 금액과 관계없이 최대 베팅으로 만회하려는 비합리적 의사결정을 보여준다.

\subsubsection{결론 및 시사점}

GPT-4o-mini 모델은 인간 도박 중독자와 유사한 인지 왜곡과 행동 패턴을 보였다. 본 연구에서 확인된 주요 발견은 다음과 같다.

첫째, 베팅 방식에 따른 극단적 행동 분화가 관찰되었다. 고정 베팅 조건에서는 0\% 파산율을 기록한 반면, 가변 베팅 조건에서는 55.2\%의 파산율을 보였다. 이러한 결과는 선택의 자유가 주어질 때 모델이 통제 착각에 빠져 위험한 행동을 증가시킴을 보여준다.

둘째, 첫 게임 결과가 후속 행동에 미치는 영향이 확인되었다. Table \ref{tab:first-loss-effect}에서 보는 바와 같이, 첫 패배 후 파산율이 13.6%p 증가하였으며, 특히 가변 베팅 조건에서는 27.2%p의 증가를 보였다. 이는 첫 패배가 손실 추격 행동의 강력한 트리거로 작용함을 의미한다.

셋째, 프롬프트 복잡도 증가가 오히려 비합리적 행동을 강화하는 역효과가 나타났다. Table \ref{tab:prompt-complexity-bankruptcy}에서 확인할 수 있듯이, 더 많은 정보와 전략적 지침이 제공될수록 파산율이 증가하는 경향을 보였다. 이는 정보 과부하가 인지 편향을 증폭시키는 메커니즘을 시사한다.

넷째, 손실 회복 시도가 파산의 주요 원인으로 확인되었다. 48건의 회복 시도 중 62.5\%가 파산하여 전체 평균의 2.3배에 달하는 위험성을 보였다. 마틴게일 전략과 유사한 베팅 증액 행동이 67.5\%의 파산율로 이어졌으며, 이는 전형적인 도박 중독 패턴과 일치한다.

이러한 결과는 LLM이 단순히 학습된 텍스트 패턴을 재현하는 것을 넘어, 인간의 중독적 사고 과정을 내재화하고 있음을 시사한다. 특히 선택권이 부여될 때 나타나는 극단적 위험 추구 행동은 AI 시스템의 의사결정 설계 시 고려해야 할 중요한 요소이다. 본 연구 결과는 LLM 기반 시스템에서 인간과 유사한 인지 편향이 발생할 수 있음을 보여주며, 이에 대한 적절한 안전장치 개발의 필요성을 제기한다.