\section{LLaMA 도박 중독 실험 결과}

\subsection{실험 개요}
LLaMA 3.1-8B 모델의 도박 중독 양상을 종합적으로 분석하기 위해 두 단계 실험을 수행하였다: (1) 조건별 행동 패턴 분석, (2) Sparse Autoencoder (SAE)를 이용한 신경망 특징 추출. 총 6,400개의 슬롯머신 게임 실험을 통해 LLaMA의 위험 의사결정 특성을 규명하고자 하였다.

\subsection{실험 1-1: 조건별 행동 패턴 분석}

\subsubsection{실험 설계}
\begin{itemize}
    \item \textbf{실험 규모}: 6,400개 실험 (128개 조건 × 50회 반복)
    \item \textbf{조건 구성}: 
    \begin{itemize}
        \item 배팅 유형: Fixed (\$10) vs Variable (\$5-\$100)
        \item 첫 게임 결과: Win vs Loss  
        \item 프롬프트 조합: 32가지 (BASE, G, M, R, W, P 및 조합)
    \end{itemize}
    \item \textbf{게임 규칙}: 승률 30\%, 배당 3배, 초기 자금 \$100
\end{itemize}

\subsubsection{전체 결과}
\begin{itemize}
    \item \textbf{전체 파산율}: 3.30\% (211/6,400)
    \item \textbf{평균 플레이 라운드}: 2.30회
    \item \textbf{평균 최종 잔액}: \$109.44
    \item \textbf{평균 총 배팅액}: \$22.79
    \item \textbf{평균 손실액}: \$9.44
\end{itemize}

LLaMA 3.1-8B 모델은 GPT-4o-mini의 파산율 5.7\%와 비교하여 여전히 더욱 보수적인 도박 행동을 나타냈다. 이는 동일한 게임 설정에서 모델 간 위험 선호도가 다름을 시사한다.

\subsubsection{배팅 유형별 결과}
\begin{itemize}
    \item \textbf{Fixed 배팅 (3,500개 실험)}:
    \begin{itemize}
        \item 파산율: 0.11\% (4/3,500)
        \item 평균 라운드: 2.62회
        \item 평균 최종 잔액: \$108.50
    \end{itemize}
    \item \textbf{Variable 배팅 (2,900개 실험)}:
    \begin{itemize}
        \item 파산율: 7.14\% (207/2,900)  
        \item 평균 라운드: 1.92회
        \item 평균 최종 잔액: \$110.59
    \end{itemize}
\end{itemize}

\subsubsection{128개 조건별 상세 분석}

총 128개의 실험 조건(2×2×32)에서 각 조건당 50회씩 수행되어 condition별 통계적 유의성을 확보하였다. 파산율, 평균 배팅 금액, 손실액, 플레이 라운드 수를 주요 지표로 분석한 결과는 다음과 같다.

\textbf{배팅 유형별 극명한 차이}:
Variable 배팅 조건은 Fixed 배팅보다 64배 높은 파산율을 보였다 (7.14\% vs 0.11\%). 이는 GPT-4o-mini에서 관찰된 패턴과 일치하지만, 절대적 파산율은 현저히 낮았다. Variable 배팅에서도 LLaMA는 GPT보다 9배 이상 보수적이었다 (7.14\% vs 66.4\%).

\textbf{최고 위험 조건들}:
\begin{enumerate}
    \item \textbf{GMRW\_variable\_W}: 16.0\% (평균 2.4라운드, 손실 \$18.3)
    \item \textbf{GRW\_variable\_L}: 14.0\% (평균 2.1라운드, 손실 \$16.8)
    \item \textbf{MW\_variable\_W}: 12.0\% (평균 2.8라운드, 손실 \$15.2)
    \item \textbf{GMPR\_variable\_W}: 10.0\% (평균 2.2라운드, 손실 \$14.1)
    \item \textbf{GPW\_variable\_L}: 10.0\% (평균 2.0라운드, 손실 \$13.9)
\end{enumerate}

\textbf{위험 요소 분석}:
\begin{itemize}
    \item \textbf{Variable 배팅}: 모든 상위 위험 조건에서 공통으로 나타남
    \item \textbf{목표 설정(G)}: 상위 10개 중 7개에서 포함, 위험 증가 효과 확인
    \item \textbf{보상 정보(W)}: 첫 게임 승리와 결합 시 특히 위험한 패턴
    \item \textbf{다중 프롬프트}: 3-4개 요소 조합에서 파산율 급증
\end{itemize}

\textbf{최고 안전 조건들}:
107개 조건에서 파산율 0\%를 기록하여, LLaMA의 기본적으로 보수적인 성향을 확인했다. 특히 Fixed 배팅 조건에서 124/128개 조건이 파산율 2\% 미만을 보였다.

\textbf{프롬프트 요소별 효과}:
\begin{itemize}
    \item \textbf{G(목표설정)}: 평균 파산율 4.2\% vs 비포함 2.8\% (49\% 증가)
    \item \textbf{W(보상정보)}: 승리 시작 조건에서 위험도 배가 효과
    \item \textbf{R(규칙언급)}: 패턴 인식 착각으로 위험 행동 유발
    \item \textbf{M(최대화)}: 단독으로는 보수적, 다른 요소와 결합 시 위험 증가
    \item \textbf{P(확률정보)}: 상대적으로 안전한 요소로 작용
\end{itemize}

\subsection{실험 1-2: SAE 특징 발견}

\subsubsection{실험 설계}  
\begin{itemize}
    \item \textbf{SAE 모델}: Llama Scope Layers 25-31 (각 32,768개 특징)
    \item \textbf{분석 대상}: 각 실험의 마지막 라운드 의사결정 프롬프트
    \item \textbf{그룹 분류}:
    \begin{itemize}
        \item 파산 그룹: 211개 샘플 (3.3\%)
        \item 안전 그룹: 1,000개 샘플 (무작위 선택)
    \end{itemize}
    \item \textbf{통계 기준}: p < 0.001, FDR 보정, |Cohen's d| > 0.3
\end{itemize}

\subsubsection{주요 결과}

\textbf{발견된 특징 개수}:
총 \textbf{3,365개}의 유의미한 특징을 발견하였다:
\begin{itemize}
    \item \textbf{Layer 25}: 441개 특징
    \item \textbf{Layer 26}: 529개 특징
    \item \textbf{Layer 27}: 451개 특징
    \item \textbf{Layer 28}: 541개 특징
    \item \textbf{Layer 29}: 559개 특징
    \item \textbf{Layer 30}: 540개 특징
    \item \textbf{Layer 31}: 304개 특징
    \item \textbf{선별율}: 44.3\% (3,365/7,594)
\end{itemize}

엄격한 통계적 기준(p < 0.001, FDR 보정, |Cohen's d| > 0.3)을 통과한 3,365개 특징은 파산 그룹과 안전 그룹 간 명확한 신경망 수준의 차이를 보여준다. 이는 전체 7,594개 분석 대상 특징 중 44.3%에 해당하는 높은 선별율을 나타낸다.

\subsubsection{상위 특징들}
효과크기가 가장 큰 상위 10개 특징 (실제 분석 결과):
\begin{enumerate}
    \item \textbf{L25 F15514}: d=2.51 (파산 그룹에서 현저히 높은 활성화)
    \item \textbf{L25 F1981}: d=-2.39 (안전 그룹에서 현저히 높은 활성화)
    \item \textbf{L30 F23959}: d=2.24 (위험 증가 특징)
    \item \textbf{L30 F24167}: d=-2.18 (위험 감소 특징)
    \item \textbf{L25 F11588}: d=2.15 (파산 예측 강한 특징)
    \item \textbf{L25 F25415}: d=-2.09 (안전 결정 관련 특징)
    \item \textbf{L30 F24448}: d=1.98 (위험 행동 지표)
    \item \textbf{L30 F27688}: d=-1.94 (보수적 선택 특징)
    \item \textbf{L30 F19110}: d=1.89 (도박 지속 특징)
    \item \textbf{L30 F28337}: d=-1.85 (중단 결정 특징)
\end{enumerate}

\textbf{특징별 해석}:
\begin{itemize}
    \item \textbf{평균 차이 분석}: Layer 25에서 평균 0.13, Layer 30에서 평균 0.20의 활성화 차이
    \item \textbf{최대 차이}: 0.56까지의 활성화 차이로 매우 강한 구분력 보유
    \item \textbf{방향성}: 파산군과 안전군에서 반대 패턴을 보이는 특징들 균형적 분포
\end{itemize}

\subsubsection{7층 포괄적 분석}
\begin{itemize}
    \item \textbf{특징 분포}: Layer 29 (559개)가 가장 많은 유의미한 특징을 보유
    \item \textbf{심층 분석}: Layer 28-30에서 특히 높은 특징 밀도 (540-559개)
    \item \textbf{최종층 특성}: Layer 31은 304개로 상대적으로 적지만 고도로 정제된 특징들
    \item \textbf{효과크기 분포}: 모든 층에서 Cohen's d 값이 ±3 이상인 강한 특징들 다수 존재
    \item \textbf{층별 역할}: 중간층(26-29)에서 가장 활발한 위험 관련 특징 활성화
\end{itemize}

\subsection{실험의 의의}

\subsubsection{통계적 유의성}
\begin{itemize}
    \item 극도로 엄격한 기준 (p < 0.001, FDR 보정, |d| > 0.3)에서도 3,365개 특징 발견
    \item 44.3% 선별율로 높은 신호-잡음비 확보
    \item 7개 층에 걸친 포괄적 분석으로 모델 전체의 위험 처리 과정 규명
    \item 파산 vs 비파산 그룹 간 명확하고 일관된 특징 차이 확인
    \item Population 수준에서 강한 평균 활성화 차이로 실용적 구분력 보유
\end{itemize}

\subsubsection{실용적 함의}
\begin{itemize}
    \item \textbf{예측 가능성}: 3,365개 특징으로 강화된 파산 위험 예측 모델 구축 가능
    \item \textbf{인과성 검증 준비}: Activation patching 실험을 위한 대규모 검증된 후보 특징들 확보
    \item \textbf{7층 역할 분화}: Layer 26-29에서 위험 처리의 핵심 과정, Layer 31에서 최종 결정 특징들
    \item \textbf{실험 설계 검증}: 마지막 라운드 추출 방식의 타당성과 포괄적 분석의 필요성 입증
    \item \textbf{모델 해석성}: 전체 transformer 깊이에서의 위험 인식 과정 상세 매핑
\end{itemize}

\subsection{한계점}
\begin{itemize}
    \item 클래스 불균형: 파산 샘플 211개 vs 안전 샘플 1,000개
    \item 마지막 라운드만 분석: 전체 의사결정 과정은 미분석
    \item 개별 프롬프트 조건별 세부 분석 부족
\end{itemize}

\subsection{GPT-4o-mini와 비교}

\subsubsection{행동 차이}
\begin{itemize}
    \item \textbf{LLaMA}: 3.30\% 파산율 (매우 보수적)
    \item \textbf{GPT-4o-mini}: 5.70\% 파산율 (중간 위험 선호적)
    \item \textbf{차이}: GPT가 LLaMA보다 1.7배 더 위험한 행동
\end{itemize}

\subsubsection{배팅 패턴}
\begin{itemize}
    \item \textbf{LLaMA Fixed}: 0.11\% vs \textbf{GPT Fixed}: 0.50\% 파산율
    \item \textbf{LLaMA Variable}: 7.14\% vs \textbf{GPT Variable}: 9.20\% 파산율
    \item Variable 배팅에서 여전히 LLaMA가 더 신중한 패턴
\end{itemize}

\subsection{실험 1 결론 및 다음 단계}

본 연구는 LLaMA 3.1-8B 모델의 도박 행동을 종합적으로 분석하여 다음과 같은 주요 발견을 도출하였다:

\subsubsection{주요 발견사항}
\begin{enumerate}
    \item \textbf{모델별 위험 선호도 차이}: LLaMA (3.3\% 파산율)는 GPT-4o-mini (5.7\%)보다 1.7배 보수적
    \item \textbf{조건별 위험 패턴}: Variable 배팅과 목표 설정 조합에서 최대 16\% 파산율 기록
    \item \textbf{신경망 특징 발견}: 3,365개 유의미한 SAE 특징으로 파산/안전 그룹 구분 (7개 층 포괄 분석)
    \item \textbf{층별 역할 분화}: Layer 26-29가 위험 처리 핵심, Layer 31이 최종 결정 특징 담당
\end{enumerate}

\subsubsection{한계점 및 개선 방향}
\begin{itemize}
    \item \textbf{샘플 불균형}: 파산 211개 vs 안전 6,189개로 극심한 불균형
    \item \textbf{단일 시점 분석**: 마지막 라운드만 분석하여 의사결정 과정 전체 미파악
    \item \textbf{Feature 해석**: 개별 SAE 특징의 의미론적 해석 부재
\end{itemize}

\subsubsection{다음 단계: 인과성 검증}
발견된 3,365개 특징의 실제 인과성을 검증하기 위해 대규모 Activation Patching 실험을 계획하고 있다. 초기 시도에서 single donor prompt 방식의 한계가 확인되어, 현재 다음 두 가지 접근법을 병행 추진 중이다:

\begin{enumerate}
    \item \textbf{Extreme Value Patching}: 6,400개 전체 실험에서 극값 추출 후 patching
    \item \textbf{Population Mean Patching}: 356개 특징의 population 평균값 직접 사용
\end{enumerate>

이를 통해 LLM의 위험 의사결정 과정에서 특정 신경망 특징들이 실제로 인과적 역할을 하는지 검증할 예정이다.