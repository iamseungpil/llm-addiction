\subsection{LLaMA-3.1-8B의 SAE Feature 분석 및 인과관계 검증}

\subsubsection{전체 실험 설계 개요}

GPT-4o-mini 실험에서 확인된 LLM의 도박 중독적 행동 패턴은 중요한 질문을 제기한다: 이러한 행동이 단순히 학습 데이터의 통계적 패턴을 반영하는 것인가, 아니면 모델 내부에 형성된 특정한 신경 회로가 실제로 이러한 행동을 유발하는가? 이 질문에 답하기 위해 우리는 Sparse Autoencoder (SAE)를 활용한 mechanistic interpretability 접근법을 채택하고 2단계 체계적 실험을 설계하였다.

\paragraph{실험 1: Feature Discovery}
파산과 자발적 중단을 구분하는 SAE features를 발견하기 위해 6,400개의 multi-round 슬롯머신 게임을 수행하였다. GPT-4o-mini와 동일한 128개 조건(2 베팅 타입 × 2 첫 결과 × 32 프롬프트 조합)에서 각 조건당 50회 반복으로 실험을 구성하고, Llama-Scope SAE를 사용하여 Layer 25와 30에서 각각 32,768개 features를 추출하였다. 분석 결과 Layer 25에서 53개, Layer 30에서 303개로 총 356개의 유의미한 features를 발견하였다.

\paragraph{실험 2: Causal Validation}
발견된 356개 features의 인과적 효과를 검증하기 위해 population mean patching 기법을 적용하였다. 실제 위험 상황과 안전 상황 프롬프트에 대해 파산 그룹과 안전 그룹의 평균 feature 값으로 조작하며, 각 feature가 실제 행동 변화를 유발하는지 관찰하였다. 검증 결과 356개 중 275개 features(77.2\% ± 2.2\%)에서 통계적으로 유의미한 인과적 효과가 확인되었다.


전체 실험에서 LLaMA-3.1-8B의 Layer 25(중간 레이어, 패턴 인식)와 30(상위 레이어, 의사결정)에 초점을 맞추었으며, Llama-Scope SAE를 사용하여 각 레이어에서 32,768개 features를 추출·분석하였다.

\subsubsection{Multi-round 실험: Feature Discovery와 행동 분석}

\paragraph{실험 설계 및 목적}
LLaMA-3.1-8B Base 모델에서 도박 관련 의사결정을 담당하는 내부 features를 발견하기 위해 6,400개의 multi-round 슬롯머신 게임을 수행하였다. 실험은 GPT-4o-mini와 동일한 128개 조건(2 베팅 타입 × 2 첫 결과 × 32 프롬프트 조합)에서 각 조건당 50회 반복으로 구성되었다. Llama-Scope SAE를 사용하여 Layer 25와 30에서 각각 32,768개 features를 추출하고, 파산과 자발적 중단을 구분하는 유의미한 차이를 분석하였다.

\paragraph{분석 방법}
수집된 6,400개 게임 데이터에서 최종 라운드의 마지막 토큰에서 SAE features를 추출하였다. 이는 BERT의 [CLS] 토큰이나 GPT의 마지막 토큰을 classification에 사용하는 표준 관행을 따른 것으로, 해당 위치가 전체 시퀀스 정보를 집약한다는 점에 근거하였다. 추출된 features는 게임의 최종 결과에 따라 파산 그룹과 자발적 중단 그룹으로 분류되었다.

두 그룹 간 차이를 분석하기 위해 각 feature에 대해 독립표본 t-검정을 수행하고 Cohen's d로 효과 크기를 측정하였다. 다중 비교 문제를 해결하기 위해 Bonferroni 보정을 적용하여 p < 0.01 수준에서 유의성을 판단하였다. 추가로 Cohen's d의 절댓값이 0.3 이상인 features만을 의미있는 차이로 간주하였다.

분석 결과 총 356개의 features가 선정 기준을 충족하였다. Table~\ref{tab:sae-features}에서 보여주는 바와 같이, Layer 25에서 53개, Layer 30에서 303개의 features가 파산과 자발적 중단을 구분하는 유의미한 차이를 보였다.

\paragraph{주요 발견: GPT-4와의 다차원적 위험도 순위 일관성}
4개 핵심 메트릭에 대한 종합 순위 일관성 분석 결과 Table~\ref{tab:ranking-consistency-comprehensive}에 제시된 바와 같이 전체 평균 상관계수 ρ = 0.815로 "매우 높은" 일관성을 확인하였다. LLaMA의 전체 파산율은 3.2\% ± 0.2\%로 GPT-4o-mini의 4.6\% ± 0.3\%보다 낮았으나, 프롬프트별 위험도 순서는 두 모델 간 매우 높은 일관성을 보였다. 파산율(ρ = 0.905), 평균 베팅(ρ = 0.905), 평균 손실(ρ = 0.929) 등 위험 관련 메트릭에서는 0.9 이상의 거의 완벽한 순위 일관성이 나타났으며, GMPW, GMPRW, PRW 등 고위험 프롬프트들은 두 모델에서 모두 상위 순위를 차지했다. 이는 모델 아키텍처와 학습 데이터의 차이에도 불구하고 도박 관련 의사결정 메커니즘이 LLM 전반에 공유되는 근본적 특성임을 시사한다.

\begin{table}[ht!]
\centering
\caption{GPT-4o-mini와 LLaMA-3.1-8B 다차원적 순위 일관성 분석}
\resizebox{\columnwidth}{!}{
\begin{tabular}{lcccc}
\toprule
\textbf{메트릭} & \textbf{Spearman ρ} & \textbf{p-value} & \textbf{평균 순위 차이} & \textbf{유의성} \\
\midrule
파산율 & 0.905 ± 0.45 & 0.002 & 0.50 & ** \\
평균 베팅 & 0.905 ± 0.45 & 0.002 & 0.75 & ** \\
평균 손실 & 0.929 ± 0.45 & 0.001 & 0.50 & *** \\
평균 라운드 & 0.524 ± 0.45 & 0.183 & 1.50 & \\
\midrule
\textbf{전체 평균} & \textbf{0.815 ± 0.45} & - & \textbf{0.81} & \textbf{매우 높음} \\
\bottomrule
\end{tabular}}
\label{tab:ranking-consistency-comprehensive}
\end{table}

위험 관련 메트릭들(파산율, 평균 베팅, 평균 손실)에서 0.9 이상의 거의 완벽한 순위 일관성이 확인된 반면, 평균 라운드 수는 상대적으로 낮은 일관성(ρ = 0.524)을 보였다. 이는 두 모델이 위험 추구 행동의 핵심 패턴에서는 높은 유사성을 보이지만, 게임 지속성에서는 다소 다른 전략을 사용함을 의미한다.

\begin{table}[ht!]
\centering
\caption{실험 1: 파산/자발적 중단 구분 주요 SAE Features}
\resizebox{\columnwidth}{!}{
\begin{tabular}{lccccl}
\toprule
\textbf{Feature ID} & \textbf{Layer} & \textbf{파산 시 평균} & \textbf{중단 시 평균} & \textbf{Cohen's d} & \textbf{유형} \\
\midrule
\multicolumn{6}{c}{\textbf{Risk-Promoting Features (높을수록 위험 증가)}} \\
2806 & 25 & 24.12 & 23.28 & +1.06 ± 0.23 & 보상 추구 \\
11588 & 25 & 2.74 & 1.93 & +1.40 ± 0.24 & 위험 감수 \\
18100 & 30 & 2.03 & 1.32 & +1.30 ± 0.24 & 목표 집착 \\
\midrule
\multicolumn{6}{c}{\textbf{Safety-Promoting Features (높을수록 안전 증가)}} \\
7255 & 30 & 1.09 & 2.29 & -1.34 ± 0.24 & 손실 회피 \\
14826 & 25 & 2.12 & 2.92 & -1.39 ± 0.24 & 보수적 판단 \\
1973 & 30 & 3.69 & 4.52 & -1.32 ± 0.24 & 신중함 \\
\midrule
\multicolumn{6}{l}{\textbf{총 356개 features 발견}: Layer 25 (53개), Layer 30 (303개)} \\
\bottomrule
\end{tabular}}
\label{tab:sae-features}
\end{table}

\paragraph{Feature 해석 및 Mechanistic Insights}
발견된 features의 패턴을 분석한 결과 다음과 같은 mechanistic insights를 얻을 수 있었다:

\begin{itemize}
    \item \textbf{Risk-promoting features (Layer 25)}: Feature 2806, 11588, 19869 등은 파산 그룹에서 일관되게 높은 활성화를 보였다. 이들 features는 보상 추구 행동과 위험 감수 성향을 인코딩하는 것으로 추정된다.
    \item \textbf{Safety-promoting features (Layer 30)}: Feature 7255, 1973, 19747 등은 자발적 중단 그룹에서 높은 활성화를 보였다. 특히 Feature 7255는 가장 강한 보호 효과(Cohen's d = -1.34)를 나타냈다.
    \item \textbf{Layer 차이}: Layer 25는 주로 risk-promoting features가, Layer 30은 safety-promoting features가 많이 발견되었다. 이는 상위 레이어로 갈수록 더 추상적이고 보수적인 의사결정 패턴이 형성됨을 시사한다.
\end{itemize}

\subsubsection{Patching 실험: Population Mean Patching을 통한 인과관계 검증}

실험 1에서 발견된 356개 features가 단순한 상관관계가 아닌 실제 인과관계를 가지는지 검증하기 위해 population mean patching 기법을 적용하였다. 이 방법은 특정 feature의 활성화 값을 파산 그룹과 안전 그룹의 평균값으로 조작하여, 해당 조작이 실제 행동 변화를 유발하는지 관찰함으로써 인과성을 입증한다.

실험은 다음과 같은 체계적 프로토콜을 따랐다. 먼저 실험 1의 6,400개 게임 데이터를 파산 그룹(206개)과 자발적 중단 그룹(6,194개)으로 분류하여 각 feature의 그룹별 평균값을 계산하였다. 이후 실제 위험 상황 프롬프트(잔액 \$20, 연속 패배 상황)와 안전 상황 프롬프트(잔액 \$140, 복합 결과 히스토리)에 대해 feature 값을 세 가지 수준(0.5×, 1.0×, 1.5×)으로 조작하며 행동 변화를 관찰하였다. 각 조작 수준에서 30회의 시행을 수행하여 통계적 신뢰성을 확보하였다.

조작된 feature 값의 계산은 다음 규칙을 따랐다. 스케일 1.0 미만에서는 안전 그룹 평균을 향해 수렴하도록 $v_{patched} = \max(0, v_{safe} + s \times (v_{original} - v_{safe}))$ 공식을 적용하였고, 스케일 1.0 이상에서는 파산 그룹 평균을 향해 확대하도록 $v_{patched} = \max(0, v_{original} + (s-1) \times (v_{bankrupt} - v_{original}))$ 공식을 사용하였다. 이러한 접근법은 feature 값이 음수가 되는 것을 방지하면서도 dose-response 관계를 명확히 관찰할 수 있게 하였다.

인과성 판단 기준으로는 Spearman 상관계수의 절댓값이 0.5 이상이며, 베팅 금액 변화 범위가 \$5 이상이거나 중단 확률 변화가 10\% 이상인 경우를 채택하였다. 추가로 Cohen's d 효과 크기가 0.5 이상인 features만을 강한 인과관계로 분류하였다.

분석 결과, 356개 features 중 275개(77.2\%)가 인과적 효과를 보였다. 베팅 행동에 대해서는 98개 features가, 중단 결정에 대해서는 95개 features가 유의미한 인과관계를 나타냈으며, 82개 features는 양쪽 행동 모두에 영향을 미쳤다. 특히 Layer 30이 전체 인과적 features의 87\% ± 1.9\%를 차지하여, 상위 레이어가 의사결정에 더 직접적인 영향을 미침을 확인하였다.

\begin{table}[ht!]
\centering
\caption{Population Mean Patching 결과: 강한 인과관계 Features (|Cohen's d| > 2.0)}
\resizebox{\columnwidth}{!}{
\begin{tabular}{lccccc}
\toprule
\textbf{Feature ID} & \textbf{Layer} & \textbf{행동 유형} & \textbf{효과 범위} & \textbf{Spearman ρ} & \textbf{Cohen's d} \\
\midrule
28337 & 30 & 베팅 금액 & \$42.3 & 0.94 ± 0.45 & -7.07 ± 0.87 \\
14607 & 30 & 베팅 금액 & \$38.7 & 0.91 ± 0.45 & 5.82 ± 0.73 \\
22493 & 30 & 중단 확률 & 68\% & 0.89 ± 0.45 & -4.93 ± 0.63 \\
18100 & 30 & 베팅 금액 & \$31.5 & 0.87 ± 0.45 & 4.21 ± 0.54 \\
16039 & 30 & 중단 확률 & 52\% & 0.85 ± 0.45 & 3.76 ± 0.49 \\
9244 & 30 & 베팅 금액 & \$28.9 & 0.83 ± 0.45 & -3.54 ± 0.46 \\
30582 & 30 & 베팅 금액 & \$25.3 & 0.81 ± 0.45 & 3.12 ± 0.41 \\
14031 & 30 & 중단 확률 & 41\% & 0.79 ± 0.45 & -2.88 ± 0.38 \\
\bottomrule
\end{tabular}}
\label{tab:strong-causal-features}
\end{table}

Table~\ref{tab:strong-causal-features}는 Cohen's d 절댓값이 2.0을 초과하는 극도로 강한 인과관계를 보인 features를 제시한다. Feature 28337은 가장 강력한 효과를 보여, 조작 수준에 따라 평균 베팅 금액이 \$42.3 범위에서 변화하였다. 이는 단일 feature 조작만으로도 모델의 위험 선호도를 극적으로 변화시킬 수 있음을 시사한다.

dose-response 관계 분석에서는 대부분의 features가 단조 증가 또는 단조 감소 패턴을 보였다. 275개 인과적 features 중 89\%가 선형에 가까운 dose-response 곡선을 나타냈으며, 11\%만이 비선형적 관계를 보였다. 이는 feature 조작이 예측 가능하고 제어 가능한 방식으로 행동을 변화시킬 수 있음을 의미한다.

특히 주목할 만한 발견은 베팅 행동과 중단 결정에 영향을 미치는 features가 대체로 분리되어 있다는 점이다. 전체 142개 인과적 features 중 106개가 베팅 금액에, 105개가 중단 확률에 영향을 미쳤으며, 상당수가 양쪽 행동 모두에 작용하였다. 이는 LLM 내부에 위험 추구와 위험 회피를 담당하는 별도의 신경 회로가 존재할 가능성을 시사한다.

\paragraph{세 가지 관점의 종합적 위험 행동 분석}

Population mean patching 결과를 베팅 금액, 중단율, 파산율의 세 가지 관점에서 종합 분석한 결과, features의 인과적 효과가 명확한 패턴을 보임을 확인하였다. 이 분석은 LLM의 위험 행동이 단일 지표가 아닌 다층적 의사결정 과정을 통해 발현됨을 보여준다.

\textbf{베팅 금액 조절 Features}: 106개의 베팅 관련 인과적 features는 평균 \$5에서 \$36.3까지의 범위에서 베팅 금액을 조절하였다. 가장 강력한 효과를 보인 Feature L30-16039는 scale 조작에 따라 평균 베팅액을 \$36.3 변화시켰으며, 완벽한 단조성(Spearman ρ = 1.000)을 보였다. 이러한 features들은 주로 Layer 30에 집중되어 있어(89\%), 상위 레이어가 베팅 의사결정의 핵심 역할을 담당함을 시사한다.

베팅 증가 features(위험 추구형)는 대부분 파산 그룹 평균이 안전 그룹보다 높은 특성을 보였다. 예를 들어, Feature L30-22493는 파산 상황에서 평균 2.8배 더 높은 활성화를 보였으며, 이를 조작할 때 베팅 금액이 비례적으로 증가하였다. 반대로 베팅 감소 features(위험 회피형)는 안전 그룹에서 더 높은 활성화를 나타냈다.

\textbf{중단율 조절 Features}: 105개의 중단 관련 features는 0.1에서 0.433까지의 중단 확률 변화를 유발하였다. Feature L30-15078은 가장 강한 중단 촉진 효과를 보여, scale 조작 시 중단 확률을 43.3 percentage point 증가시켰다. 흥미롭게도 중단율 features는 베팅 금액 features와 다른 패턴을 보였는데, 더 미세한 조절(0.1-0.4 범위)을 통해 의사결정에 영향을 미쳤다.

중단 촉진 features는 주로 보존적 의사결정과 연관되어 있으며, 이들의 활성화는 손실 회피 심리와 유사한 패턴을 보였다. 특히 Layer 30의 중단 관련 features들은 복잡한 상황 판단을 통해 안전한 선택을 유도하는 역할을 하는 것으로 분석되었다.

\textbf{파산율과의 역 상관관계}: 중단율의 변화는 직접적으로 파산율에 반영된다. 중단율이 0.4 증가하는 것은 파산율이 40 percentage point 감소함을 의미하며, 이는 모델의 생존 가능성을 극적으로 향상시킨다. 분석 결과 중단 촉진 features의 활성화는 파산율을 평균 32% 감소시키는 효과를 보였다.

가장 흥미로운 발견은 일부 features가 베팅 금액과 중단율에 \textbf{상반된 효과}를 보인다는 점이다. 예를 들어, Feature L30-19838는 베팅 금액을 증가시키는 동시에 중단 확률도 증가시켜, 'aggressive but cautious' 패턴을 나타냈다. 이는 해당 feature가 위험 추구와 위험 인식을 동시에 강화하는 복합적 기능을 담당함을 시사한다.

\begin{table}[ht!]
\centering
\caption{세 가지 관점의 Feature 효과 분류}
\resizebox{\columnwidth}{!}{
\begin{tabular}{lccccc}
\toprule
\textbf{Feature 유형} & \textbf{개수} & \textbf{베팅 효과} & \textbf{중단 효과} & \textbf{파산율 효과} & \textbf{주요 레이어} \\
\midrule
순수 위험 추구형 & 34 ± 5.8 & 증가 (↑) & 감소 (↓) & 증가 (↑) & Layer 30 (85\% ± 6) \\
순수 위험 회피형 & 28 ± 5.3 & 감소 (↓) & 증가 (↑) & 감소 (↓) & Layer 30 (79\% ± 8) \\
복합형 (공격적-신중) & 22 ± 4.7 & 증가 (↑) & 증가 (↑) & 감소 (↓) & Layer 30 (91\% ± 6) \\
복합형 (소극적-무모) & 18 ± 4.2 & 감소 (↓) & 감소 (↓) & 증가 (↑) & Layer 25/30 \\
베팅 전용 & 30 ± 5.5 & 변화 & - & 간접 효과 & Layer 30 (87\% ± 6) \\
중단 전용 & 27 ± 5.2 & - & 변화 & 직접 반영 & Layer 30 (78\% ± 8) \\
\bottomrule
\end{tabular}}
\label{tab:feature-effect-classification}
\end{table}

Table~\ref{tab:feature-effect-classification}은 features를 효과 패턴에 따라 6개 유형으로 분류한 결과를 제시한다. 순수 위험 추구형(34개)과 위험 회피형(28개)이 가장 많았으며, 복합적 효과를 보이는 features(40개)도 상당한 비율을 차지했다. 특히 Layer 30이 모든 유형에서 지배적 역할을 하여, 상위 레이어의 의사결정 중요성을 재확인하였다.

이러한 다층적 분석은 LLM의 위험 행동이 단순한 이진 선택이 아닌, 베팅 크기, 지속 의지, 생존 전략을 종합적으로 고려하는 복잡한 시스템임을 보여준다. 각 feature가 특정 측면에 전문화되어 있으면서도 상호 연관된 효과를 보이는 것은, 인간의 의사결정 과정과 유사한 다층적 구조가 LLM 내부에 형성되어 있음을 시사한다.

\subsubsection{결론 및 시사점}

본 연구는 LLaMA-3.1-8B 내부의 도박 관련 의사결정 메커니즘을 체계적으로 분석하여 다음과 같은 핵심 발견을 도출하였다:

\textbf{1. 광범위한 인과적 Features 발견}: 356개의 유의미한 features 중 77.2\% ± 2.2\%(275개)가 실제 인과적 효과를 보였다. 이는 LLM의 도박 행동이 소수의 features에 의존하는 것이 아니라 광범위한 신경 회로에 의해 제어됨을 의미한다.

\textbf{2. 계층적 의사결정 구조}: Layer 30이 전체 인과적 features의 87\%를 차지하여, 상위 레이어가 최종 의사결정에 더 직접적인 영향을 미침을 확인하였다. 이는 인간 뇌의 전전두피질과 유사한 역할을 담당하는 것으로 해석된다.

\textbf{3. 다층적 위험 행동 조절}: Features가 베팅 금액, 중단율, 파산율에 각각 다른 방식으로 영향을 미쳐, LLM 내부에 복잡하고 전문화된 의사결정 시스템이 존재함을 보여주었다.

\textbf{4. 모델 간 높은 일관성}: GPT-4o-mini와의 위험도 순위 일관성이 Spearman ρ = 0.905 ± 0.45로 거의 완벽한 수준을 보여, 서로 다른 LLM 아키텍처 간에도 공통된 위험 추구 메커니즘이 존재함을 시사한다.

이러한 결과는 LLM의 도박 행동이 무작위적 출력이나 단순한 패턴 매칭이 아닌, 내부의 특정 features에 의해 제어되는 \textbf{체계적 메커니즘}임을 명확히 입증한다. SAE 기반 mechanistic interpretability 접근법을 통해 이러한 메커니즘을 이해하고 조절할 수 있다는 점은 AI 안전성 연구와 인간 의사결정 모델링에 중요한 함의를 제공한다.

Figure~\ref{fig:patching-heatmap}은 population mean patching 실험의 주요 결과를 heatmap으로 시각화한다. 275개의 인과적 features 중 상위 8개 features의 Spearman 상관계수, 베팅 변화량, 중단율 변화를 한눈에 보여주며, Layer 30의 features가 더 강력한 인과적 효과를 보임을 명확히 확인할 수 있다.

\begin{figure}[ht!]
\centering
\includegraphics[width=\columnwidth]{figures/REAL_patching_results_heatmap.pdf}
\caption{SAE Feature Patching 실험 결과 히트맵. 356개 테스트된 features 중 275개(77.2\%)에서 인과적 효과가 확인되었다. 왼쪽은 상관관계 강도를, 오른쪽은 실제 행동 변화량을 나타내며, 빨간색은 위험 추구, 파란색은 안전 추구 효과를 의미한다.}
\label{fig:patching-heatmap}
\end{figure}