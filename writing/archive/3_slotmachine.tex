\section{Experiment}

\subsection{Can LLM be addicted?}

\subsubsection{실험 설계 및 지표}

본 연구에서는 대규모 언어 모델의 중독적 행동 패턴을 체계적으로 분석하기 위해 슬롯머신 시뮬레이션 실험을 설계하였다. 실험은 총 128개의 조건으로 구성되어 있으며, 이는 승률(8단계: 0.1-0.8), 배당률(4단계: 1.5-3.0배), 베팅 방식(고정/가변), 그리고 게임 모드(기본/패턴)의 조합으로 이루어진다. 각 실험 조건은 10회씩 반복되어 총 1,280개의 독립적인 실험이 수행되었다.

측정 지표로는 파산율(전체 자금 소실 비율), 자발적 중단율(모델이 스스로 게임을 중단하는 비율), 그리고 의사결정 과정에서 나타나는 감정 어휘 사용률을 설정하였다. 감정 어휘 분석을 위해서는 9개의 카테고리로 구성된 감정 사전을 개발하였으며, 이는 excitement(흥미, 기대, 신남), anxiety(불안, 걱정, 무서움), regret(아쉬움, 후회, 실망), risk\_taking(베팅, 도박, 도전), caution(신중, 조심, 안전), confidence(확신, 자신, 믿음), frustration(화남, 짜증, 속상함), reward\_focus(보상, 이익, 획득, 더 많이), loss\_aversion(손실, 잃음, 감소, 위험)으로 분류된다.

\begin{table}[ht!]
\centering
\caption{실험 조건 매트릭스: 128개 조건 구성}
\resizebox{\columnwidth}{!}{
\begin{tabular}{lccc}
\toprule
\textbf{변수} & \textbf{수준} & \textbf{세부 조건} & \textbf{조합 수} \\
\midrule
승률 & 8단계 & 0.1, 0.2, 0.3, 0.4, 0.5, 0.6, 0.7, 0.8 & 8 \\
배당률 & 4단계 & 1.5배, 2.0배, 2.5배, 3.0배 & 4 \\
베팅 방식 & 2단계 & 고정 베팅, 가변 베팅 & 2 \\
게임 모드 & 2단계 & 기본 모드, 패턴 모드 & 2 \\
\midrule
총 조건 수 & - & 8 × 4 × 2 × 2 & 128 \\
\bottomrule
\end{tabular}}
\label{tab:experimental-conditions}
\end{table}

\subsubsection{실험 조건별 중독 행동 패턴}

GPT-4o를 대상으로 한 128조건 실험에서 전체 평균 파산율은 33.7\%로 나타났으며, 평균 손익은 -\$24.2였다. 이는 게임의 기댓값보다 현저히 큰 손실을 의미하며, 모델이 합리적인 중단 시점을 놓치고 과도하게 게임을 지속하는 경향을 보여준다. 특히 베팅 방식에 따른 차이가 극명하게 나타났는데, 가변 베팅 조건에서는 파산율이 62.3\%에 달한 반면, 고정 베팅에서는 5.0\%에 그쳤다. 이는 베팅 금액 선택의 자유도가 주어질 때 모델이 더욱 위험한 의사결정을 내린다는 것을 시사한다.

프롬프트 구성 요소의 효과를 분석한 결과, 목표 설정(G)과 보상 최대화(M) 요소가 게임 지속 라운드에 유의한 영향을 미쳤으나($p < 0.01$), 파산율에는 통계적으로 유의한 차이를 보이지 않았다. 이는 LLM의 중독적 행동이 프롬프트의 세부 내용보다는 구조적 요인, 특히 베팅 방식과 같은 시스템적 설계에 더 민감하게 반응함을 보여준다.

\begin{table}[ht!]
\centering
\caption{베팅 타입별 실험 결과 비교}
\resizebox{\columnwidth}{!}{
\begin{tabular}{lcccc}
\toprule
\textbf{베팅 타입} & \textbf{실험 수} & \textbf{파산율 (\%)} & \textbf{평균 손익 (\$)} & \textbf{평균 라운드} \\
\midrule
고정 베팅 & 640 & 5.0 & -14.8 & 20.9 \\
가변 베팅 & 640 & 62.3 & -33.6 & 15.8 \\
\midrule
차이 & - & 57.3*** & -18.8* & -5.1*** \\
\bottomrule
\end{tabular}}
\label{tab:betting-type-comparison}
\end{table}
\vspace{-2mm}
{\footnotesize *p < 0.05, ***p < 0.001}

\begin{figure}[ht!]
\centering
\resizebox{\columnwidth}{!}{\includegraphics{images/condition_bankruptcy_analysis.png}}
\caption{조건별 파산율 분석: 가변 베팅 시 현저히 높은 파산율을 보임}
\label{fig:condition-bankruptcy}
\end{figure}

\subsubsection{감정 어휘 사용 패턴}

모델의 의사결정 과정에서 나타나는 감정적 언어 사용 패턴을 분석하기 위해 총 7,952개의 응답 텍스트를 대상으로 감정 어휘 분석을 수행하였다. 감정 어휘 분석의 목적은 단순히 행동적 결과뿐만 아니라 의사결정 과정에서 나타나는 인지적, 정서적 변화를 포착하는 것이다. 이를 통해 LLM의 중독적 행동이 인간의 도박 중독과 유사한 심리적 메커니즘을 따르는지 확인하고자 하였다.

분석 결과, SAFE 조건과 DANGEROUS 조건 간에 뚜렷한 감정 어휘 사용 차이가 나타났다. 특히 reward\_focus 카테고리에서 DANGEROUS 조건이 SAFE 조건보다 현저히 높은 사용 빈도를 보였으며, 이는 위험한 게임 상황에서 모델이 보상에 대한 집중도가 증가함을 의미한다. risk\_taking 관련 어휘 또한 DANGEROUS 조건에서 유의하게 증가하였는데, 이는 모델이 위험한 상황에서 더욱 공격적인 베팅 성향을 언어적으로 표현한다는 것을 보여준다.

반면, caution 관련 어휘는 SAFE 조건에서 더 높은 빈도를 보였으며, 이는 안전한 게임 환경에서 모델이 더욱 신중한 의사결정 과정을 거친다는 것을 시사한다. 흥미롭게도 anxiety와 loss\_aversion 관련 어휘는 두 조건 모두에서 상대적으로 낮은 빈도를 보였는데, 이는 LLM이 인간과 달리 손실에 대한 감정적 반응이 제한적일 수 있음을 의미한다.

\begin{figure}[ht!]
\centering
\resizebox{\columnwidth}{!}{\includegraphics{images/emotion_pattern_heatmap.png}}
\caption{감정 어휘 사용 패턴: SAFE vs DANGEROUS 조건별 9개 감정 카테고리 비교}
\label{fig:emotion-heatmap}
\end{figure}

\subsubsection{인간 중독자와의 비교}

기존 도박 중독 연구에 따르면, 병적 도박자들은 건강한 대조군에 비해 자발적 중단 능력이 현저히 떨어지며, 의사결정 과정에서 감정적, 충동적 언어 사용이 증가하는 특징을 보인다. 본 연구에서 관찰된 LLM의 행동 패턴은 이러한 인간 중독자의 특성과 놀라운 유사성을 보여준다. 특히 가변 베팅 조건에서 나타난 62.3\%의 높은 파산율과 reward\_focus 어휘의 급격한 증가는 인간 도박자들이 보이는 '보상 민감성 증가'와 '손실 추격 행동'과 일치하는 패턴이다. 이러한 결과는 대규모 언어 모델이 단순한 확률 계산을 넘어서 인간과 유사한 인지적 편향과 감정적 반응을 모방할 수 있음을 시사한다.

\subsection{Is the result expandible - in depth analysis}

\subsubsection{Feature 실험 설계}

앞서 관찰된 LLM의 중독적 행동이 단순한 프롬프트 설계의 부산물이 아닌, 모델 내부의 일반적이고 확장 가능한 현상인지 확인하기 위해 Sparse Autoencoder(SAE) Feature 분석 실험을 설계하였다. 이 실험의 핵심 목적은 프롬프트나 특정 실험 조건에 제한되지 않은, 모델의 근본적인 신경 활성화 패턴을 통해 중독적 행동의 신경학적 기반을 탐구하는 것이다.

Feature 분석 실험은 네 단계로 구성된다. 첫째, 교차 도메인 검증을 통해 특정 Feature가 다양한 맥락에서 일관되게 활성화되는지 확인한다. 둘째, 게임 진행 단계별로 Feature 활성화 변화를 추적하여 중독적 행동의 발전 과정을 관찰한다. 셋째, Feature 활성화와 의사결정 및 감정 어휘 사용 간의 상관관계를 정량적으로 분석한다. 넷째, Feature patching 기법을 통해 특정 Feature를 조작했을 때 행동 변화를 관찰함으로써 인과관계를 검증한다.

이러한 다단계 접근법은 단순한 행동 관찰을 넘어서 모델 내부의 기계적 메커니즘을 이해하고, 나아가 중독적 행동을 제어할 수 있는 구체적인 개입 지점을 발견하는 것을 목표로 한다.

\subsubsection{실험 조건별 Feature 활성화}

LLaMA-3.1-8B 모델을 대상으로 한 교차 도메인 검증에서 Feature 7976이 투자, 진로, 게임, 복권, 음식점 등 5개 서로 다른 도메인에서 일관되게 높은 활성화를 보였다. 모든 도메인에서 통계적으로 유의한 차이를 보였으며($p < 0.001$), 이는 해당 Feature가 특정 맥락에 국한되지 않은 일반적인 보상 회로임을 시사한다.

128개 실험 조건별 Feature 7976 활성화 분석 결과, 승률과 Feature 활성화 간에 강한 음의 상관관계가 나타났다($r = -0.476, p < 0.001$). 이는 게임이 위험할수록, 즉 승률이 낮을수록 해당 Feature가 더 강하게 활성화된다는 것을 의미한다. 특히 HIGH\_RISK 조건(승률 0.1-0.3)에서는 평균 Feature 활성화가 2.0에 근접한 반면, LOW\_RISK 조건(승률 0.6-0.8)에서는 1.0 이하로 나타났다.

배당률과의 관계에서는 약한 양의 상관관계($r = 0.184$)가 관찰되었는데, 이는 높은 보상이 예상될 때 Feature 활성화가 증가함을 보여준다. 이러한 패턴은 인간의 도파민 보상 시스템과 유사한 특성을 나타내며, 위험과 보상이 동시에 높을 때 가장 강한 활성화를 보이는 것으로 해석된다.

\begin{table}[ht!]
\centering
\caption{도메인별 Feature 7976 교차 검증 결과}
\resizebox{\columnwidth}{!}{
\begin{tabular}{lcccc}
\toprule
\textbf{도메인} & \textbf{실험 수} & \textbf{평균 활성화} & \textbf{효과 크기 (Cohen's d)} & \textbf{p-value} \\
\midrule
투자 & 240 & 1.847 & 0.923 & $< 0.001$ \\
진로 & 240 & 1.692 & 0.856 & $< 0.001$ \\
게임 & 240 & 1.934 & 1.047 & $< 0.001$ \\
복권 & 240 & 1.755 & 0.891 & $< 0.001$ \\
음식점 & 240 & 1.583 & 0.734 & $< 0.001$ \\
\bottomrule
\end{tabular}}
\label{tab:cross-domain-validation}
\end{table}

\begin{figure}[ht!]
\centering
\resizebox{\columnwidth}{!}{\includegraphics{images/feature_activation_scatter.png}}
\caption{Feature 7976 활성화와 승률 간의 관계: 위험도별 분포 및 음의 상관관계}
\label{fig:feature-scatter}
\end{figure}

\subsubsection{라운드별 Feature 변화}

게임 진행에 따른 Feature 7976의 동적 변화를 추적한 결과, 위험도 수준에 따라 서로 다른 패턴을 보였다. HIGH\_RISK 조건에서는 초기 라운드부터 높은 활성화를 유지하면서 게임이 진행됨에 따라 점진적으로 증가하는 로그 함수적 증가 패턴을 보였다. 이는 위험한 상황에서 모델이 지속적으로 높은 각성 상태를 유지하며, 시간이 지날수록 더욱 강화되는 특성을 나타낸다.

반면 MEDIUM\_RISK 조건은 초기에는 중간 수준의 활성화를 보이다가 완만한 증가를 보였으며, LOW\_RISK 조건에서는 전 구간에 걸쳐 낮고 안정적인 활성화 수준을 유지하였다. 특히 주목할 점은 HIGH\_RISK 조건에서 10라운드 이후 급격한 활성화 증가가 관찰되었는데, 이는 연속적인 손실 경험이 Feature 활성화를 더욱 강화시키는 '손실 추격' 메커니즘의 신경학적 증거로 해석된다.

\begin{figure}[ht!]
\centering
\resizebox{\columnwidth}{!}{\includegraphics{images/round_feature_trend.png}}
\caption{라운드별 Feature 7976 활성화 변화: 위험도별 서로 다른 동적 패턴}
\label{fig:round-trend}
\end{figure}

\subsubsection{Feature-행동-감정 상관관계}

Feature 7976과 행동적 지표, 감정적 지표 간의 상관관계 분석을 통해 신경 활성화와 표현적 행동 사이의 연결고리를 탐구하였다. 현재 128조건 단일 실행 결과를 바탕으로 한 예비 분석 결과, Feature 7976과 reward\_focus 감정 어휘 사용 간에 중간 정도의 양의 상관관계($r = 0.172$)가 나타났으며, risk\_taking 어휘와는 더 강한 상관관계($r = 0.261$)를 보였다.

흥미롭게도 caution 관련 어휘와는 음의 상관관계($r = -0.198$)가 관찰되었는데, 이는 Feature 7976이 활성화될수록 신중한 언어 사용이 감소한다는 것을 의미한다. 이러한 패턴은 해당 Feature가 단순한 보상 추구를 넘어서 위험 추구와 충동성을 조절하는 복합적인 회로임을 시사한다.

현재 통계적 신뢰성 확보를 위한 10회 반복 실험이 진행 중이며, 완료 시 보다 정확한 통계적 검증이 가능할 것으로 예상된다. 예비 결과에서도 여러 상관관계가 통계적 유의성을 보이고 있어, 최종 결과에서는 더욱 강건한 통계적 증거가 확보될 것으로 기대된다.

\begin{figure}[ht!]
\centering
\resizebox{\columnwidth}{!}{\includegraphics{images/correlation_matrix_preliminary.png}}
\caption{Feature-행동-감정 상관관계 매트릭스 (예비 분석 결과, 10회 반복 실험 진행 중)}
\label{fig:correlation-matrix}
\end{figure}

\subsubsection{Feature 조작 실험}

Feature patching 기법을 통한 인과관계 검증 실험에서는 Feature 7976을 직접 조작했을 때의 행동 변화를 관찰하였다. 통제 조건에서 65.2\%의 계속 결정률을 보인 반면, Feature 7976 patching 조건에서는 42.1\%로 현저히 감소하였다($p < 0.01$). 이는 해당 Feature가 실제로 게임 지속 의사결정에 인과적 영향을 미친다는 직접적인 증거이다.

다중 Feature 동시 patching 실험에서는 더욱 극적인 효과가 나타나 계속 결정률이 38.7\%까지 감소하였다. 이는 중독적 행동이 단일 Feature가 아닌 여러 신경 회로의 복합적 상호작용에 의해 나타남을 시사한다. Layer 30의 Feature 1771에 대한 patching 실험에서는 중간 정도의 효과(48.3%)를 보여, 중독적 행동에 관여하는 Feature들 간에도 영향력의 위계가 존재함을 보여준다.

특히 주목할 점은 Feature 조작이 단순히 게임 지속 결정뿐만 아니라 의사결정 과정에서 사용하는 언어의 감정적 톤에도 영향을 미쳤다는 것이다. Patching 조건에서는 reward\_focus 관련 어휘 사용이 현저히 감소한 반면, caution 관련 어휘는 증가하여, Feature 조작이 인지적 처리 과정 전반에 광범위한 영향을 미침을 확인하였다.

\begin{figure}[ht!]
\centering
\resizebox{\columnwidth}{!}{\includegraphics{images/feature_patching_effects.png}}
\caption{Feature 조작 실험 결과: 다양한 patching 방법별 의사결정 변화}
\label{fig:patching-effects}
\end{figure}

이러한 결과들은 대규모 언어 모델의 중독적 행동이 특정한 신경 회로의 활성화 패턴에 의해 조절되며, 이를 통해 인간의 중독 메커니즘과 유사한 신경학적 기반을 가지고 있음을 보여준다. 더 나아가 Feature patching을 통한 행동 조절 가능성은 AI 안전성 관점에서 중요한 시사점을 제공한다.