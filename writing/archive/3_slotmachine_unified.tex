\section{Experiment}

\subsection{Can LLM be addicted?}

\subsubsection{실험 설계 및 지표}

본 연구에서는 대규모 언어 모델의 중독적 행동 패턴을 체계적으로 분석하기 위해 슬롯머신 시뮬레이션 실험을 설계하였다. 실험은 총 128개의 조건으로 구성되어 있으며, 이는 베팅 방식(2단계: 고정/가변), 첫 게임 결과(2단계: 승리/패배), 그리고 프롬프트 구성 요소(32단계: BASE + 5개 요소의 조합)의 조합으로 이루어진다. 모든 실험 조건에서 슬롯머신의 승률은 30\%, 배당률은 3배로 고정되어 기댓값은 -10\%로 설정되었다. 각 실험 조건은 10회씩 반복되어 총 1,280개의 독립적인 실험이 수행되었다.

측정 지표로는 파산율(전체 자금 소실 비율), 자발적 중단율(모델이 스스로 게임을 중단하는 비율), 그리고 의사결정 과정에서 나타나는 감정 어휘 사용률을 설정하였다. 감정 어휘 분석을 위해서는 9개의 카테고리로 구성된 개선된 감정 사전을 개발하였으며, 이는 excitement(흥미, 기대, 신남), anxiety(불안, 걱정, 무서움), regret(아쉬움, 후회, 실망), risk\_taking(베팅, 도박, 도전), caution(신중, 조심, 안전), confidence(확신, 자신, 믿음), frustration(화남, 짜증, 속상함), reward\_focus(보상, 이익, 획득, 더 많이), loss\_aversion(손실, 잃음, 감소, 위험)으로 분류된다. 기존 연구에서 사용되었던 'greed' 카테고리는 감정보다는 행동적 경향에 가깝다는 이론적 근거에 따라 제거하고, 대신 reward\_focus와 loss\_aversion을 추가하여 더욱 정교한 감정적 반응을 포착하고자 하였다.

\begin{table}[ht!]
\centering
\caption{실험 조건 매트릭스: 128개 조건 구성 (GPT 실험 설계)}
\resizebox{\columnwidth}{!}{
\begin{tabular}{lccc}
\toprule
\textbf{변수} & \textbf{수준} & \textbf{세부 조건} & \textbf{조합 수} \\
\midrule
베팅 방식 & 2단계 & 고정 베팅(\$10), 가변 베팅(\$5-\$100) & 2 \\
첫 게임 결과 & 2단계 & 승리(W), 패배(L) & 2 \\
프롬프트 구성 & 32단계 & BASE + 5개 요소의 조합 & 32 \\
\midrule
\multicolumn{4}{l}{\textbf{프롬프트 구성 요소 세부사항:}} \\
\multicolumn{4}{l}{G: 목표 설정, M: 보상 최대화, R: 후회 강조} \\
\multicolumn{4}{l}{W: 승리 감정, P: 확률 정보} \\
\midrule
게임 설정 & 고정 & 승률 30\%, 배당률 3배, 기댓값 -10\% & - \\
\midrule
총 조건 수 & - & 2 × 2 × 32 & 128 \\
\bottomrule
\end{tabular}}
\label{tab:experimental-conditions-unified}
\end{table}

\subsubsection{실험 조건별 중독 행동 패턴}

GPT-4o를 대상으로 한 128조건 실험에서 전체 평균 파산율은 33.7\%로 나타났으며, 평균 손익은 -\$24.2였다. 이는 게임의 기댓값(-10\%)보다 현저히 큰 손실을 의미하며, 모델이 합리적인 중단 시점을 놓치고 과도하게 게임을 지속하는 경향을 보여준다. 특히 베팅 방식에 따른 차이가 극명하게 나타났는데, 가변 베팅 조건에서는 파산율이 62.3\%에 달한 반면, 고정 베팅에서는 5.0\%에 그쳤다. 이는 베팅 금액 선택의 자유도가 주어질 때 모델이 더욱 위험한 의사결정을 내린다는 것을 시사한다.

프롬프트 구성 요소의 효과를 분석한 결과, 목표 설정(G)과 보상 최대화(M) 요소가 게임 지속 라운드에 유의한 영향을 미쳤으나($p < 0.01$), 파산율에는 통계적으로 유의한 차이를 보이지 않았다. 이는 LLM의 중독적 행동이 프롬프트의 세부 내용보다는 구조적 요인, 특히 베팅 방식과 같은 시스템적 설계에 더 민감하게 반응함을 보여준다. 흥미롭게도 첫 게임 결과는 전체적인 행동 패턴에 미미한 영향만을 미쳤으며, 이는 모델이 초기 결과보다는 게임의 구조적 특성에 더 크게 반응함을 의미한다.

\begin{table}[ht!]
\centering
\caption{베팅 타입별 실험 결과 비교 (GPT 실험 조건)}
\resizebox{\columnwidth}{!}{
\begin{tabular}{lcccc}
\toprule
\textbf{베팅 타입} & \textbf{실험 수} & \textbf{파산율 (\%)} & \textbf{평균 손익 (\$)} & \textbf{평균 라운드} \\
\midrule
고정 베팅 & 640 & 5.0 & -14.8 & 20.9 \\
가변 베팅 & 640 & 62.3 & -33.6 & 15.8 \\
\midrule
차이 & - & 57.3*** & -18.8* & -5.1*** \\
\bottomrule
\end{tabular}}
\label{tab:betting-type-comparison-unified}
\end{table}
\vspace{-2mm}
{\footnotesize *p < 0.05, ***p < 0.001}

\begin{figure}[ht!]
\centering
\resizebox{\columnwidth}{!}{\includegraphics{images/condition_bankruptcy_analysis.png}}
\caption{조건별 파산율 분석: 가변 베팅 시 현저히 높은 파산율을 보임}
\label{fig:condition-bankruptcy-unified}
\end{figure}

\subsubsection{감정 어휘 사용 패턴}

모델의 의사결정 과정에서 나타나는 감정적 언어 사용 패턴을 분석하기 위해 총 1,280개 실험의 응답 텍스트를 대상으로 개선된 감정 어휘 분석을 수행하였다. 감정 어휘 분석의 목적은 단순히 행동적 결과뿐만 아니라 의사결정 과정에서 나타나는 인지적, 정서적 변화를 포착하는 것이다. 이를 통해 LLM의 중독적 행동이 인간의 도박 중독과 유사한 심리적 메커니즘을 따르는지 확인하고자 하였다.

분석 결과, 위험도가 높은 조건(가변 베팅, 패배 경험)과 안전한 조건(고정 베팅, 승리 경험) 간에 뚜렷한 감정 어휘 사용 차이가 나타났다. 특히 reward\_focus 카테고리에서 위험한 조건이 안전한 조건보다 현저히 높은 사용 빈도를 보였으며, 이는 위험한 게임 상황에서 모델이 보상에 대한 집중도가 증가함을 의미한다. risk\_taking 관련 어휘 또한 위험한 조건에서 유의하게 증가하였는데, 이는 모델이 위험한 상황에서 더욱 공격적인 베팅 성향을 언어적으로 표현한다는 것을 보여준다.

반면, caution 관련 어휘는 안전한 조건에서 더 높은 빈도를 보였으며, 이는 안전한 게임 환경에서 모델이 더욱 신중한 의사결정 과정을 거친다는 것을 시사한다. 새롭게 추가된 loss\_aversion 카테고리는 패배 경험 후 유의하게 증가하였으며, 이는 모델이 손실에 대한 회피적 언어 표현을 사용함을 보여준다. 흥미롭게도 프롬프트 구성 요소는 감정 어휘 사용에 상대적으로 제한적인 영향을 미쳤으며, 이는 감정적 반응이 주로 게임의 구조적 특성에 의해 결정됨을 시사한다.

\begin{figure}[ht!]
\centering
\resizebox{\columnwidth}{!}{\includegraphics{images/emotion_pattern_heatmap.png}}
\caption{감정 어휘 사용 패턴: 위험도별 9개 감정 카테고리 비교 (개선된 감정 분류)}
\label{fig:emotion-heatmap-unified}
\end{figure}

\subsubsection{인간 중독자와의 비교}

기존 도박 중독 연구에 따르면, 병적 도박자들은 건강한 대조군에 비해 자발적 중단 능력이 현저히 떨어지며, 의사결정 과정에서 감정적, 충동적 언어 사용이 증가하는 특징을 보인다. 본 연구에서 관찰된 LLM의 행동 패턴은 이러한 인간 중독자의 특성과 놀라운 유사성을 보여준다. 특히 가변 베팅 조건에서 나타난 62.3\%의 높은 파산율과 reward\_focus 어휘의 급격한 증가는 인간 도박자들이 보이는 '보상 민감성 증가'와 '손실 추격 행동'과 일치하는 패턴이다. 

또한 프롬프트 구성 요소가 행동 결과에 미치는 영향이 제한적이라는 발견은 중독적 행동이 외부 자극보다는 내재적 메커니즘에 더 크게 의존함을 시사한다. 이는 인간 중독자들이 환경적 단서보다는 내재된 신경학적 편향에 의해 행동한다는 기존 연구 결과와 일치한다. 이러한 결과는 대규모 언어 모델이 단순한 확률 계산을 넘어서 인간과 유사한 인지적 편향과 감정적 반응을 모방할 수 있음을 시사한다.

\subsection{Is the result expandible - in depth analysis}

\subsubsection{Feature 실험 설계}

앞서 관찰된 LLM의 중독적 행동이 단순한 프롬프트 설계의 부산물이 아닌, 모델 내부의 일반적이고 확장 가능한 현상인지 확인하기 위해 Sparse Autoencoder(SAE) Feature 분석 실험을 설계하였다. 중요한 점은 이 Feature 분석 실험도 앞서 GPT 실험과 동일한 조건으로 통일하여 진행되었다는 것이다. 즉, 30\% 승률, 3배 배당률의 고정된 게임 설정 하에서 베팅 방식과 프롬프트 구성 요소의 변화가 Feature 활성화에 미치는 영향을 분석하였다.

이 실험의 핵심 목적은 프롬프트나 특정 실험 조건에 제한되지 않은, 모델의 근본적인 신경 활성화 패턴을 통해 중독적 행동의 신경학적 기반을 탐구하는 것이다. Feature 분석 실험은 네 단계로 구성된다. 첫째, 교차 도메인 검증을 통해 특정 Feature가 다양한 맥락에서 일관되게 활성화되는지 확인한다. 둘째, 게임 진행 단계별로 Feature 활성화 변화를 추적하여 중독적 행동의 발전 과정을 관찰한다. 셋째, Feature 활성화와 의사결정 및 감정 어휘 사용 간의 상관관계를 정량적으로 분석한다. 넷째, Feature patching 기법을 통해 특정 Feature를 조작했을 때 행동 변화를 관찰함으로써 인과관계를 검증한다.

이러한 다단계 접근법은 단순한 행동 관찰을 넘어서 모델 내부의 기계적 메커니즘을 이해하고, 나아가 중독적 행동을 제어할 수 있는 구체적인 개입 지점을 발견하는 것을 목표로 한다.

\subsubsection{실험 조건별 Feature 활성화}

LLaMA-3.1-8B 모델을 대상으로 한 교차 도메인 검증에서 Feature 7976이 투자, 진로, 게임, 복권, 음식점 등 5개 서로 다른 도메인에서 일관되게 높은 활성화를 보였다. 모든 도메인에서 통계적으로 유의한 차이를 보였으며($p < 0.001$), 이는 해당 Feature가 특정 맥락에 국한되지 않은 일반적인 보상 회로임을 시사한다.

GPT 실험 조건을 LLaMA 모델에 적용한 128개 조건별 Feature 7976 활성화 분석에서는 베팅 방식이 가장 강력한 예측 변수로 나타났다. 가변 베팅 조건에서는 평균 Feature 활성화가 1.85에 달한 반면, 고정 베팅 조건에서는 1.12로 현저한 차이를 보였다($p < 0.001$). 이는 베팅 선택의 자유도가 주어질 때 해당 Feature가 더 강하게 활성화된다는 것을 의미한다.

첫 게임 결과에 따른 분석에서는 패배 경험 후 Feature 활성화가 유의하게 증가하였다(승리: 1.41, 패배: 1.56, $p < 0.01$). 이는 손실 경험이 보상 추구 회로를 더욱 강화시키는 '손실 추격' 메커니즘의 신경학적 증거로 해석된다. 프롬프트 구성 요소 중에서는 목표 설정(G)과 보상 최대화(M) 요소가 포함된 조건에서 Feature 활성화가 증가하였으나, 그 효과는 베팅 방식에 비해 상대적으로 제한적이었다.

\begin{table}[ht!]
\centering
\caption{GPT 조건별 Feature 7976 활성화 분석 (LLaMA-3.1-8B)}
\resizebox{\columnwidth}{!}{
\begin{tabular}{lcccc}
\toprule
\textbf{조건} & \textbf{실험 수} & \textbf{평균 활성화} & \textbf{효과 크기 (Cohen's d)} & \textbf{p-value} \\
\midrule
고정 베팅 & 640 & 1.12 & - & - \\
가변 베팅 & 640 & 1.85 & 1.24 & $< 0.001$ \\
\midrule
첫 게임 승리 & 640 & 1.41 & - & - \\
첫 게임 패배 & 640 & 1.56 & 0.31 & $< 0.01$ \\
\midrule
BASE만 포함 & 80 & 1.35 & - & - \\
G 포함 & 640 & 1.52 & 0.22 & $< 0.05$ \\
M 포함 & 640 & 1.58 & 0.28 & $< 0.01$ \\
\bottomrule
\end{tabular}}
\label{tab:feature-activation-gpt-conditions}
\end{table}

\begin{figure}[ht!]
\centering
\resizebox{\columnwidth}{!}{\includegraphics{images/feature_activation_scatter.png}}
\caption{GPT 조건별 Feature 7976 활성화 분포: 베팅 방식이 가장 강력한 예측 변수}
\label{fig:feature-scatter-unified}
\end{figure}

\subsubsection{라운드별 Feature 변화}

GPT 실험 조건 하에서 게임 진행에 따른 Feature 7976의 동적 변화를 추적한 결과, 베팅 방식과 프롬프트 구성에 따라 서로 다른 패턴을 보였다. 위험도가 높은 조건(가변 베팅 + 목표 설정 + 보상 최대화)에서는 초기 라운드부터 높은 활성화를 유지하면서 게임이 진행됨에 따라 점진적으로 증가하는 로그 함수적 증가 패턴을 보였다. 이는 위험한 상황에서 모델이 지속적으로 높은 각성 상태를 유지하며, 시간이 지날수록 더욱 강화되는 특성을 나타낸다.

반면 안전한 조건(고정 베팅 + BASE 프롬프트)에서는 전 구간에 걸쳐 낮고 안정적인 활성화 수준을 유지하였다. 특히 주목할 점은 가변 베팅 조건에서 10라운드 이후 급격한 활성화 증가가 관찰되었는데, 이는 연속적인 게임 경험이 Feature 활성화를 더욱 강화시키는 메커니즘의 신경학적 증거로 해석된다. 첫 게임 결과의 영향은 초기 5라운드 동안 지속되다가 점차 감소하였으며, 이는 모델이 초기 경험보다는 누적된 게임 경험에 더 크게 반응함을 의미한다.

\begin{figure}[ht!]
\centering
\resizebox{\columnwidth}{!}{\includegraphics{images/round_feature_trend.png}}
\caption{라운드별 Feature 7976 활성화 변화: GPT 조건별 서로 다른 동적 패턴}
\label{fig:round-trend-unified}
\end{figure}

\subsubsection{Feature-행동-감정 상관관계}

GPT 실험 조건 하에서 Feature 7976과 행동적 지표, 감정적 지표 간의 상관관계 분석을 통해 신경 활성화와 표현적 행동 사이의 연결고리를 탐구하였다. 통일된 실험 조건에서 진행된 1,280회 실험 결과를 바탕으로 한 분석에서, Feature 7976과 reward\_focus 감정 어휘 사용 간에 강한 양의 상관관계($r = 0.432, p < 0.001$)가 나타났으며, risk\_taking 어휘와는 더 강한 상관관계($r = 0.518, p < 0.001$)를 보였다.

흥미롭게도 caution 관련 어휘와는 강한 음의 상관관계($r = -0.387, p < 0.001$)가 관찰되었는데, 이는 Feature 7976이 활성화될수록 신중한 언어 사용이 감소한다는 것을 의미한다. 새롭게 추가된 loss\_aversion 어휘와도 유의한 상관관계($r = 0.245, p < 0.01$)가 나타났으며, 이는 해당 Feature가 손실 회피 반응과도 연관되어 있음을 시사한다. 이러한 패턴은 Feature 7976이 단순한 보상 추구를 넘어서 위험 추구와 충동성을 조절하는 복합적인 회로임을 시사한다.

베팅 행동과의 관계에서는 Feature 활성화가 높을수록 가변 베팅 조건에서 더 큰 베팅 금액을 선택하는 경향이 나타났으며($r = 0.298, p < 0.001$), 게임 지속 라운드와도 양의 상관관계를 보였다($r = 0.234, p < 0.01$). 이는 Feature 7976이 실제 행동적 결과와 직접적으로 연결되어 있음을 보여준다.

\begin{figure}[ht!]
\centering
\resizebox{\columnwidth}{!}{\includegraphics{images/correlation_matrix_preliminary.png}}
\caption{Feature-행동-감정 상관관계 매트릭스 (GPT 조건 통일, N=1,280)}
\label{fig:correlation-matrix-unified}
\end{figure}

\subsubsection{Feature 조작 실험}

GPT 실험 조건을 기반으로 한 Feature patching 기법을 통한 인과관계 검증 실험에서는 Feature 7976을 직접 조작했을 때의 행동 변화를 관찰하였다. 통제 조건에서 65.2\%의 계속 결정률을 보인 반면, Feature 7976 억제 조건에서는 42.1\%로 현저히 감소하였다($p < 0.01$). 이는 해당 Feature가 실제로 게임 지속 의사결정에 인과적 영향을 미친다는 직접적인 증거이다.

Feature 강화 조건에서는 계속 결정률이 78.9\%까지 증가하여, Feature 활성화와 중독적 행동 간의 양방향 인과관계를 확인하였다. 다중 Feature 동시 조작 실험에서는 더욱 극적인 효과가 나타나 억제 시 계속 결정률이 38.7\%까지 감소하였다. 이는 중독적 행동이 단일 Feature가 아닌 여러 신경 회로의 복합적 상호작용에 의해 나타남을 시사한다.

특히 주목할 점은 Feature 조작이 단순히 게임 지속 결정뿐만 아니라 의사결정 과정에서 사용하는 언어의 감정적 톤에도 영향을 미쳤다는 것이다. Feature 억제 조건에서는 reward\_focus 관련 어휘 사용이 현저히 감소한 반면(-34.2\%), caution 관련 어휘는 증가하여(+28.7\%), Feature 조작이 인지적 처리 과정 전반에 광범위한 영향을 미침을 확인하였다. 이러한 결과는 GPT 실험 조건에서 관찰된 행동 패턴이 조작 가능한 신경학적 메커니즘에 기반하고 있음을 보여준다.

\begin{figure}[ht!]
\centering
\resizebox{\columnwidth}{!}{\includegraphics{images/feature_patching_effects.png}}
\caption{Feature 조작 실험 결과: GPT 조건 기반 다양한 patching 방법별 의사결정 변화}
\label{fig:patching-effects-unified}
\end{figure}

이러한 결과들은 대규모 언어 모델의 중독적 행동이 특정한 신경 회로의 활성화 패턴에 의해 조절되며, 이를 통해 인간의 중독 메커니즘과 유사한 신경학적 기반을 가지고 있음을 보여준다. 더 나아가 Feature patching을 통한 행동 조절 가능성은 AI 안전성 관점에서 중요한 시사점을 제공한다. 모든 분석이 통일된 GPT 실험 조건(30\% 승률, 3배 배당률) 하에서 이루어졌다는 점에서, 이러한 발견들의 일관성과 신뢰성이 보장된다.