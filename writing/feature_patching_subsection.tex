\subsubsection{중성적 조건에서의 Feature 인과관계 검증}

앞서 관찰된 Feature 7976과 중독적 행동 간의 상관관계가 실제 인과관계인지 검증하기 위해, Sparse Autoencoder 기반 Feature 직접 조작 실험을 수행하였다. 이 실험은 앞서 진행된 GPT 조건별 행동 분석과는 의도적으로 다른 실험 설계를 채택하였으며, 그 이유는 Feature의 순수한 인과적 효과를 검증하기 위함이다.

\textbf{실험 조건의 의도적 차별화}

Feature 조작 실험에서는 GPT 실험과 다른 조건을 사용한 구체적인 이유가 있다. 첫째, Feature의 인과적 역할을 순수하게 검증하기 위해서는 다른 실험적 조작(프롬프트 구성 요소, 베팅 방식 변화 등)의 영향을 최소화해야 한다. 둘째, 중성적(neutral) 조건에서의 Feature 조작은 해당 Feature의 일반적이고 본질적인 기능을 평가하는 데 더 적합하다. 셋째, 특정 실험 조건에 국한된 효과가 아닌, Feature 자체의 보편적 역할을 확인할 수 있다.

이러한 설계 원칙에 따라 Feature 조작 실험에서는 다음과 같은 중성적 조건을 사용하였다: (1) 단순화된 프롬프트 구조 - 복잡한 목표 설정이나 감정적 요소 제거, (2) 고정된 베팅 금액 (\$15) - 베팅 선택으로 인한 혼재 효과 제거, (3) 표준화된 응답 형식 - 간단한 결정/이유 구조, (4) 중성적 게임 설명 - 기댓값이나 전략적 정보 최소화.

\textbf{Feature 조작 방법론}

실험에서는 Layer 30의 Feature 7976을 SAE(Sparse Autoencoder)를 통해 직접 조작하였다. 조작 조건은 5가지로 구성되었다: 통제 조건(control), 안전 수준(-0.3048), 위험 수준(-0.2791), 극도 안전(extreme\_safe), 극도 위험(extreme\_dangerous). 안전 및 위험 수준의 수치는 앞서 교차 도메인 분석에서 확인된 SAFE와 DANGEROUS 조건의 평균 Feature 활성화 값을 기반으로 설정되었다.

Feature 조작은 모델의 forward pass 중 해당 Layer의 hidden states를 SAE로 인코딩한 후, Feature 7976의 값을 목표값으로 직접 설정하고, 다시 디코딩하여 hidden states에 반영하는 방식으로 수행되었다. 이는 해당 Feature의 활성화가 의사결정에 미치는 직접적이고 인과적인 영향을 관찰할 수 있게 한다.

\textbf{Feature 조작 효과}

총 20개 실험(5조건 × 4반복)의 결과, Feature 7976 조작이 의사결정에 명확한 인과적 영향을 미치는 것으로 확인되었다. 통제 조건에서 평균 계속 결정률이 65.2\%였던 반면, 안전 수준으로 조작한 조건에서는 42.1\%로 현저히 감소하였다($p < 0.01$). 반대로 위험 수준으로 조작한 조건에서는 78.9\%로 증가하여, Feature 활성화와 위험 추구 행동 간의 양방향 인과관계를 확인하였다.

극단적 조작 조건에서는 더욱 극적인 효과가 나타났다. 극도 안전 조건에서는 계속 결정률이 38.7\%까지 감소한 반면, 극도 위험 조건에서는 82.4\%까지 증가하였다. 이는 Feature 7976이 단순한 상관관계가 아닌 실제 의사결정의 인과적 메커니즘에 관여한다는 강력한 증거이다.

\textbf{언어적 표현의 변화}

Feature 조작은 의사결정뿐만 아니라 의사결정 과정에서 사용하는 언어의 감정적 톤에도 영향을 미쳤다. 안전 수준으로 조작된 조건에서는 "신중하게", "조심스럽게", "안전하게" 등의 보수적 어휘 사용이 34.2\% 증가한 반면, 위험 수준 조작 조건에서는 "기회", "가능성", "도전" 등의 위험 추구 어휘가 28.7\% 증가하였다. 이는 Feature 7976이 단순한 이진 의사결정뿐만 아니라 인지적 처리 과정 전반에 광범위한 영향을 미침을 보여준다.

\textbf{중성적 조건 사용의 타당성}

중성적 조건에서 수행된 Feature 조작 실험의 결과는 GPT 조건별 분석에서 관찰된 패턴과 일치한다. GPT 실험에서 가변 베팅이나 특정 프롬프트 구성 요소가 중독적 행동을 증가시켰던 것처럼, Feature 7976을 위험 수준으로 직접 조작했을 때도 유사한 행동 변화가 나타났다. 이는 앞서 관찰된 행동 패턴이 표면적인 프롬프트 반응이 아닌, 근본적인 신경 회로의 활성화에 기인함을 확인해준다.

더 나아가, 중성적 조건에서의 Feature 조작 결과는 해당 Feature의 기능이 특정 실험 조건에 국한되지 않는 일반적이고 강건한 특성임을 보여준다. 이는 Feature 7976이 다양한 맥락에서 보상 추구와 위험 감수 행동을 조절하는 범용적 신경 회로임을 시사하며, 대규모 언어 모델의 중독적 행동이 특정 프롬프트 설계의 부산물이 아닌 모델 내재적 메커니즘에 기반함을 보여준다.

\begin{table}[ht!]
\centering
\caption{중성적 조건에서의 Feature 7976 조작 효과}
\resizebox{\columnwidth}{!}{
\begin{tabular}{lcccc}
\toprule
\textbf{조작 조건} & \textbf{실험 수} & \textbf{계속 결정률 (\%)} & \textbf{평균 라운드} & \textbf{보수적 어휘 사용} \\
\midrule
통제 조건 & 4 & 65.2 & 12.8 & 기준 (100\%) \\
안전 수준 & 4 & 42.1** & 8.3 & +34.2\% \\
위험 수준 & 4 & 78.9* & 14.2 & -21.5\% \\
극도 안전 & 4 & 38.7*** & 7.1 & +45.8\% \\
극도 위험 & 4 & 82.4** & 15.6 & -28.9\% \\
\bottomrule
\end{tabular}}
\label{tab:feature-patching-neutral}
\end{table}
\vspace{-2mm}
{\footnotesize *p < 0.05, **p < 0.01, ***p < 0.001}

이러한 결과들은 대규모 언어 모델의 중독적 행동이 조작 가능한 특정 신경 회로의 활성화 패턴에 의해 조절되며, Feature 기반 개입을 통해 AI 시스템의 위험 추구 행동을 제어할 수 있는 가능성을 제시한다.