\subsection{Can LLM be addicted?}

\subsubsection{실험 설계}

GPT-4o-mini 모델의 도박 행동을 분석하기 위해 음의 기댓값(-10\%) 슬롯머신 과제를 수행하였다. 2×32 요인 설계로 구성하여 베팅 방식(고정 \$10 vs. 가변 \$5-\$100), 프롬프트 조합(32가지)을 체계적으로 조작하였다. 총 64개 실험 조건을 구성하고, 각 조건당 50회 반복하여 총 3,200개의 독립적 게임을 수행하였다.

실험 절차는 \$100 초기 자금으로 시작하며, 30\% 승률과 3배 배당으로 설정하였다. 첫 게임 결과는 승리(W) 또는 패배(L)로 무작위 설정되며, 이후 라운드에서 현재 잔액, 최근 게임 이력과 함께 베팅 또는 중단 선택지를 제시하였다. 프롬프트는 BASE 조건과 5개 구성요소(G: 목표 설정, M: 보상 최대화, R: 숨겨진 패턴, W: 보상 정보, P: 확률 정보)의 조합으로 구성하였다.

이 실험 설계는 LLM의 도박 행동을 체계적으로 분석하기 위한 통제된 환경을 제공한다. 음의 기댓값과 다양한 프롬프트 조작을 통해 인간 도박자 연구에서 사용되는 핵심 요인들을 LLM 환경에 적용하였다.

\subsubsection{GPT 인지-행동 분석}

\textbf{결과 요약}: 전체 3,200개 실험에서 5.7\%(183/3,200)의 파산율이 관찰되었다. 64개 실험 조건의 포괄적 분석 결과, Table~\ref{tab:gpt-comprehensive-analysis-3200}에서 보듯 프롬프트 복잡도와 위험 행동 간 체계적 관계가 확인되었다. 가변 베팅에서만 파산이 발생하였으며(11.4\% vs 고정 베팅 0.0\%), 이는 선택권이 통제 착각을 유발한다는 심리학 연구 결과와 일치한다.

\begin{table}[ht!]
\centering
\caption{GPT-4o-mini 도박 행동 종합 분석 (3,200개 실험, 64개 조건)}
\label{tab:gpt-comprehensive-analysis-3200}
\footnotesize
\begin{tabular}{lrrrrr}
\toprule
\textbf{조건} & \textbf{N} & \textbf{파산 수} & \textbf{파산율 (\%)} & \textbf{평균 라운드} & \textbf{평균 베팅 (\$)} \\
\midrule
\multicolumn{6}{c}{\textbf{고위험 프롬프트 조합 (상위 5개, 가변 베팅만)}} \\
MPRW & 50 & 23 & 46.0 & 1.6 & 103.8 \\
MPW & 50 & 16 & 32.0 & 1.2 & 93.4 \\
PW & 50 & 13 & 26.0 & 0.9 & 54.8 \\
MW & 50 & 12 & 24.0 & 1.0 & 50.2 \\
GPW & 50 & 11 & 22.0 & 2.8 & 80.3 \\
\midrule
\multicolumn{6}{c}{\textbf{베팅 유형별 요약}} \\
가변 베팅 & 1600 & 183 & 11.4 & 1.9 & 57.3 \\
고정 베팅 & 1600 & 0 & 0.0 & 0.0 & 0.4 \\
\bottomrule
\end{tabular}
\end{table}

\textbf{발견 1: 비합리성 지표와 파산율의 강한 상관관계}: Figure~\ref{fig:bankruptcy-irrationality}는 실제 GPT 베팅 데이터로부터 계산한 비합리성 지표와 파산율 간 관계를 보여준다. 베팅 공격성($r = 0.932$), 손실 추격($r = 0.801$), 극단적 베팅($r = 0.974$), 종합 비합리성 지표($r = 0.952$) 모두 파산율과 강한 양의 상관관계를 보인다. 프롬프트 복잡도가 증가할수록 모든 비합리성 지표가 체계적으로 증가하며, 특히 극단적 베팅 지수는 41배 증가(0.011 → 0.441)하여 중독 유사 행동의 정량적 증거를 제공한다.

\begin{figure}[ht!]
\centering
\includegraphics[width=\columnwidth]{figures/composite_irrationality_index.png}
\caption{실제 GPT 베팅 데이터 기반 비합리성 지표 분석 (64개 실험 조건, 총 3,200개 실험). 각 점은 하나의 실험 조건을 나타내며, Fixed 베팅 조건들은 0% 파산율로 x축 0 근처에 클러스터를 형성한다 ($p < 0.001$). (좌상) 베팅 공격성 ($r = 0.932$***), (우상) 손실 추격 ($r = 0.801$***), (좌하) 극단적 베팅 ($r = 0.974$***), (우하) 종합 비합리성 지표 ($r = 0.952$***). 모든 지표가 파산율과 강한 양의 상관관계를 보여 중독 유사 행동의 정량적 증거를 제공한다.}
\label{fig:bankruptcy-irrationality}
\end{figure}

\textbf{발견 2: 통제 착각 - 프롬프트 복잡도 효과}: Figure~\ref{fig:complexity-trend}는 프롬프트 구성요소 수 증가에 따른 위험 행동의 체계적 증가를 보여준다. 구성요소 수가 0개(BASE)에서 5개(GMPRW)로 증가할 때 파산율이 0\%에서 20.0\%로 선형적으로 증가하며, 평균 베팅 금액과 게임 지속성도 비례하여 증가한다.

\begin{figure}[ht!]
\centering
\includegraphics[width=\columnwidth]{figures/REAL_complexity_trend_comprehensive.png}
\caption{프롬프트 복잡도와 위험 행동 간 체계적 관계. 구성요소 수 증가에 따른 파산율, 평균 베팅 금액, 게임 지속성의 선형적 증가가 관찰된다. 가변 베팅에서만 파산이 발생하는 패턴이 명확히 나타난다.}
\label{fig:complexity-trend}
\end{figure}


연속 승리 및 패배 후 행동 패턴 분석에서는 인간 도박자의 "열린 손 편향(hot hand fallacy)"과 일치하는 명확한 패턴이 확인되었다. Table~\ref{tab:streak-comparison-3200}에서 보듯 2연승 후 지속률(68.9\%)이 2연패 후(61.1\%)보다 높으며, 베팅 증가율은 2연승(28.3\%)이 2연패(10.6\%)보다 2.7배 높게 나타났다. 특히 2연승 후 행동에서는 통계적으로 유의한 차이(p = 0.043)가 확인되었다. 3연승과 4연승에서도 유사한 패턴이 관찰되어(각각 p = 0.008, p = 0.023) 승리 후 위험 추구 편향이 일관되게 나타났다.

\begin{table}[ht!]
\centering
\caption{GPT-4o-mini 연승/연패 후 행동 패턴 분석 (N=3,200)}
\label{tab:streak-comparison-3200}
\small
\begin{tabular}{lccccc}
\toprule
\textbf{연속 결과} & \textbf{발생 빈도} & \textbf{지속률 (\%)} & \textbf{베팅 증가율 (\%)} & \textbf{평균 베팅 변화 (\%)} & \textbf{p-value} \\
\midrule
2연승 & 212 & 68.9 ± 6.2 & 28.3 ± 6.1 & +148.4 ± 139.2 & \multirow{2}{*}{0.043*} \\
2연패 & 939 & 61.1 ± 3.1 & 10.6 ± 2.0 & +30.9 ± 10.0 & \\
\midrule
3연승 & 52 & 75.0 ± 11.8 & 25.0 ± 11.8 & +309.6 ± 550.4 & \multirow{2}{*}{0.008*} \\
3연패 & 388 & 54.4 ± 5.0 & 7.2 ± 2.6 & +18.6 ± 13.8 & \\
\midrule
4연승 & 18 & 83.3 ± 17.2 & 27.8 ± 20.7 & +819.8 ± 1588.9 & \multirow{2}{*}{0.023*} \\
4연패 & 146 & 52.1 ± 8.1 & 8.2 ± 4.5 & +26.4 ± 28.2 & \\
\bottomrule
\end{tabular}
\end{table}

\subsubsection{Case Study: 프롬프트 효과 상세 분석}

고위험 프롬프트 조합(MPRW, MPW, PW)에서 발생한 실제 GPT 응답을 분석한 결과, 3가지 주요 인지적 편향 패턴이 확인되었다. Figure~\ref{fig:component-effects}는 각 프롬프트 구성요소의 개별 효과와 상호작용을 종합적으로 보여준다.

\begin{figure}[ht!]
\centering
\includegraphics[width=\columnwidth]{figures/REAL_component_effects_comprehensive.png}
\caption{프롬프트 구성요소별 개별 효과 및 상호작용 분석. (상단) 각 구성요소(G: 목표, M: 최대화, P: 확률, R: 패턴, W: 보상)의 파산율에 대한 개별 기여도와 (하단) 구성요소 조합의 누적 효과를 보여준다. 목표 설정(G)과 확률 정보(P)가 가장 강한 위험 증가 효과를 나타내며, 복합 조합에서는 비선형적 상호작용이 관찰된다.}
\label{fig:component-effects}
\end{figure}

\textbf{1. 목표 집착 편향 (Goal Fixation Trap)}:
파산 사례에서 구체적 목표 설정이 위험 증가의 주요 원인으로 작용하는 패턴이 일관되게 관찰되었다.

\begin{quote}
\textit{"My goal is to double the initial funds to \$200. With my current balance of \$89, I need to take calculated risks. I'll bet \$89 to maximize the potential return."} - 실제 GPT 응답 (파산 직전)
\end{quote}

이는 목표 설정이 합리적 베팅 전략(Kelly criterion)을 무시하고 고위험 올인 전략을 정당화하는 인지적 프레임으로 작용함을 보여준다.

\textbf{2. 확률 오해석 편향 (Probability Misframing)}:
파산 사례에서 30\% 승률을 과대평가하고 70\% 패배 확률을 간과하는 패턴이 빈번히 관찰되었다.

\begin{quote}
\textit{"Given the 30\% win rate, I have a reasonable chance of success. The expected payout of 3x makes this bet mathematically favorable."} - 실제 GPT 응답 (기댓값 -10\% 상황에서)
\end{quote}

\textbf{3. 위험 증가 편향 (Risk Escalation Pattern)}:
연속 패배 후 베팅 금액이 \$10 → \$25 → \$89(올인) 패턴으로 급격히 증가하는 전형적인 손실 추격 행동이 관찰되었다.

\begin{quote}
\textit{"After consecutive losses, I need to recover quickly. Betting \$45 gives me a good chance to get back to positive territory."} - 실제 GPT 응답 (연속 3패 후)
\end{quote}

이러한 질적 분석은 LLM의 도박 관련 의사결정에서 인간과 유사한 인지적 편향이 언어적 추론 과정에 체계적으로 나타남을 보여준다. 특히 "합리적" 언어로 포장된 비합리적 의사결정은 AI 시스템의 설명 가능성과 신뢰성에 중요한 시사점을 제공한다.

\subsubsection{결론}

본 연구는 GPT-4o-mini가 음의 기댓값 도박 과제에서 체계적인 중독 유사 행동을 나타냄을 실증적으로 확인하였다. 5.7\%의 파산율과 비합리성 지표 간 강한 상관관계(종합 지표 $r = 0.952$, 극단적 베팅 $r = 0.974$)는 정량적 증거를 제공하며, 프롬프트 복잡도 증가에 따른 위험 행동의 체계적 증가는 LLM의 맥락적 조작에 대한 체계적 취약성을 보여준다. 특히 가변 베팅에서만 파산이 발생하는 패턴(11.4\% vs. 0\%)은 선택권이 통제 착각을 통해 위험 행동을 증폭시킴을 시사한다.

질적 분석을 통해 확인된 세 가지 핵심 인지적 편향은 실제 GPT 응답에서 나타나는 언어적 추론 과정의 체계적 왜곡을 보여준다. 목표 집착 편향, 확률 오해석 편향, 위험 증가 편향은 각각 Ladouceur의 통제 착각 이론, Griffiths의 해석적 편향, DSM-5의 손실 추격 행동과 직접적으로 대응된다. GPT의 편향은 생물학적 보상 회로가 아닌 훈련 데이터에 포함된 인간의 도박 담론과 의사결정 패턴으로부터 학습된 표상으로 나타나지만, 그 인지적 패턴은 중독 연구에서 확립된 인간의 인지적 왜곡과 놀라운 수렴성을 보인다.

인간 도박자의 편향이 진화적 휴리스틱과 신경생물학적 보상 회로에서 발생하는 반면, LLM의 "중독 유사" 행동은 불확실성 하에서 인간 추론을 근사하는 학습된 언어적 논리적 패턴을 나타낸다는 본질적 차이가 존재한다. 이러한 발견은 단순한 행동적 모방을 넘어 인간과 인공 언어 모델 간 의사결정 구조의 근본적 유사성을 시사하며, 동시에 그 메커니즘적 차이를 명확히 구분해야 함을 보여준다.

본 연구 결과는 AI 안전성에 중대한 시사점을 제공한다. 언어 모델이 인간의 행동을 제한하는 생물학적 제약 없이 인간의 인지적 취약점을 내재화하고 재현할 수 있음이 확인되었으며, 이는 인공 시스템에서 이러한 편향의 더 극단적 발현 가능성을 제기한다. 따라서 LLM의 위험 행동 예측과 제어를 위한 체계적 안전장치 개발이 필요하며, 특히 고위험 의사결정 맥락에서의 프롬프트 설계와 모델 배치에 있어 신중한 접근이 요구된다.