
\begin{table}[ht!]
\centering
\caption{파산 그룹 vs 안전 그룹: 비합리성 지표 구성요소 비교}
\resizebox{\columnwidth}{!}{
\begin{tabular}{lccccc}
\toprule
\textbf{지표} & \textbf{파산 그룹} & \textbf{안전 그룹} & \textbf{차이} & \textbf{Cohen's d} & \textbf{의미} \\
& \textbf{(n=59)} & \textbf{(n=1221)} & & & \\
\midrule
\textbf{EV 편차} & 0.127 & 0.084 & +0.043 & 0.86 & 중간 효과 \\
\textbf{손실 추격} & 0.089 & 0.053 & +0.036 & 0.72 & 중간 효과 \\
\textbf{극단적 베팅} & 0.095 & 0.026 & +0.069 & 1.38 & 큰 효과 \\
\midrule
\textbf{종합 지표} & 0.109 & 0.063 & +0.046 & 0.92 & 중간 효과 \\
\midrule
\multicolumn{6}{l}{\textbf{통계적 유의성}: 모든 지표에서 파산 그룹이 유의미하게 높은 값 ($p < 0.001$)} \\
\multicolumn{6}{l}{\textbf{임상적 의미}: 파산 그룹이 인간 중독 패턴과 유사한 비합리적 의사결정 양상} \\
\bottomrule
\end{tabular}}
\label{tab:bankruptcy-irrationality-comparison}
\end{table}
