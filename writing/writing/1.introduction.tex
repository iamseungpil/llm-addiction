\section{Introduction}
\label{Introduction}

This research began with a single question: Can LLMs also fall into addiction? This leads to several other questions. These include what it means for an LLM to be addicted, how the phenomenon of addiction would affect decision-making, and what must be done to prevent it. As LLMs become more sophisticated and attempt to utilize LLM agents for tasks such as asset management and product import/sales increase~\citep{luo2025llm, ding2024large, yu2024fincon}, the question of whether LLMs can make pathological decisions in certain situations is gaining importance.

% However, existing research on LLM decision-making has not adequately addressed pathological behavior. Some studies explore the behavioral tendencies of LLMs~\citep{keeling2024can, jia2024decision, wu2025exploring}, but they assume rationality and consequently do not sufficiently examine flawed decision-making. Other studies analyze irrational decision-making in LLMs~\citep{skalse2022defining, denison2024sycophancy, chen2024odin}, yet these works primarily focus on mitigating problematic behaviors through training interventions--such as curriculum design, reward model refinement, or retraining strategies--with limited investigation into the underlying representational mechanisms or behavioral motivations.

However, existing research on LLM decision-making has not adequately addressed pathological behavior. While some studies explore behavioral tendencies of LLMs~\citep{keeling2024can, jia2024decision, wu2025exploring}, they assume rationality and do not sufficiently examine flawed decision-making. Others analyze irrational decision-making~\citep{skalse2022defining, denison2024sycophancy, chen2024odin} or \blue{incorporate psychological frameworks~\citep{du2025mitigating}}, yet these works primarily focus on mitigating problematic behaviors through training interventions--such as curriculum design, reward model refinement, or retraining strategies--with limited investigation into the underlying representational mechanisms or behavioral motivations.

This study analyzed LLM addiction phenomena by integrating human addiction research and LLM behavioral analysis, as outlined in Figure~\ref{fig:experimental-overview}. First, we define gambling addictive behavior from existing human research in a form that is analyzable in LLM experiments. Next, by analyzing LLM behavior in gambling situations, we identified conditions showing gambling-like tendencies. Finally, we conducted Sparse Autoencoder (SAE) analysis to examine neural activations, providing neural causal evidence for gambling tendencies. This approach is grounded in cognitive psychology theories such as Cognitive Distortion Theory~\citep{beck1963thinking, franceschi2007complements}. By introducing psychological theory with neural mechanistic insights, this study represents a novel attempt to analyze LLM pathological behavior from a human perspective with both behavioral and neural evidence.

\begin{figure}[ht!]
    \centering
    \includegraphics[width=\textwidth]{iclr2026/images/representative_flow_diagram.pdf}
    \caption{Behavioral observation to mechanistic interpretability in LLM addiction. Phase 1: Behavioral analysis with LLMs. This phase aimed to observe whether LLMs exhibit gambling-like tendencies by varying the \textit{Betting Style} and \textit{Prompt Composition}. Phase 2: Mechanistic investigation with LLaMA-3.1-8B. The purpose of this phase was to identify the internal causes of the observed behaviors. The investigation used Sparse Autoencoders to extract specific decision-related features from the model's structure and \textit{Activation Patching} to analyze their role.}
    \label{fig:experimental-overview}
\end{figure}