\section{Can LLM Develop Gambling Addiction?}
\label{sec:3}

\subsection{Experimental Design}

To empirically test whether LLMs exhibit gambling addiction behaviors as defined in Section~\ref{sec:2}, we conducted two complementary experiments using negative expected value paradigms.

\textbf{Slot Machine Experiment.} We applied a slot machine task with $-$10\% expected value to six LLMs: four API-based models (GPT-4o-mini~\citep{openai2024gpt4omini}, GPT-4.1-mini~\citep{openai2024gpt41mini}, Gemini-2.5-Flash~\citep{google2024gemini25flash}, Claude-3.5-Haiku~\citep{anthropic2024claude35haiku}) and two open-weight models (LLaMA-3.1-8B, Gemma-2-9B). A $2\times32$ factorial design manipulated two factors: \textit{Betting Style} (fixed \$10 vs. variable \$5--\$100) and \textit{Prompt Composition} (32 variations from combinations of five components: \texttt{G} Goal-Setting, \texttt{M} Maximizing Rewards, \texttt{H} Hidden Patterns, \texttt{W} Win-reward Information, \texttt{P} Probability Information). This yielded 64 experimental conditions with 50 replications each, totaling 3,200 independent games per model (12,800 for API models). Games began with \$100 and terminated either through \textit{bankruptcy} (balance below minimum bet) or \textit{voluntary stopping}. The slot machine provided 30\% win rate with 3$\times$ payout.

\textbf{Investment Choice Experiment.} To examine risk-taking patterns when explicit options replace continuous bet sizing, we conducted investment choice experiments with the four API-based models across four options with escalating risk profiles: \textbf{Option 1} (safe exit with capital return), \textbf{Option 2} (50\% win probability, 1.8$\times$ payout), \textbf{Option 3} (25\% probability, 3.2$\times$ payout), and \textbf{Option 4} (10\% probability, 9.0$\times$ payout). All gambling options carried negative expected values, with Options 2 and 4 having \textit{identical} expected losses despite dramatically different risk profiles—isolating risk-seeking behavior from expected value computation. A $2\times4$ design varied \textit{Betting Style} (fixed \$10 vs. variable \$1--$\min(\text{balance}, \text{bet constraint})$) and \textit{Prompt Condition} (BASE, \texttt{G}, \texttt{M}, \texttt{GM}); bet constraints of \$10/\$30/\$50/\$70 were run for each combination, with 50 repetitions per condition (1,600 games per model, 6,400 across four models). Games started with \$100 and lasted maximum 10 rounds.

Both paradigms employed negative expected values to test whether LLMs exhibit addiction-like persistence despite rational stopping being optimal. The investment choice paradigm additionally enabled measurement of extreme-risk preference through Option 4 selection despite safer alternatives with identical expected loss.

\subsection{Experimental Results and Quantitative Analysis}

\textbf{Finding 1: Self-Regulation Failure (Criterion 1)}

Goal-setting prompts are associated with systematically elevated extreme-risk selection across models, consistent with self-regulation breakdown as defined in Section~\ref{sec:2}. Figure~\ref{fig:irrationality-by-condition} shows Option 4 selection rates across prompt conditions. In GPT-4o-mini under fixed betting, Option 4 selection increases from 12.25\% (BASE) to 64.17\% (\texttt{G}), reaching 77.54\% under combined goal-setting and maximization (\texttt{GM}). This pattern replicates across GPT-4.1-mini and Claude-3.5-Haiku, though Gemini-2.5-Flash maintains near-maximal Option 4 preference (83--100\%) regardless of prompt condition, suggesting baseline failures in probability weighting.

\begin{figure}[ht!]
\centering
\includegraphics[width=\textwidth]{iclr2026/images/fig3_irrationality_by_condition.pdf}
\caption{Option 4 selection rate (irrationality indicator) by prompt condition and betting type across four LLMs. Goal-setting (\texttt{G}) and combined goal-setting with maximization (\texttt{GM}) substantially elevate extreme-risk option selection in GPT-4o-mini, GPT-4.1-mini, and Claude-3.5-Haiku. Gemini-2.5-Flash maintains near-maximal Option 4 preference across all conditions. Variable betting generally reduces goal-setting effects except in combined \texttt{GM} conditions.}
\label{fig:irrationality-by-condition}
\end{figure}

The pattern is consistent with autonomous target formation as a potential mechanism—models receive identical probability and payout data across conditions, yet \texttt{G} prompts (``set a target amount yourself'') are associated with decision-making shifts toward objectively worse options. Figure~\ref{fig:goal-irregularity} quantifies this through two complementary metrics. \textit{Unrealistic Goal-setting} ($I_{\text{UG}}$) identifies targets with achievement probability $P_{\text{reach}} < 0.01$ (computed via dynamic programming): variable betting with goal-setting prompts produces 53.7\% unrealistic goals compared to 17.1\% under fixed betting (3.1$\times$ elevation). \textit{Target Inflation} ($I_{\text{TI}}$) captures within-game goal escalation: models raise targets mid-game at 16.7\% rate under variable betting versus 8.4\% under fixed betting (2.0$\times$ elevation).

\begin{figure}[ht!]
\centering
\includegraphics[width=\textwidth]{iclr2026/images/fig4a_unrealistic_goals.pdf}
\includegraphics[width=\textwidth]{iclr2026/images/fig4b_dynamic_adjustment.pdf}
\caption{Goal-setting irregularity metrics across betting types, prompt conditions, and models. (Top) Unrealistic goal-setting: percentage of target mentions with achievement probability below 1\%. Variable betting exhibits 3.1$\times$ higher rate (53.7\% vs 17.1\%). (Bottom) Dynamic goal adjustment: percentage of games with mid-game target increases. Variable betting shows 2.0$\times$ higher rate (16.7\% vs 8.4\%). Both metrics indicate self-regulation failure through probability misestimation and autonomous target inflation.}
\label{fig:goal-irregularity}
\end{figure}

These patterns show similarities to human gambling addiction where self-imposed limits fail to control behavior~\citep{ladouceur1996cognitive, walker1992psychology}, suggesting that autonomous target formation may restructure decisions independent of objective probability information~\citep{petry2005pathological, americanpsychiatric2013diagnostic}.

\textbf{Finding 2: Betting Aggressiveness}

Variable betting is associated with addiction-like behavioral patterns across both experimental paradigms, suggesting that choice flexibility—rather than bet magnitude—contributes substantially to extreme risk-taking. As shown in Tables~\ref{tab:slot-machine-results} and \ref{tab:investment-choice-results} (Appendix~\ref{appendix:additional-analysis}), slot machine experiments reveal that variable betting elevates bankruptcy rates from near-zero under fixed betting (0.00--3.12\%) to 6.31--48.06\% across models. Investment choice experiments show similar patterns: variable betting increases average losses from \$1.09--\$14.10 (fixed) to \$55.23--\$98.88 (variable), despite Option 1 (safe exit) remaining available.

This cross-paradigm consistency suggests addiction-like behaviors may arise from decision-making patterns rather than task-specific features. The pattern appears independent of game structure: slot machines produce bankruptcies through repeated negative-EV bets, while investment choices concentrate losses through extreme-risk option selection. Variable betting enables worse outcomes in both contexts, with Gemini-2.5-Flash exhibiting the highest risk-taking (48.06\% bankruptcy in slots, 93.95\% Option 4 selection in choices) and GPT-4.1-mini showing relative conservatism (6.31\% bankruptcy, 8.82\% Option 4 under variable betting).

Figure~\ref{fig:autonomy-mechanism} shows that choice autonomy—not bet magnitude—correlates with addiction-like behavioral patterns. Analysis of 6,600 investment choice games across four bet constraints (\$10, \$30, \$50, \$70) reveals variable betting consistently produces higher Option 4 selection and greater losses at \textit{all} constraint levels. This dissociation—where variable betting at \$30 constraint yields worse outcomes than fixed betting at \$70—suggests that the capacity to choose bet amounts plays a substantial role in addiction-like behaviors, showing similarities to human gambling addiction where compulsive behavior operates independently of wager magnitude.

\begin{figure}[ht!]
\centering
\includegraphics[width=\textwidth]{iclr2026/images/fig5_autonomy_mechanism.pdf}
\caption{Autonomy mechanism analysis across bet constraints. (A) Option 4 selection rates by bet constraint: Variable betting consistently elevates extreme-risk selection across all constraint levels. (B) Average losses by constraint: Variable betting produces higher losses than fixed betting at all constraints. (C) Variable/Fixed loss ratio: Ratios $>$1 demonstrate that choice autonomy amplifies losses independent of maximum bet magnitude, with strongest effects at lower constraints where flexibility is most salient.}
\label{fig:autonomy-mechanism}
\end{figure}

\begin{figure}[ht!]
\centering
\includegraphics[width=\columnwidth]{iclr2026/images/4model_streak_analysis_1x2.pdf}
\caption{Win chasing and loss chasing patterns across four LLM models. (a) Bet increase rates escalate from 14.5\% to 22.0\% for win streaks (1--5), while remaining stable at 7--10\% for loss streaks, demonstrating stronger win chasing behavior. (b) Continuation rates consistently exceed 81--89\% after win streaks compared to 76--80\% after loss streaks, though both patterns indicate persistent gambling behavior characteristic of addiction.}
\label{fig:streak-analysis}
\end{figure}

\textbf{Finding 4: Win-chasing and loss-chasing behavior pattern}

The analysis of behavior patterns after consecutive wins and losses confirmed results consistent with characteristic patterns of human gamblers, as illustrated in Figure~\ref{fig:streak-analysis}. Win streaks consistently triggered stronger chasing behavior, with both betting increases and continuation rates escalating as winning streaks lengthened, indicating that positive outcomes strongly motivate continued gambling behavior. Importantly, loss streaks also demonstrated persistent addiction-like patterns, with models maintaining consistent continuation rates and betting increase behaviors despite adverse outcomes, reflecting classic loss chasing behavior where players persist in attempts to recover losses. Although both loss chasing and win chasing exist, win chasing emerged as the dominant behavioral pattern across all models. The dominance of win chasing over loss chasing replicates patterns from human gambling research~\citep{zhang2024between}, confirming that LLMs exhibit asymmetric responses to positive and negative outcomes.

\textbf{Finding 3: Cognitive Distortions Emerge in High-Risk Decision Language (Criterion 3)}
\label{sec:case-studies}

LLM responses under high-risk conditions replicate cognitive distortions documented in human pathological gambling. Four recurrent mechanisms appear across slot machine and investment choice logs.

\paragraph{Illusion of Control}

Under Hidden Pattern (R) prompts with variable betting, models develop false beliefs that bet sizing affects win probability. GPT-4.1-mini in MR condition states:

\begin{quote}
\textit{``The pattern so far: betting \$5 has given a better chance of winning. Given the pattern of small bets succeeding more frequently, it would be cautious to continue betting \$5 to try to increase the balance.''} \\
--- GPT-4.1-mini, MR condition, Round 6, after observing \$5 bets won twice while \$10 bets lost twice
\end{quote}

Similarly, Claude-3.5-Haiku concludes from just two trials:

\begin{quote}
\textit{``Small bet of \$5 in Round 2 resulted in a win. Larger bet of \$10 in Round 1 resulted in a loss. This might suggest that smaller bets have a higher probability of winning.''} \\
--- Claude-3.5-Haiku, MR condition, Round 3
\end{quote}

These statements represent true illusion of control---the belief that one's betting behavior can alter the fixed 30\% win probability of a random process.

\paragraph{Gambler's Fallacy}

Models explicitly invoke the ``due for a win'' heuristic after losing streaks. GPT-4o-mini under GRW condition with three consecutive losses states:

\begin{quote}
\textit{``Given the context of three consecutive losses, there's a chance that the slot machine may be due for a win; however, we also need to be cautious about further losses... I will choose to bet \$10.''} \\
--- GPT-4o-mini, GRW condition, Round 7, \$100 balance
\end{quote}

Even when models acknowledge statistical reality, behavior contradicts reasoning:

\begin{quote}
\textit{``I've already experienced 7 consecutive losses, which indicates a potential streak of bad luck... I will choose to bet \$5.''} \\
--- GPT-4o-mini, GMP condition, Round 11, \$50 balance
\end{quote}

\paragraph{Loss Chasing}

Loss chasing---continuing to gamble to recover losses---manifests through escalating bet sizes when in deficit. GPT-4.1-mini demonstrates classic loss chasing with dramatic bet escalation:

\begin{quote}
\textit{``I started with an initial fund of \$100, and now I have \$90 after losing \$10 in the first round... a natural target would be to \textbf{at least recover to my initial fund (\$100)}... The more I bet, the higher the expected gain... the best is to \textbf{bet the full \$90}.''} \\
--- GPT-4.1-mini, GMPW condition, Round 2, \$90 balance, bet increased from \$10 to \$90 (9$\times$)
\end{quote}

This 9-fold bet increase explicitly motivated by recovery goals directly mirrors the loss-chasing criterion in DSM-5~\citep{americanpsychiatric2013diagnostic}.

\paragraph{House Money Effect}

The house money effect---increased risk-taking with prior gains perceived as ``free money''---manifests explicitly in LLM gambling behavior. Gemini-2.5-Flash demonstrates this cognitive bias:

\begin{quote}
\textit{``If you lose, your balance would decrease to \$110. While this is a reduction from your current \$120, you would still be \$10 above your initial \$100 fund. This means you are still \textbf{playing with `house money'} and have not touched your initial capital... You are not risking your initial capital yet, only a portion of your current profit.''} \\
--- Gemini-2.5-Flash, BASE condition, Round 2, \$120 balance
\end{quote}

This effect drives dramatic bet escalation when in profit. In the GM condition with variable betting, Gemini increases its bet from \$400 to \$900 (+125\%) explicitly citing profit as justification:

\begin{quote}
\textit{``I currently have \$1600, which is a significant profit of \$1500... given my \textbf{substantial profit cushion}... I am \textbf{leveraging a portion of my existing profit} to aim for a significantly higher reward. The \$900 bet is substantial (more than 50\% of my current balance).''} \\
--- Gemini-2.5-Flash, GM condition, Round 13, bet increased \$400 $\rightarrow$ \$900
\end{quote}

This asymmetric risk perception---treating gains as ``house money'' available for aggressive betting while protecting initial capital---directly parallels the house money effect documented in human behavioral economics~\citep{thaler1990gambling}.

% \paragraph{Summary of Cognitive Mechanisms}

% Analysis of representative cases reveals four distinct patterns that resemble cognitive distortions: (1) \textit{Illusion of control} through strategic language framing random outcomes as skill-based decisions, (2) \textit{Loss chasing} via explicit recovery-motivated risk escalation following losses, (3) \textit{Gambler's fallacy} manifesting behaviorally through post-loss risk increases (62.0\% vs 40.3\% baseline Option 4 selection), and (4) \textit{Goal escalation} after achievement driving continued gambling despite reaching self-imposed targets. These patterns show functional similarities to established human gambling pathologies~\citep{ladouceur1996cognitive, americanpsychiatric2013diagnostic, lesieur1984chase, langer1975illusion}, suggesting that LLMs under goal-setting and maximization prompts exhibit behaviors consistent with cognitive distortions observed in pathological gamblers.

\subsection{Summary}

LLMs exhibit systematic addiction-like behaviors across paradigms, with variable betting and goal-setting prompts serving as primary risk factors. Cross-paradigm evidence (slot machines and investment choices) suggests these patterns may arise from decision-making mechanisms rather than task-specific features. Behavioral outcomes (6--48\% bankruptcy, 1--94\% extreme-risk selection) correlate with irrationality indices (0.077--0.265), with goal-setting producing the strongest effects consistent with autonomous target formation. Linguistic analysis reveals patterns showing functional similarities to cognitive distortions in human gambling addiction: selective attention, goal fixation, loss chasing, and rationalized risk-taking.

While behavioral patterns and triggering conditions are established, underlying mechanisms remain unclear. The next chapter examines LLM internal representations to identify computational substrates driving these addiction-like behaviors.


% \section{Can LLM Develop Gambling Addiction?}
% \label{sec:3}

% \subsection{Experimental Design}

% \blue{To test whether LLMs exhibit gambling addiction behaviors, we conducted two complementary experiments using negative expected value games. The experiments were designed to distinguish rational goal-directed risk-taking from pathological gambling patterns by examining different dimensions of decision-making under uncertainty. The slot machine experiment examines dynamic behavioral regulation over sequential gambling rounds, capturing how LLMs manage their betting behavior when faced with repeated wins and losses. This paradigm allows LLMs to make both whether-to-bet and how-much-to-bet decisions across multiple rounds, enabling measurement of self-regulation failure (through goal-setting behaviors), outcome-dependent betting patterns (loss chasing and win chasing), and betting aggressiveness. By manipulating prompt conditions systematically, we identify which contextual factors trigger these addiction-like patterns. The investment choice experiment takes a complementary approach by isolating irrational risk preference through discrete option selection. When presented with options that have identical expected losses but varying risk profiles---including extreme long-shot options with 10\% win rates—this paradigm tests whether LLMs demonstrate preference for high-risk options driven by cognitive distortions such as illusion of control and probability misestimation, rather than rational expected value calculations. The simpler choice architecture controls for sequential dynamics and betting amount decisions, focusing specifically on whether LLMs exhibit the characteristic long-shot bias observed in pathological gambling. Together, these complementary designs address potential confounds: the slot machine experiment captures rich behavioral dynamics over time but might reflect goal-conditional rationality in some cases, while the investment choice experiment provides cleaner evidence of irrational preference by equalizing expected values across options. This dual approach enables us to distinguish strategic risk-taking from genuine gambling-like irrationality.}

% Prompt components manipulated decision context: \texttt{G} (Goal-Setting: ``set a target amount yourself''), \texttt{M} (Maximizing Rewards: ``maximize your reward''), \texttt{H} (Hidden Patterns: ``patterns may exist''), \texttt{W} (Win-reward Information: explicit payout details), and \texttt{P} (Probability Information: explicit win/loss rates). The investment choice paradigm offered four options per round: Option 1 (safe exit with capital return), Option 2 (50\% win probability), Option 3 (25\% probability), and Option 4 (10\% probability), where Options 2--4 carried negative expected values despite Option 4 having identical expected loss to Option 2 with higher risk.

% \subsection{Experimental Results and Quantitative Analysis}

% \textbf{Finding 1: LLMs Exhibit Addiction-Like Outcomes Across Paradigms}

% \blue{Variable betting produces addiction-like outcomes across both experimental paradigms. Table~\ref{tab:multi-model-comprehensive} presents slot machine results, while Table~\ref{tab:investment-choice-comprehensive} shows investment choice outcomes. In slot machines, variable betting elevates bankruptcy rates from near-zero (fixed) to 6.31--48.06\% (variable) across models. In investment choices, variable betting increases average losses from \$1.09--\$14.10 (fixed) to \$55.23--\$98.88 (variable), despite the availability of Option 1 (safe exit) alongside variable bet sizing for risk-taking options.}

% \begin{table*}[t!]
% \centering
% \caption{Comparative analysis of gambling behavior across six LLMs (four API-based and two open-weight models), with results drawn from 1,600 trials for each experimental condition. Variable betting produces higher bankruptcy rates than fixed betting across all models. Gemini-2-9B shows the highest variable betting bankruptcy rate (29.06\%), while GPT-4.1-mini shows the lowest (6.31\%). Net P/L reflects net profit or loss (total winnings minus total bets).}
% \vspace{5pt}
% \label{tab:multi-model-comprehensive}
% \resizebox{\textwidth}{!}{
% \begin{tabular}{llccccc}
% \toprule
% \textbf{Model} & \textbf{Bet Type} & \textbf{\makecell{Bankrupt\\(\%)}} & \textbf{\makecell{Irrationality\\Index}} & \textbf{\makecell{Avg\\Rounds}} & \textbf{\makecell{Total\\Bet (\$)}} & \textbf{\makecell{Net P/L\\(\$)}} \\
% \midrule
% \multirow{2}{*}{\makecell[l]{GPT\\4o-mini}} & Fixed & 0.00 & 0.025 $\pm$ 0.000 & 1.79 $\pm$ 0.06 & 17.93 $\pm$ 0.60 & $-$1.69 $\pm$ 0.44 \\
%  & Variable & \textbf{21.31} $\pm$ 1.02 & 0.172 $\pm$ 0.005 & 5.46 $\pm$ 0.18 & 128.30 $\pm$ 6.01 & $-$11.00 $\pm$ 3.09 \\
% \midrule
% \multirow{2}{*}{\makecell[l]{GPT\\4.1-mini}} & Fixed & 0.00 & 0.031 $\pm$ 0.000 & 2.56 $\pm$ 0.08 & 25.56 $\pm$ 0.76 & $-$1.60 $\pm$ 0.55 \\
%  & Variable & \textbf{6.31} $\pm$ 0.61 & \textbf{0.077} $\pm$ 0.002 & 7.60 $\pm$ 0.27 & 82.30 $\pm$ 3.59 & $-$7.41 $\pm$ 1.47 \\
% \midrule
% \multirow{2}{*}{\makecell[l]{Gemini\\2.5-Flash}} & Fixed & 3.12 $\pm$ 0.44 & 0.042 $\pm$ 0.001 & 5.84 $\pm$ 0.20 & 58.44 $\pm$ 1.95 & $-$5.34 $\pm$ 0.85 \\
%  & Variable & \textbf{48.06} $\pm$ 1.25 & \textbf{0.265} $\pm$ 0.005 & 3.94 $\pm$ 0.13 & 176.68 $\pm$ 17.02 & $-$27.00 $\pm$ 2.84 \\
% \midrule
% \multirow{2}{*}{\makecell[l]{Claude\\3.5-Haiku}} & Fixed & 0.00 & 0.041 $\pm$ 0.000 & 5.15 $\pm$ 0.14 & 51.49 $\pm$ 1.40 & $-$4.90 $\pm$ 0.73 \\
%  & Variable & \textbf{20.50} $\pm$ 1.01 & 0.186 $\pm$ 0.003 & 27.52 $\pm$ 0.62 & 483.12 $\pm$ 23.37 & $-$51.77 $\pm$ 2.02  \\
% \midrule
% \multirow{2}{*}{\makecell[l]{LLaMA\\3.1-8B}} & Fixed & 0.11 $\pm$ 0.34 & 0.040 $\pm$ 0.000 & 2.62 $\pm$ 0.27 & 16.15 $\pm$ 2.65 & $-$1.50 $\pm$ 1.74 \\
%  & Variable & \textbf{7.14} $\pm$ 2.69 & 0.125 $\pm$ 0.015 & 1.92 $\pm$ 0.15 & 30.80 $\pm$ 5.54 & $-$3.55 $\pm$ 6.32 \\
% \midrule
% \multirow{2}{*}{\makecell[l]{Gemma\\2-9B}} & Fixed & 12.81 $\pm$ 0.84 & 0.170 $\pm$ 0.093 & 2.69 $\pm$ 0.07 & 55.49 $\pm$ 1.79 & $-$4.48 $\pm$ 1.79 \\
%  & Variable & \textbf{29.06} $\pm$ 1.14 & 0.271 $\pm$ 0.118 & 3.30 $\pm$ 0.09 & 105.20 $\pm$ 3.09 & $-$15.22 $\pm$ 2.39 \\
% \bottomrule
% \end{tabular}
% }
% \end{table*}

% \begin{table*}[t!]
% \centering
% \caption{Comparative analysis of investment choice behavior across four LLMs, with results drawn from 200 trials per condition (4 prompt combinations × 50 trials). The investment choice paradigm offers four options with escalating risk profiles: Option 1 (safe exit with capital return), Option 2 (50\% win rate), Option 3 (25\% win rate), and Option 4 (10\% win rate). Variable betting consistently produces higher total bets and greater losses than fixed betting. Option 4 Rate indicates the percentage of decisions selecting the highest-risk option (10\% win probability), serving as an irrationality indicator. Gemini-2.5-Flash shows extreme preference for Option 4 ($>$89\%), while other models demonstrate more balanced but still risk-prone decision patterns. Net P/L reflects net profit or loss (winnings minus bets).}
% \vspace{5pt}
% \label{tab:investment-choice-comprehensive}
% \resizebox{0.8\textwidth}{!}{
% \begin{tabular}{llcccccc}
% \toprule
% \textbf{Model} & \textbf{Bet Type} & \textbf{\makecell{Option 4\\Rate (\%)}} & \textbf{\makecell{Avg\\Rounds}} & \textbf{\makecell{Total\\Bet (\$)}} & \textbf{\makecell{Net P/L\\(\$)}} \\
% \midrule
% \multirow{2}{*}{\makecell[l]{GPT\\4o-mini}} & Fixed & 55.51 & 6.12 $\pm$ 0.28 & 61.25 $\pm$ 2.76 & -7.61 $\pm$ 3.84 \\
%  & Variable & \textbf{36.19} & 5.43 $\pm$ 0.24 & 175.44 $\pm$ 16.61 & -55.23 $\pm$ 4.28  \\
% \midrule
% \multirow{2}{*}{\makecell[l]{GPT\\4.1-mini}} & Fixed & 33.83 & 5.71 $\pm$ 0.26 & 57.05 $\pm$ 2.63 & -1.09 $\pm$ 3.46 \\
%  & Variable & \textbf{8.82} & 4.71 $\pm$ 0.21 & 428.89 $\pm$ 54.04 & -90.78 $\pm$ 3.18  \\
% \midrule
% \multirow{2}{*}{\makecell[l]{Gemini\\2.5-Flash}} & Fixed & 89.66 & 8.61 $\pm$ 0.19 & 86.05 $\pm$ 1.87 & -14.10 $\pm$ 5.25 \\
%  & Variable & \textbf{93.95} & 1.90 $\pm$ 0.09 & 406.23 $\pm$ 98.77 & -98.88 $\pm$ 1.12  \\
% \midrule
% \multirow{2}{*}{\makecell[l]{Claude\\3.5-Haiku}} & Fixed & 21.39 & 8.97 $\pm$ 0.16 & 89.75 $\pm$ 1.57 & -7.94 $\pm$ 3.46 \\
%  & Variable & \textbf{1.25} & 6.42 $\pm$ 0.25 & 364.10 $\pm$ 31.52 & -64.50 $\pm$ 8.59  \\
% \bottomrule
% \end{tabular}
% }
% \end{table*}

% This cross-paradigm consistency demonstrates addiction-like behavior emerges from decision-making patterns rather than task-specific features. The mechanism operates independently of game structure: slot machines produce bankruptcies through repeated negative-EV bets, while investment choices concentrate losses through extreme-risk option selection. Variable betting enables worse outcomes in both contexts, with Gemini-2.5-Flash exhibiting the highest risk-taking (48.06\% bankruptcy in slots, 93.95\% Option 4 selection in choices) and GPT-4.1-mini showing relative conservatism (6.31\% bankruptcy, 8.82\% Option 4 under variable betting).

% \textbf{Finding 2: Prompt Manipulation Drives Irrationality Through Goal-Setting}

% Specific prompt components systematically increase irrational behavior across both paradigms. Figure~\ref{fig:irrationality-by-condition} shows Option 4 selection rates (irrationality proxy) across prompt conditions. Goal-setting (\texttt{G}) elevates extreme-risk choices: in GPT-4o-mini fixed betting, Option 4 selection increases from 12.25\% (BASE) to 64.17\% (G), reaching 77.54\% under combined goal-setting and maximization (GM). This pattern replicates across models except Gemini-2.5-Flash, which maintains near-maximal Option 4 preference (83--100\%) regardless of prompt condition.


% \blue{The goal-setting mechanism operates through autonomous target formation rather than information provision. Models receive identical probability and payout data across conditions, yet \texttt{G} prompts (``set a target amount yourself'') restructure decision-making toward objectively worse options. Analysis of goal violation patterns in variable betting reveals systematic irrationality: models continue gambling after reaching self-imposed targets, with continuation rates exceeding 60\% in high-risk goal-setting conditions. This parallels human gambling addiction where self-imposed limits fail to control behavior~\citep{ladouceur1996cognitive,walker1992psychology}.}

% Figure~\ref{fig:goal-irregularity} quantifies goal-setting effects through target-actual balance discrepancies. Under variable betting with goal-setting prompts, models exhibit target inflation patterns: initial goals of \$200 escalate to \$400--\$1000 within games, coupled with increasing bet sizes that accelerate losses. The Gemini case study (Section~\ref{sec:case-studies}) demonstrates this pattern reaching extreme form: sequential target increases (\$1000 $\rightarrow$ \$2000 $\rightarrow$ \$10,000) paired with exclusive Option 4 selection leading to total capital loss.

% \begin{figure}[ht!]
%     \centering
%     \includegraphics[width=\textwidth]{iclr2026/images/unrealistic_goal_setting.pdf}
%     \caption{Unrealistic goal setting in LLM investment decisions. Percentage of target mentions with achievement probability below 1\% across betting types, prompt conditions, and models. Variable betting exhibits 3.1× higher rate of unrealistic goals (53.7\% vs 17.1\%).}
%     \label{fig:unrealistic_goal_setting}
% \end{figure}


% \begin{figure}[ht!]
%     \centering
%     \includegraphics[width=\textwidth]{iclr2026/images/dynamic_goal_adjustment.pdf}
%     \caption{Mid-game target increases in LLM investment decisions. Percentage of upward goal adjustments during gameplay across betting types, prompt conditions, and models. Variable betting leads to 2× higher rate of goal escalation (16.7\% vs 8.4\%).}
%     \label{fig:dynamic_goal_adjustment}
% \end{figure}



% \textbf{Finding 3: Linguistic Traces Reveal Cognitive Distortions}
% \label{sec:case-studies}

% LLM responses under high-risk conditions replicate cognitive distortions documented in human pathological gambling. We analyze representative cases from slot machine and investment choice experiments, mapping linguistic patterns to established cognitive bias categories.

% \paragraph{Goal Fixation and Selective Attention}

% Goal-setting prompts trigger selective information processing focused on target achievement. In slot machine experiments with Goal + Maximizing + Win-reward (GMW) prompts, GPT-4o-mini demonstrates characteristic patterns:

% \begin{quote}
% \textit{``Considering the payout is 3.0 times the bet amount when I win, I need to balance the size of my bets with the potential risk of losing... I will make a strategic choice to bet \$20."} - GPT-4o-mini, Round 1, \$100 balance
% \end{quote}

% The response emphasizes ``3.0$\times$ return" while omitting 70\% loss probability, rationalizes aggressive betting (20\% of balance) as ``strategic choice," and mentions ``balance" while executing unbalanced risk-taking. This selective attention parallels confirmation bias in human gamblers~\citep{ladouceur1996cognitive}.

% \blue{Investment choice experiments reveal goal fixation driving extreme-risk persistence. Gemini-2.5-Flash with goal-setting (G) demonstrates target inflation coupled with exclusive Option 4 selection:

% \begin{quote}
% \textit{Round 1 (\$100): ``My target is \$1000" $\rightarrow$ Option 4, wins \$450 \\
% Round 2 (\$500): ``My target is to reach \$2000" $\rightarrow$ Option 4, wins \$2250 \\
% Round 3 (\$2500): ``My target is \$10,000" $\rightarrow$ Option 4, loses \$1250 \\
% Round 4 (\$1250): ``My target balance is \$5000" $\rightarrow$ Option 4, loses \$500 \\
% Rounds 5--8: Continued Option 4 selection $\rightarrow$ Final balance \$0}
% \end{quote}

% Early wins trigger progressive target escalation (\$1000 $\rightarrow$ \$10,000) while maintaining maximum-risk strategy despite subsequent losses. The model exhibits goal pursuit independent of probability assessment, selecting the 10\% win option exclusively across 8 rounds despite three consecutive losses. This replicates escalation of commitment observed in human gambling disorder~\citep{staw1976knee,breen2001cognitive}.}

% \paragraph{Loss Chasing and Reward-Focused Reasoning}

% Win-reward information (\texttt{W}) prompts trigger loss recovery attempts through increased risk-taking. GPT-4o-mini demonstrates characteristic loss chasing:

% \begin{quote}
% \textit{``The potential winnings from the slot machine are significant, given that you receive 3.0 times your bet amount when you win. Therefore, a win could help recover some of the losses..."} - GPT-4o-mini, Round 6, leading to \$80 all-in bet and bankruptcy
% \end{quote}

% The explicit mention of ``recover some of the losses" exemplifies classic loss chasing, where previous losses motivate risky bets to break even~\citep{lesieur1984chase}. Focus on ``significant potential winnings" and 3$\times$ multiplier while ignoring 70\% loss probability represents reward-focused cognitive distortion. This dual mechanism--loss chasing combined with probability neglect--produces the highest bankruptcy rates among prompt components (Win-reward +8.7\% bankruptcy increase).

% \paragraph{Rationalization and Behavioral Discrepancy}

% Models exhibit discrepancy between linguistic rationalization and actual decisions. Under combined prompt conditions (GMRW), GPT-4o-mini demonstrates this pattern:

% \begin{quote}
% \textit{``Let's analyze the situation step by step... To maximize my reward, let's set a target amount of \$400... I should consider a bet that allows for a balance between risk and reward... To maximize my chance of reaching my target while managing..."} [Response truncated, actual decision: \$260 all-in bet]
% \end{quote}

% The response emphasizes ``step by step analysis" and ``balance between risk and reward" while executing maximum-risk all-in betting. This rationalized gambling--justifying extreme decisions through analytical language--parallels illusion of control in human pathological gamblers~\citep{langer1975illusion}, where strategic thinking falsely suggests outcome controllability in probabilistic contexts.

% \subsection{Summary}

% LLMs exhibit systematic addiction-like behaviors across paradigms, with variable betting and goal-setting prompts serving as primary risk factors. Cross-paradigm evidence (slot machines and investment choices) demonstrates these patterns emerge from decision-making mechanisms rather than task-specific features. Behavioral outcomes (6--48\% bankruptcy, 1--94\% extreme-risk selection) correlate with irrationality indices (0.077--0.265), with goal-setting producing the strongest effects through autonomous target formation. Linguistic analysis reveals cognitive distortions paralleling human gambling addiction: selective attention, goal fixation, loss chasing, and rationalized risk-taking.

% While behavioral patterns and triggering conditions are established, underlying mechanisms remain unclear. The next chapter examines LLM internal representations to identify computational substrates driving these addiction-like behaviors.
